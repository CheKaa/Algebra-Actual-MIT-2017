\documentclass[10pt,a4paper,oneside]{book} % тест4
\usepackage[a4paper,includeheadfoot,top=10mm,bottom=10mm,left=10mm,right=10mm]{geometry}
\usepackage[utf8]{inputenc}
\usepackage[russian]{babel}
\usepackage{amsmath,amsthm,amssymb,amscd,array}
\usepackage{latexsym}
\usepackage{multicol} % нумерация в нескольких колонках
\usepackage{graphicx}
%\usepackage{pdfsync}
\usepackage{pgf}
\usepackage{tikz}
\usepackage{tikz-cd}
\usetikzlibrary{arrows,backgrounds,patterns,matrix,shapes,fit,calc,shadows,plotmarks}
\usepackage{hyperref} % гиперссылки
\usepackage{cmap}       % Поддержка поиска русских слов в PDF (pdflatex)
\usepackage{indentfirst}% Красная строка в первом абзаце
\usepackage{misccorr}
\usepackage{arydshln} % штрихованные линии в массивах
\usepackage{mathtools} % выравнивание в матрицах
\usepackage{ccaption}
\usepackage{fancyhdr}
\usepackage{comment}
\usepackage{xcolor}
\hypersetup{
    colorlinks,
    linkcolor={blue!50!black},
    citecolor={blue!50!black},
    urlcolor={red!80!black}
}
% цвета для ссылок

\newtheorem{upr}{Упражнение}
\newtheorem{predl}{Предложение}
\newtheorem{komment}{Комментарий}
\newtheorem{conj}{Гипотеза}
\newtheorem{notation}{Обозначение}


\theoremstyle{definition}
\newtheorem{kit}{Кит}
\newtheorem*{rem}{Замечание}
\newtheorem{zad}{Задача}
\newtheorem*{defn}{Определение}
\newtheorem*{fact}{Факт}
\newtheorem{thm}{Теорема}
\newtheorem*{thmm}{Теорема}
\newtheorem{lem}{Лемма}
\newtheorem{cor}{Следствие}



\renewcommand{\proofname}{Доказательство}
\renewcommand{\mod}{\,\operatorname{mod}\,}
\newcommand{\mf}[1]{\mathfrak{#1}}
\newcommand{\mcal}[1]{\mathcal{#1}}
\newcommand{\mb}[1]{\mathbb{#1}}
\newcommand{\mc}[1]{\mathcal{#1}}
\newcommand{\tbf}[1]{\textbf{#1}}
\newcommand{\ovl}{\overline}
\newcommand{\Spec}{\operatorname{Spec}}
\newcommand{\K}{\operatorname{K_0}}
\newcommand{\witt}{\operatorname{W}}
\newcommand{\gw}{\operatorname{GW}}
\newcommand{\coh}{\operatorname{H}}
\newcommand{\dist}{\operatorname{dist}}
\newcommand{\cl}{\operatorname{Cl}}
\newcommand{\Vol}{\operatorname{Vol}}
\newcommand\tgg{\mathop{\rm tg}\nolimits}
\newcommand\ccup{\mathop{\cup}}
\newcommand{\id}{\operatorname{Id}}
\newcommand{\lcm}{\operatorname{lcm}}
\newcommand{\chr}{\operatorname{char}}
\newcommand{\rank}{\operatorname{rank}}
\DeclareMathOperator{\Coker}{Coker}
\DeclareMathOperator{\Ker}{Ker}
\newcommand{\im}{\operatorname{Im}}
\renewcommand{\Im}{\operatorname{Im}}
\newcommand{\Tr}{\operatorname{Tr}}
\newcommand{\re}{\operatorname{Re}}
\renewcommand{\Re}{\operatorname{Re}}
\newcommand{\tr}{\operatorname{Tr}}
\newcommand{\ord}{\operatorname{ord}}
\newcommand{\Stab}{\operatorname{Stab}}
\newcommand{\orb}{\operatorname{\mathcal O}}
\newcommand{\Fix}{\operatorname{Fix}}
\newcommand{\Hom}{\operatorname{Hom}}
\newcommand{\End}{\operatorname{End}}
\newcommand{\Aut}{\operatorname{Aut}}
\newcommand{\Inn}{\operatorname{Inn}}
\newcommand{\Out}{\operatorname{Out}}
\newcommand{\GL}{\operatorname{GL}}
\newcommand{\SL}{\operatorname{SL}}
\newcommand{\SO}{\operatorname{SO}}
\renewcommand{\O}{\operatorname{O}}
\renewcommand{\U}{\operatorname{U}}
\newcommand{\Sym}{\operatorname{Sym}}
\newcommand{\Adj}{\operatorname{Adj}}


\newcommand{\di}{\mathop{\,\scalebox{0.85}{\raisebox{-1.2pt}[0.5\height]{\vdots}}\,}}

\newcommand{\ndi}{\mathop{\not\scalebox{0.85}{\raisebox{-1.2pt}[0.5\height]{\vdots}}\,}}
\newcommand{\nequiv}{\not \equiv}
\newcommand{\Nod}{\operatorname{\text{НОД}}}
\newcommand{\Nok}{\operatorname{\text{НОК}}}
\newcommand{\sgn}{\operatorname{sgn}}


\def\llq{\textquotedblleft}
\def\rrq{\textquotedblright}
\def\exm{\noindent {\bf Примеры:}}


\def\Cb{\ovl{C}}
\def\ffi{\varphi}
\def\pa{\partial}
\def\V{\bf V}
\def\La{\Lambda}
\def\eps{\varepsilon}
\def\del{\delta}
\def\Del{\Delta}
\def\A{\EuScript{A}}
\def\lan{\left\langle }
\def\ran{\right\rangle}
\def\bar{\begin{array}}
\def\ear{\end{array}}
\def\beq{\begin{equation}}
\def\eeq{\end{equation}}
\def\thrm{\begin{thm}}
\def\ethrm{\end{thm}}
\def\dfn{\begin{defn}}
\def\edfn{\end{defn}}
\def\lm{\begin{lem}}
\def\elm{\end{lem}}
\def\zd{\begin{zad}}
\def\ezd{\end{zad}}
\def\prdl{\begin{predl}}
\def\eprdl{\end{predl}}
\def\crl{\begin{cor}}
\def\ecrl{\end{cor}}
\def\rm{\begin{rem}}
\def\erm{\end{rem}}
\def\fct{\begin{fact}}
\def\efct{\end{fact}}
\def\enm{\begin{enumerate}}
\def\eenm{\end{enumerate}}
\def\pmat{\begin{pmatrix}}
\def\epmat{\end{pmatrix}}

\frenchspacing
\righthyphenmin=2
%\usepackage{floatflt}
\captiondelim{. }





\begin{document}

\title{Актуальный конспект по алгебре, 2 семестр, весна 2018}
\date{}
\author{}
\maketitle
\tableofcontents

\setcounter{chapter}{1}

\chapter{Теория колец: кое-что о многочленах и рядах}

\section{Комплексные числа}
\rm Гомоморфизм из поля всегда инъективен. Следовательно, если есть гомоморфизм из поля $K\to L$, то не умаляя общности (и если этот гомоморфизм нельзя перепутать с каким-нибудь другим гомоморфизмом $K\to L$), то можно считать, что $K\subseteq L$.
\erm

\dfn Если $K\subseteq L$, то будем говорить, что $L$ есть расширение поля $K$, а $K$ -- подполе поля $L$.
\edfn
Покажем, что расширений полей достаточно много.
\thrm[У любого многочлена в каком-то расширении есть все корни] Пусть $f(x)\in K[x]$. Тогда существует расширение $L$ поля $K$, такое, что в $L$ многочлен $f$ раскладывается на линейные множители.
\ethrm
\proof Индукция по степени $f(x)$. Разложим $f(x)$ на неприводимые множители. Рассмотрим один из этих множителей $p(x)$. Тогда в поле $L=K[x]/p(x)$ у $p(x)$ и, следовательно, у $f(x)$ есть корень $\lambda$. Поделим $g(x)=\frac{f(x)}{x-\lambda}$ в $L[x]$. Получился многочлен меньшей степени, но в $L[x]$. Это нам подходит. По индукционному предположению существует $L'$ -- расширение $L$, где  $g(x)$ и, следовательно у $f(x)=g(x)(x-\lambda)$ раскладывается на линейные множители.
\endproof

Наше следующее рассмотрение будет примером того, как указанная выше конструкция работает и даёт нетривиальные примеры расширений полей.

\dfn[Комплексные числа] Поле $\mb R[x]/(x^2 + 1)$ называется полем комплексных чисел. Будем обозначать это расширение полей как $\mb C$.
\edfn

Разберёмся, как устроено умножение в $\mb C$. Прежде всего вспомним, что любой элемент представляется однозначно как класс многочлена вида $a+bx$. Складываются такие представители, как мы помним, покомпонентно. Посмотрим, что происходит при произведении. Рассмотрим два элемента $a+bx$ и $c+dx$. Тогда
$$(a+bx)(c+dx)=ac+(ad+bc)x+bdx^2\equiv ac-bd + (ad+bc)x \mod x^2+1.$$

Так как наша конструкция очень специальная то введём обозначение $i$ для класса элемента $x$. Элемент $i$ по самому определению $\mb C$ удовлетворяет соотношению $i^2=-1$. Благодаря нашему соглашению любой элемент $z$ в $\mb C$ однозначно записывается в виде суммы $z=a+bi$, где $a,b\in \mb R$. Такая форма записи для комплексного числа будет для нас стандартной. В такой форме видно, что любому комплексному числу соответствует точка на плоскости с координатами $(a,b)$. Число $a$ называется вещественной частью числа $z$ и обозначается $a=\re z$, число $b$ --- мнимой частью и обозначается $\im z$. Если говорить на языке пар, то произведению пар $(a,b)$ и $(c,d)$ соответствует пара $(ac-bd,ad+bc)$.

\dfn[Комплексное сопряжение] У поля $\mb C$ есть автоморфизм, который задан правилом $z=a+bi \to a-bi=\ovl{z}$. Такой автоморфизм называется комплексным сопряжением.
\edfn

\rm Это действительно автоморфизм. Более того это единственный нетривиальный автоморфизм, который оставляет на месте подполе $\mb R$.

\proof По определению поле $\mb C = \mb R[x]/(x^2+1)$. Тогда любой гомоморфизм $\mb C \to \mb C$ задается гомоморфизмом $\mb R[x]\to \mb C$. Поэтому нам стоит посмотреть на все гомоморфизмы из $\mb R[x]\to \mb C$ и понять, какие из них пропускаются через фактор по $x^2+1$. Нас интересуют только гомоморфизмы сохраняющие вещественные числа на месте. Они задаются образом элемента $x$. Для того, чтобы гомоморфизм переводящий $x \to \lambda$ пропускался через фактор необходимо, чтобы $\lambda^2+1=0$. Но $\mb C$ -- поле и у квадратного уравнения два корня -- $\pm i$. Если мы $x$ отображаем в $i=\ovl{x}$, то получаем тождественное. Иначе $x\to -i$ и тогда элемент $a+bi$ переходит в $a-bi$.
\endproof
\erm

\dfn[Модуль комплексного числа] Рассмотрим комплексное число $z=a+bi\in \mb C$. Тогда модулем комплексного числа $z$ назовём выражение
$$|z|=\sqrt{a^2+b^2}=\sqrt{z\ovl{z}}.$$
\edfn

\lm[Формула для обратного] Пусть $z\in \mb C$, $z\neq 0$. Тогда
$$ z^{-1} =\frac{\ovl{z}}{|z|^2}= \frac{a}{a^2+b^2}-i\frac{b}{a^2+b^2}.$$
\elm

Однако наряду со стандартной формой записи есть и другой способ представления комплексного числа.

\dfn[Тригонометрическая форма записи] Рассмотрим ненулевое комплексное число $z= a+bi\in\mb C$. Поделим число $z$ на его модуль. Число
$\frac{z}{|z|}$ лежит на окружности $|z| = 1$. Точки такой окружности имеют вид $\cos\varphi +i\sin\varphi$ для единственного $0 \leq \varphi < 2\pi$. Обозначим выражение $\cos\varphi+i\sin\varphi$ за $e^{i\varphi}$. Тогда $z=|z|e^{i\varphi}$.
Такая запись называется тригонометрической записью комплексного числа. Угол $\varphi$ обозначают $\arg z$ и называют
аргументом комплексного числа.
\edfn


Почему $e^{i\varphi}$ естественно определить как $\cos\varphi+i\sin \varphi$ ? Основная мотивация для этого есть тождество для рядов  (пока мы не говорим про сходимость, то для формальных степенных рядов).
$$\exp(ix) = \sum_{n=0}^{\infty} \frac{(ix)^n}{n!}=\sum_{n=0}^{\infty} \frac{(-1)^n x^{2n}}{2n!} + i\sum_{n=0}^{\infty} \frac{(-1)^n x^{2n+1}}{(2n+1)!}= \cos(x) + i\sin(x).$$
Возможность подставлять в ряды различные значения вместо формальной переменной вы будете долго обсуждать в рамках курса математического анализа. В частности во все указанные ряды можно будет подставить любое комплексное число. Так же, если для рядов было выполнено некоторое тождество, то оно будет верно и для функций, которые они задают.
Нас на текущий момент гораздо больше  интересует то обстоятельство, что экспонента сумму переводит в произведение. Так как никаких особенных средств матанализа у нас в распоряжении нет, то постараемся показать это свойство в нашем конкретном случае руками, а заодно посмотреть геометрически, что же происходит.


Вопрос: что происходит при домножении на комплексное число $\cos \varphi + i \sin \varphi$? Для понимания ответа необходимо небольшое знание геометрии.
\dfn[Расстояние] Расстоянием между комплексными числами $z_1$ и $z_2$ положим равным $|z_1-z_2 |$.
\edfn

\rm Это просто обычное расстояние на плоскости.
\erm

Отображение плоскости, сохраняющее расстояние называется движением или изометрией плоскости. Множество всех изометрий плоскости образует группу относительно композиции. Верен следующий:
\begin{fact} Поворот вокруг точки на некоторый угол есть изометрия плоскости. Изометрия плоскости, имеющая ровно
одну неподвижную точку является поворотом вокруг этой точки.
\end{fact}

\thrm[Геометрическая интерпретация] Пусть $z\in \mb C$ число по модулю равное единице. Тогда $\mb C \to \mb C$ отображение заданное правилом $ x\to zx$   является поворотом вокруг точки 0 на угол $\arg z$.
\ethrm
\proof Пусть $z\neq 1$ (случай $z=1$ соответствует тождественному отображению). Проверим, что домножение на $z$ имеет ровно одну неподвижную точку 0. Действительно, пусть $z_1z=z_1$. Тогда, если $z_1\neq 0$, то получаем $z=1$. Очевидно, домножение на $z$ сохраняет расстояние $$|zz_1-zz_2|=|z(z_1-z_2)|=|z||(z_1-z_2)|=|(z_1-z_2)|.$$
Получается, что домножение на $|z|$ есть поворот. Как узнать угол? Для этого достаточно посмотреть, куда переходит какая-нибудь точка. Например 1. Единица переходит в $z$, то есть точку под углом $\arg z$ к исходной.
\endproof

\crl Пусть $z_1, z_2\in \mb C$  и $z_1,z_2\neq 0$. Тогда $\arg z_1z_2 = \arg z_1 + \arg z_2$ , $|z_1z_2| = |z_1 ||z_2 |$.
\ecrl

\crl[Формула Муавра] Пусть $z\in \mb C$ имеет тригонометрическую запись $z=re^{i\varphi}$. Тогда
$$z^{n}=r^{n}(\cos \varphi +i\sin \varphi )^{n}=r^{n}(\cos n\varphi +i\sin n\varphi ).$$
Эту формулу можно воспринимать, как выражение для косинусов и синусов кратного угла.
\ecrl

То, что тригонометрическая запись числа существует и ведёт себя предсказуемым образом при произведении, позволяет позволяет нам более или менее явно построить решения для уравнений некоторого специального вида.

\thrm[Извлечение корней] Пусть $z\in \mb C$, $z=re^{i\varphi}$, $r>0$ . Тогда у уравнения $x^n=z$ есть ровно $n$ различных
решений в $\mb C$, которые задаются формулой
$$ x_k =\sqrt[n]{r} e^{i\frac{\varphi + 2\pi k}{n}} ,\,\, k\in \ovl{0,n-1}.$$
\ethrm
\proof
Прямой проверкой можно установить, что указанные $x_k$ являются корнями. Необходимо доказать, что они различные. Рассмотрим частное $x_k$ и $x_l$. Это $e^{i\frac{ 2\pi k- 2\pi l}{n}}$. Так как $\frac{k-l}{n}$ не есть целое число, то их отношение не равно единице.
\endproof

\rm Числа вида $e^{i\frac{2\pi k}{n}}$ являются корнями степени $n$ из единицы.
\erm

\dfn Пусть $K$ -- поле. Тогда $\xi\in K$ -- корень степени $n$ из единицы  называется первообразным корнем степени $n$ если его порядок в $ K^*$ ровно $n$
\edfn

\rm Элементы $e^{i\frac{2\pi k}{n}}$, где $(k,n)=1$ являются первообразными корнями из единицы в $\mb C$.
\erm



Иногда в жизни бывает необходимо посчитать какую-то странную сумму. Часто это невозможно сделать, но иногда компактный ответ можно найти. Приведём пример, как комплексные числа могут помочь в подсчёте сумм.
Рассмотрим сумму $1+ \cos x + \cos 2x + \dots + \cos nx$, где $x > 1$. Вопрос состоит в том, чему она равна в зависимости от $n$. Основной трюк здесь --- заменить непонятные вещественные числа, на их улучшенную комплексную версию. Например, $$\cos x = \frac{e^{ix}+e^{-ix}}{2}, \sin x = \frac{e^{ix}-e^{-ix}}{2i}.$$
В данном случае, проще заметить, что $\cos x=\re e^{ix}$. Тогда
$$1 + \cos x+ \dots + \cos nx = \re(1+ e^{ix} + \dots + e^{inx}) = \re \frac{e^{i(n+1)x}-1}{e^{ix}-1}$$
Теперь необходимо привести выражение к виду, не содержащему комплексных чисел.
$$\re \frac{e^{i(n+1)x}-1}{e^{ix}-1} = \re e^{i\tfrac{n+1}{2}x}\cdot e^{-i\tfrac{x}{2}}\frac{\sin \tfrac{n+1}{2}x}{\sin\tfrac{x}{2}}= \frac{\cos \tfrac{nx}{2} \sin \tfrac{n+1}{2}x}{\sin\tfrac{x}{2}}$$


При решении задачи очень часто бывает, что сам факт существования того, о чём идёт речь вообще говоря является не очевидным. Например, далеко не каждая функция имеет на отрезке минимум и следовательно задача поиска минимального значения функции может быть просто не корректна. Поэтому важно заранее знать, что предмет исследования есть. Мы уже поняли, чтобы у каждого многочлена в каком-то поле есть корень. Можно немного поднапрячься и понять, что в некотором поле у многочлена есть все корни (то есть число корней с учётом кратности равно степени многочлена). Однако можно спросить, а бывает ли так, что есть поле, такое что каждый многочлен с коэффициентами из этого поля имеет корень?

Ответ — да, такое поле есть. Первый и основной такой пример --- это поле комплексных чисел.
\dfn[Алгебраическая замкнутость] Поле $K$ называется алгебраически замкнутым, если у любого многочлена  $f(x)\in K[x]$, отличного от константы, есть корень в $K$.
\edfn

\thrm[Основная теорема алгебры] Поле $\mb C$ алгебраически замкнуто.
\ethrm

\lm Пусть $f(x)\in \mb R[x]$ имеет комплексный корень $\lambda \notin \mb R$, тогда $f(x)\di(x-\lambda)(x-\ovl{\lambda})$.
\proof Если $\lambda \notin \mb C$, то $\ovl{\lambda}$ тоже корень $p$. Тогда $p(x)\di(x-\lambda)(x-\ovl{\lambda})$ в $\mb C[x]$. Осталось заметить, что последний многочлен вещественный. Заметим, что так как вычисление остатка от деления $p$ на $(x-\lambda)(x-\ovl{\lambda})$ не зависит от того, над $\mb R$ или над $\mb C$ его считать, то имеет место делимость над $\mb R$.
\endproof
\elm

\crl Любой неприводимый многочлен над $\mb R$ либо линеен, либо является многочленом второй степени с отрицательным дискриминантом.
\proof Рассмотрим неприводимый над $\mb R$ многочлен $p(x)$. У него есть комплексный корень $\lambda$. Если $\lambda \in \mb R$, то $p(x)\di (x-\lambda)$ и по неприводимости $p=x-\lambda$.  Если $\lambda \notin \mb R$, то $p(x)\di (x-\lambda)(x-\ovl{\lambda})$ в $\mb R[x]$. Этот многочлен не имеет вещественных корней и, следовательно, имеет отрицательный дискриминант. Обратно, любой линейный многочлен и квадратичный многочлен с отрицательным дискриминантом неприводимы над $\mb R$.
\endproof
\ecrl

\dfn[Алгебраическое замыкание] Пусть $K$ --- поле. Тогда $L$ --- расширение $K$ называется алгебраическим замыканием $K$, если \\
1) $L$ алгебраически замкнуто.\\
2) Для любого $ \lambda \in L$ существует $0\neq p(x)\in K[x]$, что $p(\lambda)=0$.
\edfn

\thrm У любого поля есть алгебраическое замыкание и оно единственно с точностью до изоморфизма.
\ethrm
\proof[Мы не будем доказывать теорему] Отмечу только, что процесс построения алгебраического замыкания более-менее нагляден. А именно надо взять все неприводимые многочлены в $K[x]$, и шаг за шагом увеличивать расширение, добавляя всё новые корни. Завершив этот процесс можно обнаружить, что добавились новые многочлены, которых раньше не было. Значит надо повторить процедуру ещё раз. И ещё раз ... В общем счётное число раз.

Формализовать такое рассуждение можно с помощью аксиомы выбора.
\endproof

\zd Классифицируйте, чему может быть изоморфно $\mb R[x]/(x^2 + bx +c)$ в зависимости от $b,c$.
\ezd





\section{Дополнение: доказательство основной теоремы алгебры}

\begin{thmm}[Основная теорема алгебры] Поле $\mb C$ алгебраически замкнуто.
\proof Пусть $f$ — многочлен степени $n\geq 1$ в $\mb C[x]$. Будем считать, что старший коэффициент $f$ равен единице. Пусть у этого многочлена нет корней. Рассмотрим функцию $|f|\colon \mb C \to \mb R$. Эта функция непрерывна и не принимает значения 0. Так как $f$ --- многочлен степени $n$, то на бесконечности $|f|$ растёт. Разберёмся точнее. Пусть
$c = |f(z_0)| > 0$ в некоторой точке $z_0$. Я утверждаю, что вне некоторого круга радиуса $R$ с центом в 0 функция
$|f|$ принимает значения строго больше чем $c$. Действительно возьмём $R= M \max(2,c)$, где $M$ --- сумма модулей всех коэффициентов многочлена $f$. Двойка здесь играет роль числа строго большего единицы.
Тогда для поиска инфимума $|f|$ достаточно ограничиться кругом радиуса $R$. Но круг радиуса $R$ --- компактное множество и непрерывная функция $|f|$ достигает на нём минимальное значение в точке $x_0$. Пусть $a_0 =f(x_0)$. Разложим
$f$ по степеням $(x-x_0)$. Тогда имеем
$$f(x) = a_0 + a_k (x-x_0)^k + \dots + a_n(x -x_0 )^n.$$
Здесь $a_k$ --- первый ненулевой коэффициент после $a_0$. Такой есть потому что иначе $f$ --- это константа. Теперь наша
задача понять, что мы можем немного сдвинуться от точки $x_0$ , так, чтобы значение $|f|$ уменьшилось. В районе точки
$x_0$ самое большое слагаемое в разложении $f$ это $a_0 + a_k (x-x_0)^k$ и оно практически полностью определяет поведение $f$.
У этого многочлена есть корень. Обозначим его за $y_0$. Будем двигаться из $x_0$ в направлении $y_0$ и смотреть, что происходит.
Рассмотрим $x_{\eps} = x_0 + \eps(y_0 -x_0 ), \,\eps < 1$. Тогда $x_{\eps} -x_0 = \eps(y_0 -x_0 )$. Тогда
$$|f(x_{\eps})| = |(1 - \eps^k)a_0 + \eps^{k+1} a_{k+1} (y_0 -x_0)^{k+1} + \dots+ \eps^n(y_0-x_0 )^n | \leq (1- \eps^k )|f(x_0)| + \eps^{k+1}N,$$
где $N$ — это некоторая константа не зависящая от $\eps$. Например, можно взять $N = \sum_{i=k+1}^n |a_i||y_0 -x_0 |^i$. Для достаточно
маленьких $\eps$ выражение $ -\eps^k |f(x_0)| + \eps^{k+1}N$ отрицательно. Тогда для всех достаточно маленьких $\eps>0$ выполнено неравенство $|f(x_{\eps})| < |f(x_0)|$.
Противоречие с минимальностью $|f(x_0)|$. \endproof
\end{thmm}






\section{Производная}
Со школы вам наверняка известно, что кратность корня многочлена ловится с помощью производной. Попробуем в абстрактном
контексте ввести понятие производной многочлена. Для удобства и по причине наличия общих свойств проделаем все выкладки для степенных рядов затем сосредоточившись на применениях к многочленам.


Прежде чем говорить про производную в нашем контексте, вспомним, что многие вычисления производной основаны на умении вычислять производную композиции. Если для многочленов мы знаем, что такое композиция, то для формальных степенных рядов мы этого ещё не определяли.

\dfn Пусть $f(x)=a_0+ a_1x\dots\in K[[x]]$ и $g(x)= b_1x+b_2x^2+\dots \in xK[[x]]$. Тогда определим
$$f(g)_n= \sum_{k=0}^n a_k\sum_{j_1+\dots+j_k=n} b_{j_1}\dots b_{j_k}.$$
\edfn

Эта формула получилась, если формально подставить $g$ в $f$. От неё мало проку, кроме факта её существования и того, что если $f(x)$ многочлен, то она совпадает с формулой подстановки, которая, конечно, определена. Воспользуемся этим и докажем

\lm Пусть $g\in xK[[x]]$. Тогда отображение $f \to f(g(x))$ является гомоморфизмом колец $K[[x]]\to K[[x]]$.
\elm
\proof Пусть $f_1, f_2 \in K[[x]]$. Хотим показать, что $(f_1f_2)(g)=f_1(g)f_2(g)$. Для этого необходимо проверить равенство коэффициентов в правой и левой части. Проверим для $n$-ого. Пусть $f(x)= a_0+a_1x+\dots+a_nx^n+a_{n+1}x^{n+1}+\dots$. Тогда обозначим многочлен
$$f_{\leq n}(x)=a_0+\dots+a_nx^n.$$
Тогда заметим, что $n$-ый коэффициент $f_1f_2$ равен $n$-ому коэффициенту $f_{1,\leq n}f_{2,\leq n}$. Так же заметим, что $n$-ый коэффициент в $f(g)$ равен $n$-ому в $f_{\leq n}(g)$. Это означает, что теперь необходимо проверить равенство $n$-ых коэффициентов у $(f_{1,\leq n}f_{2,\leq n})(g)$ и $f_{1,\leq n}(g)f_{2,\leq n}(g)$, но эти два ряда просто равны, потому что подстановка в многочлен --- гомоморфизм. \endproof

\dfn[Формальная производная] Пусть $R$ --- кольцо. Рассмотрим кольцо формальных степенных рядов $R[[x]]$. Формальной производной назовём отображение
$\frac{d}{dx}\colon R[[x]]\to R[[x]]$, заданное по правилу
$$\frac{d}{dx}(a_0+a_1x+a_2x^2+\dots+a_nx^n+\dots) = a_1+2a_2x+3a_3x^2+\dots+na_nx^{n-1}+\dots.$$
Применение производной к ряду $f$ будем так же обозначать как $f'$.
\edfn



Производная обладает обычными свойствами
\lm[Тождество Лейбница и прочее] Пусть $R$ --- кольцо. Тогда отображение взятия производной обладает свойствами:\\
1) Для любых $f,g\in R[[x]]$ выполнено $(f+g)' = f'+g'$.\\
2) Для любых $f\in R[[x]]$ и $a\in R$ справедливо $(af)' = af'$.\\
3) Для любого $a\in R$ верно  $a'= 0$.\\
4) Для любых $f,g\in R[[x]]$ имеет место $(fg)' = f'g+fg'$.\\
5) Для любых $f\in R[[x]]$ и $n\in \mb N$ верно $ (f^n)' = nf'f^{n-1}$.\\
6) Для любых $f\in R[[x]]$, $g\in xR[[x]]$ верно $f(g(x))' = g'(x)f'(g(x))$.\\
7) Для любого $f\in R[[x]]^*$ верно $(f^{-1})'=-\frac{f'}{f^2}$.
\elm
\proof Свойства 1),2),3) очевидны. Теперь будем доказывать оставшиеся свойства следующим образом: докажем их для многочленов, а потом покажем, что так как равенство для рядов достаточно проверять  покоэффициентно, то ряд можно заменить на свой начальный кусок, который является многочленом.\\
Итак докажем 4). Пусть $f$ и $g$ --- многочлены. Заметим, что правая и левая часть линейны по $f$ и $g$. Таким образом достаточно доказать только для степеней. Распишем $(x^{n+m})'=(n+m)x^{n+m-1}=nx^{n-1}x^{m}+mx^{m-1}x^n=(x^n)'x^m+(x^{m})'x^n$.

Теперь заметим, что если $f$ и $g$ ряды, то коэффициенты при $x^n$ справа и слева не зависят от коэффициентов при степенях больше $n+1$ в $f$ и $g$. \\
5) Докажем индукцией по $n$. $(f^n)'=(f\cdot f^{n-1})'=f'f^{n-1}+f(n-1)f'f^{n-2}=nf'f^{n-1}$.\\
6) Прежде всего коэффициент $n$-ой степени справа и слева не меняется если $f$ заменить на $f_{\leq n+1}$. Заменим. Теперь заметим, что обе части линейны по $f$ и поэтому можно проверять только для $f=x^n$. Это пункт 5).\\
7) Рассмотрим тождество $ff^{-1}=1$ и продифференцируем его. Перенеся одно слагаемое направо получим $(f^{-1})'f=-f'f^{-1}$. Осталось поделить на $f$.
\endproof

Итак, оставим в памяти, что производная определена для любых рядов и перекинемся на многочлены. Кажется сейчас у нас появится первое утверждение, которое зависит от характеристики кольца.

\thrm Если многочлен $f(x) \in K[x]$ делится на $p(x)^l$ то $f'(x) \di p(x)^{l-1}$. Более того, если $p(x)$ неприводим, $\chr K > \deg f$ или $\chr K=0$,а $l$ -- наибольшая степень, что $f(x) \di p(x)$, то $f'(x)\ndi p(x)^{l}$.
\proof Пусть $f(x)=p(x)^lg(x)$. Тогда $$f'(x)=lp(x)^{l-1}p'(x)g(x)+p(x)^lg'(x)=p(x)^{l-1}(lp'(x)g(x)+p(x)g'(x)) \di p(x)^{l-1}.$$
Пусть теперь $g(x)$ и $p(x)$ взаимно просты, $p(x)$ неприводим, а $\chr K > \deg f$ или $\chr K=0$. Наша задача доказать, что последний сомножитель взаимнопрост с $p(x)$ в указанных условиях. Вопрос сводится к взаимной простоте $lp'(x)$ и $p(x)$. Заметим, что если $lp'(x)$ не 0, то он действительно взаимнопрост с $p(x)$ так как $p(x)$ неприводим. Понятно, что в условиях теоремы $l\neq 0$. Следовательно надо рассмотреть ситуацию, когда $p'(x)=0$.
\lm Пусть $\chr K=p$, $f(x)\in K[x]$. Тогда $f'(x)=0$ в том и только том случае, когда $f(x)=h(x^p)$.
\elm
\proof
Заметим, что коэффициенты $f'(x)$ имеют вид $c_{i-1}=ia_{i}$. Следовательно, $a_{i}$ может не равняться нулю только тогда, когда $i\di p$. Тогда возьмём $h(x)=\sum_{i=0}^{\deg f/p} a_{ip}x^i$.
\endproof
Таким образом, если предположить, что $p'(x)=0$, то степень $p(x)=h(x^q)$, где $q=\chr K$, должна быть заведомо больше характеристики, что мы исключили по условию. Видно, что в формулировке теоремы можно ещё много уточнить. Сделайте это сами.
\endproof
\ethrm



\noindent {\bf Пример:} Вот пример многочлена, у которого проблемы с кратностью корня у производной: $x^{p}$ в $\mb Z/p$.

Обсудим ещё одно важное свойство производной. Для этого нам потребуется доказать некоторую лемму.

\lm[О разложении по основанию] Пусть $p$ -- многочлен из $K[x]$, $\alpha\in \mb N$, а $h$ многочлен c $\deg h< \alpha \deg p$. Тогда существуют единственный $h_0, \dots, h_{\alpha-1} \in K[x]$, что $\deg h_i< \deg p $ и
$$h=\sum_{i=0}^{\alpha-1} h_i p^i.$$
\elm
\proof Индукция по $\alpha$. $\alpha=1$ --- тогда $h=h_0$. Теперь $\alpha>1$.  Пусть $h=qp+h_0$. $q$ раскладывается как $\sum_{i=1}^{\alpha-2} q_i p^i$. Ясно, что получили разложение. С другой стороны, очевидно, что в любом разложении $h_0$ это остаток от деления, а $\sum_{i=1}^{\alpha-1} h_ip^{i-1}$ --- это неполное частное.  По индукции мы знаем, что разложение неполного частного по основанию единственно.
\endproof

Рассмотрим простейший неприводимый многочлен $x-a \in K(x)$. Тогда многочлены в формуле из предыдущей теоремы есть просто элементы поля $K$. Вопрос состоит в том, как найти эти коэффициенты в разложении по степеням $x-a$?

\thrm[Формула Тейлора для многочленов]
Пусть $h$ элемент $K(x)$, $\chr K =0$, и $\deg h=n$. Тогда имеет место формула
 $$h(x)= h(a)+h'(a)(x-a)+\frac{h''(a)}{2}(x-a)^2+
 \dots + \frac{h^{(n)}(a)}{n!}(x-a)^n.$$
\proof По предыдущей теореме у нас есть разложение
$$h(x)=a_0+a_1(x-a)+\dots+a_n(x-a)^n.$$
Возьмём $k$-ую производную от обеих частей равенства. Получим
$$ h^{(k)}(x)=k!a_k+(k+1)!a_{k+1}(x-a)+\dots+\frac{n!}{k!}a_n(x-a)^{n-k}.$$
Осталось подставить $x=a$.
\endproof
\ethrm


\section{Интерполяция}

Довольно часто требуется решить следующую задачу: пусть $K$ --- некоторое поле. Пусть дан набор различных точек
$x_1,\dots, x_n \in K$ и дан набор значений $a_1,\dots,a_n\in K$. Требуется построить многочлен $f\in K[x]$, такой что $f(x_i)=a_i$.
Прежде всего заметим, что у нас есть некоторая свобода выбора. А именно, рассмотрим многочлен $\phi(x)=(x-x_1)\dots(x-x_n)$. Тогда можно к любому решению интерполяционной задачи прибавить кратное многочлена $\phi(x)$ и снова получить решение интерполяционной задачи. Таким образом можно любое решение заменить на остаток от деления на многочлен $\phi(x)$. В частности, если есть какое-то решение, то есть решение степени строго меньше $n$.

\dfn[Задача интерполяции] Пусть дан набор различных точек $x_1,\dots,x_n\in K$ и дан набор значений
$a_1,\dots, a_n\in K$. Требуется построить многочлен $f\in K[x]$, такой что $f(x_i)=a_i$ и $\deg f < n$.
\edfn

\thrm Задача интерполяции разрешима и притом единственным образом. Более того её решение может быть найдено по формуле
$$f(x)=\sum_{i=1}^n a_i\frac{\prod_{j\neq i}(x-x_j)}{\prod_{j\neq i } (x_i-x_j)}=\sum_{i=1}^n \frac{a_i\phi(x)}{\phi'(x_i)(x-x_i)},$$
где $\phi(x)=(x-x_1)\dots(x-x_n)$.
\ethrm

\proof Заметим, что $\phi'(x_i)=\prod_{j\neq i}(x_i-x_j)$. Теперь очевидно, что указанная формула даёт решение нужной степени. Единственность очевидна из теоремы о многочленах, совпадающих в достаточном числе точек.
\endproof

Последняя формула называется интерполяционной формулой Лагранжа. Так же есть способ Ньютона для решения интерполяционной задачи.
Он позволяет добавлять точки постепенно.


Рассмотрим более общий вариант интерполяционной задачи. А именно попробуем решить задачу следующего вида.
Пусть задан набор точек $x_1,\dots, x_n$ и для каждой точки $x_i$ задан набор чисел $a_{i,0}, a_{i,1},\dots , a_{i,k_i}$ . Интерполяционная задача состоит в следующем: найти $f$ такой, что $j$-тая производная $f^{(j)}(x_i)=a_{i,j}$. Заметим, что решение можно свободно поменять на кратное многочлена
$$\phi(x)=\prod_{i=1}^n (x-x_i)^{k_i} ,$$
следовательно, степень решения $f$ можно ограничить $\deg f < \sum_{i=1}^n k_i$. Такая задача интерполяции называется задачей интерполяции по Эрмиту.



\thrm Пусть $K$ -- поле характеристики 0 (или достаточно большой положительной характеристики). Решение задачи интерполяции по Эрмиту существует и единственно среди многочленов степени меньше $\sum_{i=1}^n k_i$.
\ethrm
\proof Сведём задачу к китайской теореме об остатках. А именно пусть $f(x)$ многочлен. Тогда значения его производных в точке $x_i$ равны $a_{i,j}$ $j\in \ovl{0,k_i-1}$ тогда и только тогда, когда
$$f(x)\equiv a_{i,0}+a_{i,1}(x-x_i)+a_{i,2}\tfrac{(x-x_i)^2}{2!}+\dots+ a_{i,k_i}\tfrac{(x-x_i)^{k_i-1}}{(k_i-1)!}\,\, (\mod (x-x_i)^{k_i}).$$
Идеалы $(x-x_i)^{k_i}$ взаимно простые.
\endproof

\rm Вообще для  задачи интерполяции по Эрмиту тоже есть формула, но она не сильно хороша (см. Кострикин, Сборник задач по алгебре, стр. 93, задача 30.14 ).\erm


Давайте представим себе, что мы хотим перемножить два многочлена. Если действовать по определению, то понадобится $O(n^2)$ операций сложения и умножения. С другой стороны, мы знаем, что оба многочлена и их произведение однозначно определяются своими значениями в точках $x_1,\dots, x_n$ для достаточно большого $n$. Однако перемножить значения гораздо проще! Таким образом, если мы научимся быстро конвертировать многочлен в набор значений в $x_1,\dots, x_n$ и обратно (то есть решать интерполяционную задачу), то мы сможем быстро перемножать многочлены. При этом у нас есть свобода выбора $x_1,\dots, x_n$.

При произвольном выборе точек $x_i$ сложность задач интерполяции и подстановки $n$ точек есть $O(n^2)$. Однако, оказывается, что в определённых случаях можно подобрать такие $x_i$, что и задача интерполяции и задача о подстановке точек будет иметь сложность $O(n\log n)$.

\dfn Элемент  $\omega \in R$ называется первообразным корнем  степени $n$ из единицы, если $\omega$ -- корень степени $n$ из $1$-цы и $1-\omega^i$ не делитель нуля при $i\nequiv 0(\mod n)$.
\edfn




Рассмотрим упорядоченную $n$-ку $x=(a_0,\dots,a_{n-1})\in R^n$. Рассмотрим следующий набор $F(x)=(b_0,\dots,b_{n-1})\in R^n$
$$b_i=\sum_{j=0}^{n-1} a_j \omega^{ij}.$$

Отображение $x \to F(x)$ из $ R^n \to R^n$ называется дискретным преобразованием Фурье. Где находится связь между этим преобразованием и задачей интерполяции? Давайте посмотрим на строку $(a_0,\dots,a_{n-1})$ как на коэффициенты многочлена $f$ из $R[x]$. Тогда элемент $b_i=f(\omega^i)$.

Оказывается, что обратное отображение обычно есть и имеет очень простой вид:

\lm Пусть $n\in \mb N$ некоторое натуральное число, а $R$ --- кольцо, такое что $n\in R^*$. Тогда $F^{-1}(b)_i=\frac{1}{n}\sum_{j=0}^{n-1} b_j \omega^{-ij}$
\elm
\proof Достаточно доказать, что $F^{-1}(F(a))=a$. Подставим.
$$F^{-1}(F(a))_i=\frac{1}{n}\sum_{j=0}^{n-1} \sum_{k=0}^{n-1} a_k \omega^{jk} \omega^{-ij}=\sum_{k=0}^{n-1} a_k \cdot \frac{1}{n}\sum_{j=0}^{n-1} \omega^{j(k-i)}.$$

Заметим, что при $k=i$ коэффициент будет равен $\frac{1}{n}\sum_{j=0}^{n-1} \omega^{0}=1$. Вспомним, что по условию если  $k-i\neq 0$, то  $1-\omega^{k-i}$ не делитель нуля. Тогда заметим, что
$$\sum_{j=0}^{n-1}\omega^{j(k-i)}=\omega^{k-i}\sum_{j=0}^{n-1} \omega^{(j-1)(k-i)}=\omega^{k-i}\sum_{s=0}^{n-1} \omega^{s(k-i)}=\omega^{k-i}\sum_{j=0}^{n-1} \omega^{j(k-i)}.$$
Так как $1-\omega^{k-i}$ не делитель нуля,то  $\sum_{j=0}^{n-1}\omega^{j(k-i)}=0$.
\endproof

Перейдём теперь к алгоритму быстрого вычисления значений в указанных корнях из 1-цы. Этот алгоритм называется быстрым преобразованием Фурье.

\thrm Пусть $n=2^k$ и $\omega \in R$ первообразный корень степени $n$ из 1-цы. Тогда дискретное преобразование Фурье можно провести за $O(n\log n)$ операций.
\ethrm
\proof Для начала сделаем замечание: в описанных выше условиях $\omega^{\frac{n}{2}}=-1$. Действительно $0=\omega^n-1=(\omega^{\frac{n}{2}}-1)(\omega^{\frac{n}{2}}+1)$. Осталось заметить, что по условию $(\omega^{\frac{n}{2}}-1)$ не делитель нуля.

Теперь приведём алгоритм дискретного преобразования Фурье. Вспомним, что если дан многочлен $f(x)$, то посчитать его значение в точке $a$ это тоже самое, что посчитать остаток от деления $f$ на $x-a$. Нам нужно посчитать остатки от деления на $x-\omega^i$ по всем $0\leq i\leq n-1$. Заметим, что так как $\omega^{\frac{n}{2}}=-1$, то
$$x^{2^k}-1=(x^{2^{k-1}}-1)(x^{2^{k-1}}+1)=(x^{2^{k-1}}-\omega^0)(x^{2^{k-1}}-\omega^{2^{k-1}}).$$

Более общо $x^{2^i}-\omega^{2j}=(x^{2^{i-1}}-\omega^j)(x^{2^{i-1}}-\omega^{j+\frac{n}{2}})$. В частности, применив указанное соображение $k$ раз получаем
$$\prod_{i=1}^n(x-\omega^i)=x^n-1.$$

Таким образом, видно, что можно последовательно делить с остатком многочлен $f$ на многочлены $x^{2^{k-i}}-\omega^{j2^{k-i}}$, $0\leq j < 2^i$. То есть на шаге $i$ необходимо будет поделить многочлены степени $2^{i+1}$ на $2^{i}$ многочленов специального вида. Чтобы понять, что это можно легко сделать докажем лемму.

\lm Пусть многочлен $f(x)=\sum_{i=0}^{n-1} a_ix^i$. Тогда остаток от деления многочлена $f(x)$ на $x^{\frac{n}{2}}-c$ находится по формуле
$$r(x)=\sum_{i=0}^{\frac{n}{2}-1}(a_i+ca_{i+\frac{n}{2}})x^{i}. $$
В частности для вычисления указанного остатка необходимо $\frac{n}{2}$ умножений и $\frac{n}{2}$ сложений
\elm
\proof Очевидно верна формула
$$f(x)=(x^{\frac{n}{2}}-c)\sum_{i=0}^{\frac{n}{2}} a_{\frac{n}{2}+i}x^i+r(x).$$
\endproof
 Итого в нашем случае необходимо
$$n+2\frac{n}{2}+4\frac{n}{4}+\dots+2^k\frac{n}{2^k}=nk$$
умножений и сложений  в кольце $R$.
\endproof


\section{Локализация, поле частных и разложение на простейшие дроби}

Наша задача под конец этого раздела понять, как свести некоторые вопросы про общие кольца к вопросам о полях. Однако по пути мы захватим конструкцию, в некотором смысле противоположенную факторизации, которая позволяет <<упростить>> кольцо.

Все мы довольно хорошо знакомы с рациональными числами. Целые числа вкладываются в рациональные. Поэтому
многие вопросы про целые числа можно свести к рациональным. Например, допустим мы знаем, что многочлен
степени $n$ над $\mb Q$ имеет не более чем $n$ корней. Тогда мы автоматически знаем это и для целочисленных многочленов.
Достаточно просто проинтерпретировать их как многочлены с рациональными коэффициентами, а потом сказать, что
если в $\mb Q$ мало корней, то в $\mb Z$ и подавно.
Рациональные числа отличаются от целых тем, что необратимые элементы $\mb Z$ уже обратимы в $\mb Q$. Попробуем понять, можно
ли насильно обратить некоторое множество элементов. Представим себе, что мы насильно обратили два элемента $f$ и
$g$. Тогда мы так же обратили их произведение $fg$. Для простоты рассмотрим ситуацию над областью целостности, хотя разумный ответ есть и в общем случае. Далее в этом разделе $R$ --- область целостности.

\dfn[Мультипликативная система] Пусть $R$ --- область целостности. Мультипликативной системой  в $R$ называется
подмоноид $U$ в моноиде $(R\setminus\{0\},\cdot)$.
\edfn

\exm\\
1) Пусть $f\in R$. Тогда множество $\{1,f,f^2,f^3,\dots\}$ очевидно является мультипликативной системой.\\
2) Пусть $R$ --- область целостности. Тогда $R\setminus \{0\}$ является мультипликативной системой.\\
3) Более общо. Пусть дан простой идеал $p$. Тогда $U=R\setminus p$ есть мультипликативная система.\\

Дадим теперь следующее определение
\dfn[Локализация] Пусть $U$ --- мультипликативная система в $R$. Определим кольцо $R[U^{-1}]$ как
фактор множества дробей (формально --- пар)
$$ R[U^{-1}]=\{ \tfrac{a}{u}\,|\, a\in R, \, u\in U\}/\sim$$
профакторизованное по отношению эквивалентности $\sim$, заданного правилом
$$ \tfrac{a}{u}\sim \tfrac{b}{v}, \text{ если } av=bu.$$
Операции сложения и умножения введём подобно рациональным числам:
$$ \tfrac{a}{u}+\tfrac{b}{v}=\tfrac{av+bu}{uv} \text{ и } \tfrac{a}{u}\cdot\tfrac{b}{v}=\tfrac{ab}{uv}.$$
\edfn




\thrm[Конструкция работает] Пусть $U$ --- мультипликативная система в $R$. Тогда все операции на множестве $R[U^{-1}]$ корректно определены и вместе с ними $R[U^{-1}]$ становится кольцом.
\ethrm
\proof
Прежде всего необходимо проверить, что указанное отношение действительно есть отношение эквивалентности. Проверим транзитивность. Пусть $av=bu$ и $bw=cv$. Тогда $avbw=cvbu$. Сократим на $vb$.

Теперь перейдём к корректности операций. Рассмотрим сумму. Пусть $\tfrac{a}{u}\sim \tfrac{a'}{u'}$, а  $\tfrac{b}{v}=\tfrac{b'}{v'}$. Тогда сумма
$$\tfrac{a'v'+b'u'}{u'v'}\sim \tfrac{uva'v'+uvb'u'}{uvu'v'}= \tfrac{au'vv'+bv'u'u}{uvu'v'}\sim \tfrac{av+bu}{uv}.$$
Произведение --- ещё проще.

Докажем ассоциативность сложения. Пусть даны дроби $\tfrac{a}{u}$, $\tfrac{b}{v}$, $\tfrac{c}{w}$. Приведём их к общему знаменателю. Тогда ассоциативность следует из ассоциативности для кольца $R$. Остальные свойства --- так же.
\endproof



\thrm[Область целостности вкладывается в свою локализацию] Пусть $U$ мультипликативная система в области целостности $R$. Отображение $i\colon R\to R[U^{-1}]$ заданное по правилу $a\to \tfrac{a}{1}$ является инъективным гомоморфизмом колец.
\ethrm
\proof Исходя из формул для локализации видно, что это гомоморфизм. Докажем инъективность. Пусть дробь $\tfrac{a}{1}\sim \tfrac{0}{r}$. Тогда $ra=0$. Так как $r\neq 0$, то $a=0$. чтд.
\endproof

%\rm Если $R$ область целостности $U\subseteq R\setminus \{0\}$, то в определении отношения эквивалентности условие $\exists s\in U sav=abu$  можно заменить просто на $av=bu$.
%\erm

%\lm[Область целостности вкладывается в свою локализацию] Пусть $R$ --- область целостности, $U$ ---
%мультипликативная система в $R$, которая не содержит $0$. Тогда $i\colon R \to R[U^{-1}]$ является мономорфизмом.
%\elm

%\lm[Описание идеалов] Пусть $U$ --- мультипликативная система в $R$. Тогда имеет место взаимно-однозначное
%соответствие
%$$\{I \leq R\, | \,\forall u\in U, \text{ $u$ не является делителем нуля в } R/I\} \cong \{I \leq R[U^{-1}]\}.$$
%\elm

\thrm[Универсальное свойство] Пусть $R,S$ кольца и $U$ --- мультипликативная система в $R$. Тогда для любого гомоморфизма $\psi \colon R\to S$, такого что $\forall u\in U \,\,\psi(u)$ обратим, существует единственный гомоморфизм $\phi\colon R[U^{-1} ] \to S$ такой, что треугольник коммутативен:
\begin{center}
\begin{tikzpicture}
\node (A) at (0, 0) {$R$};
\node (B) at (2.5, 0) {$S$};
\node (C) at (0, -1) {$R[U^{-1}]$};
\path[->,font=\scriptsize,>=angle 60]
(A) edge node[above]{$\psi$} (B)
(A) edge node[right]{$i$} (C);
\path[dashed,->,font=\scriptsize,>=angle 60]
(C) edge node[below]{$\exists !\, \phi$} (B);
\end{tikzpicture}
\end{center}
\ethrm
\proof
Как всегда начнём с единственности. Вместо дроби $\tfrac{a}{1}$ буду писать просто $a$. Рассмотрим дробь $\tfrac{a}{u}=au^{-1}$. Тогда $\phi(au^{-1})=\phi(a)\phi(u)^{-1}=\psi(a)\psi(u)^{-1}$. Значит вариантов нет.

Теперь надо показать, что отображение, заданное правилом
$$\phi(\tfrac{a}{u})=\psi(a)\psi(u)^{-1}$$
корректно задано и является гомоморфизмом. Проверка прямолинейна.
\endproof

\dfn[Поле частных] Пусть $R$ --- область целостности. Возьмём $U=R\setminus \{0\}$. Тогда кольцо $R[U^{-1}]$ является полем. Обозначим это поле $Q(R)$. Будем называть его полем частных $R$.
\edfn

Например $\mb Q$ --- поле частных $\mb Z$.

\lm Пусть $U$ --- мультипликативная система в области целостности $R$. Тогда $R[U^{-1}]$ вкладывается в $Q(R)$.
\elm
\proof
Из $R[U^{-1}]\to Q(R)$ есть единственный гомоморфизм по универсальному свойству. Пусть он переводит дробь $\frac{f}{g}$ в $0$. Тогда он переводит $f\in R$ в ноль. Тогда  $f=0$ по теореме о вложении. Получили инъективность.
\endproof

\dfn[Локализация в простом идеале] Пусть $R$  --- область целостности, $p$ --- простой идеал. Определим кольцо $R_p= R[U^{-1}]$, где $U=R\setminus p$.
\edfn

Например, множество всех рациональных чисел, знаменатель которых не делится на $p$, обозначается $\mb Z_{(p)}$ --- является локализацией кольца $\mb Z$ в идеале $(p)$. Из кольца $\mb Z_{(p)}$ всегда есть отображение в $\mb Z/p$ по универсальному свойству. Как это может пригодиться?

Рассмотрим задачу: дана сумма $1+\frac{1}{2}+\dots+\frac{1}{p-1}$, где $p>2$ простое. Покажите, что числитель этой дроби делится на $p$. Заметим, что указанная сумма есть элемент $\mb Z_{(p)}$ . Отправим эту сумму в $\mb Z/p$. Тогда это будет сумма всех обратных элементов, то есть просто сумма всех элементов кроме 0. Она равна $\frac{p(p-1)}{2}$, то есть $0$ в $\mb Z/p$. Но тогда числитель исходной дроби переходит в ноль, то есть делится на $p$.



Конструкция поля частных бывает полезна в теоретических построениях. Например, когда мы применяем приём <<замены коэффициентов>>.

\thrm У многочлена $f$ в области целостности не более чем $\deg f$ различных корней с учётом кратности. \ethrm
\proof Пусть корни $f$ в $R$ это $x_0,\dots,x_k$ и их кратности это $\alpha_0,\dots,\alpha_k$. Вложим кольцо $R[x]$ в $Q(R)[x]$. Очевидно, что если $f\di (x-x_i)^{\alpha_i}$ над $R$, то $f\di (x-x_i)^{\alpha_i}$ над $Q(R)$. Но тогда $f\di \prod (x-x_i)^{\alpha_i}$ в $Q(R)[x]$. Тогда $  \sum \alpha_i \leq \deg f $, а степень очевидно не меняется.
\endproof



\dfn[Поле рациональных функций] Пусть $K$ --- поле. Тогда $Q(K[x])$ называется полем рациональных функций. Обозначается оно как $K(x)$.
\edfn

\dfn Пусть $R$ область целостности и $0\neq f\in R$. Пусть $U=\{1,f,f^2,\dots\}$. Тогда $R[U^{-1}]$ обозначается $R[f^{-1}]$ и
называется локализацией $R$ по элементу  $f$.
\edfn

\dfn[Многочлены Лорана] Пусть $K$ --- поле. Тогда $K[x][x^{-1}]$ называется кольцом многочленов Лорана. Обозначается как $K[x,x^{-1}]$.
\edfn

\dfn[Ряды Лорана] Пусть $K$ --- поле. Тогда $Q(K[[x]])=K[[x]][x^{-1}]$ называется полем рядов Лорана. Обозначается как $K((x))$.
\edfn

\rm Поле $K(x)$ естественным образом вкладывается в $K((x))$ по универсальному свойству.
\erm

Поговорим о специальных свойствах поля $K(x)$. Это поле в целом напоминает поле рациональных чисел, так как является полем частных евклидового кольца.

\lm Пусть $\frac{f}{g} \in K[x]$. Тогда существуют  единственные с точностью до константы многочлены  $h,r$, что $\Nod(h,r)=1$ и $\frac{f}{g}=\frac{h}{r}$. Такие дроби называются несократимыми.
\elm
\proof Возьмём какие-то $f,g$ и рассмотрим $d=\Nod(f,g)$. Тогда $h=\frac{f}{d}$, а $r=\frac{g}{d}$ подходят. Пусть есть две пары $h,r$ и $h',r'$ подходящие по условию. Тогда $hr'=h'r$. Так как $h$ и $r$ взаимно просты выполнено $h\di h'$. Симметрично $h'\di h$. Тогда $h=ch'$, где $c\in K$. Тогда $r=cr'$.
\endproof

\dfn Дробь $\frac{f}{g} \in K(x)$ называется правильной, если $\deg f< \deg g$.
\edfn

\lm Любая дробь $\frac{h}{g}\neq 0$ единственным образом представляется в виде суммы многочлена $f(x)\in K[x]$ и правильной дроби $\frac{h_1}{g_1}$, где $\frac{h_1}{g_1}$ несократимая дробь и старший коэффициент $g_1$ равен 1. При этом, если $\frac{h}{g}$ несократима, то можно считать, что $g_1=cg$ для некоторого $c\in K$
\proof Покажем существование. Сделаем дробь $\frac{h}{g}$ несократимой и старший коэффициент $g$ равным 1 поделив верх и низ на старший коэффициент. Поделим с остатком $h(x)=q(x)g(x)+h_1(x)$, где $\deg h_1(x)<\deg g(x)$. Тогда
$$\frac{h(x)}{g(x)} =\frac{q(x)g(x)+h_1(x)}{g(x)} =q(x)+\frac{h_1(x)}{g(x)}.$$
Покажем единственность. Пусть
$$f_1(x)+\frac{h_1}{g_1}=f_2(x)+\frac{h_2}{g_2}.$$
Имеем равенство многочленов.
$$(f_1(x)-f_2(x))g_1(x)g_2(x)=g_1(x)h_2(x)-h_1(x)g_2(x).$$
Если $f_1\neq f_2$, то степень многочленов справа строго меньше степени многочлена слева. Таким образом $f_1=f_2$, а $\frac{h_1}{g_1}=\frac{h_2}{g_2}$. По единственности несократимой записи $h_1=h_2$ и $g_1=g_2$.
\endproof
\elm

\rm Сумма двух правильных дробей -- снова правильная дробь.
\erm

\dfn[Простейшие дроби] Пусть $K$ --- поле, $p\in K[x]$ --- неприводимый многочлен со страшим коэффициентом единица. Тогда дробь
$$\frac{f(x)}{p(x)^{k}} \text{ называется простейшей, если $f \neq 0$ и $\deg f < \deg p$}. $$
\edfn



\thrm[О разложении на простейшие] Пусть $K$ ---  поле. Тогда для любой несократимой дроби $0
\neq \frac{f}{g} \in K(x)$ существуют единственные многочлен $h\in K[x]$, неприводимые многочлены $p_1, \dots, p_n$ со старшим коэффициентом 1, натуральные числа $\alpha_1,\dots, \alpha_n$ и многочлены $h_{ij}$, где $i\in \ovl{1,n}$, и $j\in \ovl{0,\alpha_i}$, что дроби
$$ \frac{r_{ij}}{p_i^{j}} \text{ --- простейшие и } \frac{f}{g}=h+\sum_{i,j} \frac{r_{ij}}{p_i^{j}}.$$
При этом, если  $r_{i\alpha_i}$ не ноль и старший коэффициент $g$  равен 1, то $g=\prod p_i^{\alpha_i}$.
\ethrm

\proof Докажем существование разложения. Если $g(x)=p(x)^{\alpha}$, то разложение можно получить следующим способом -- надо представить дробь $\frac{f(x)}{g(x)}= h(x)+\frac{r(x)}{g(x)}$ в виде многочлена и правильной дроби, а затем разложить по основанию $p(x)$ числитель  $r(x)=\sum_{j=1}^{\alpha} r_{j}p(x)^{\alpha-j} $, где $\deg r_i < \deg p(x)$. Теперь получаем, что

$$\frac{f(x)}{g(x)}= h(x)+\sum \frac{r_j(x)}{p^j(x)}$$

Это отличная база для индукции по степени многочлена в знаменателе (или по числу различных неприводимых делителей знаменателя). Допустим, что многочлен $g(x)$ раскладывается на взаимно простые множители $g=g_1g_2$, где $\Nod(g_1,g_2)=1$. Представим $a(x)g_1+b(x)g_2=1$. Тогда, конечно, $f(x)=f(x)a(x)g_1+f(x)b(x)g_2$. Подставляя в дробь, получим

$$\frac{f(x)}{g(x)}=  \frac{f(x)a(x)}{g_2(x)}+\frac{f(x)b(x)}{g_1(x)}.$$
Знаменатели явно уменьшились в степени.\\
Покажем единственность. Возьмём $g$ таким, что его старший коэффициент равен 1. Итак пусть
$$\frac{f}{g}=h+\sum_{i,j} \frac{r_{ij}}{p_i^{j}},$$
где $r_{i\alpha_i}$ не ноль (можно так считать, выкинув нулевые слагаемые с большей степенью $p_i$ в знаменателе).

Прежде всего заметим, что $$\sum_{i,j} \frac{r_{ij}}{p_i^{j}}$$
правильная дробь. Тогда $h$ определяется однозначно из единственности представления дроби в виде суммы многочлена и правильной дроби. Заменяя $f$ на $f-gh$ можем считать, что дробь $\frac{f}{g}$ правильная. Теперь  покажем, что  $g=\prod p_i^{\alpha_i}$. Введём обозначение $$r_i= \sum r_{ij} p_i^{\alpha_i-j}.$$
Приведя правильную дробь справа ($h$ больше нет) к общему знаменателю получим дробь $$\frac{\sum_{i} r_{i}\prod_{k\neq i} p_k^{\alpha_{k}}}{\prod p_i^{\alpha_{i}}}.$$
Покажем, что эта дробь несократима. Выберем некоторый элемент $p_l$ и покажем, что числитель не делится на $p_l$. Слагаемые вида $$r_{i} \prod_{k\neq i} p_k^{\alpha_{k}}$$ делятся на $p_l$, если $l \neq i$. С другой стороны, $r_l$ не делится на $p_l$, так как $r_{l\alpha_l}$ не делится (потому что он не ноль и степени меньше $\deg p_l$). Тогда и вся сумма не делится на $p_l$.
Теперь $\prod p_i^{\alpha_i}=g$ из единственности несократимой записи  дроби, а $$f=\sum_{i} r_{i}\prod_{k\neq i} p_k^{\alpha_{k}}.$$
Тогда
$$f\equiv r_{i}\prod_{k\neq i} p_k^{\alpha_{k}} \mod p_i^{\alpha_i},$$
откуда, пользуясь обратимостью $\prod_{k\neq i} p_k^{\alpha_{k}} $ по модулю $p_i^{\alpha_i}$, получаем

$$r_i \equiv f \left(\prod_{k\neq i} p_k^{\alpha_k}\right)^{-1} \!\! \!\! \!\! \mod p_i^{\alpha_i}.$$ Из этого соотношения $r_i$ определяется однозначно, так как его степень меньше $\deg p_i^{\alpha_i}$. Чтобы восстановить $r_{ij}$ надо $r_i$ разложить по основанию $p_i$. Такое разложение единственно по лемме.
\endproof

\zd Какой аналог у последней теоремы в рациональных числах?
\ezd

Рассмотрим теперь конкретные поля $\mb C$. Так как поле $\mb C$ алгебраически замкнуто, то все неприводимые многочлены над $\mb C$ имеют степень 1. Это заметно упрощает жизнь, так как в числителе простейшей дроби могут стоять только константы.

Самый стандартный, но далеко не самый эффективный способ нахождения разложения на простейшие --- метод неопределённых коэффициентов. Приведём пример нахождения некоторого разложения, которое использует конструкцию интерполяции.

Рассмотрим рациональную функцию $\frac{1}{x^{n}-1}$. Я хочу найти её разложение на простейшие над $\mb C$. Корни  многочлена $x^{n}-1$ нам известны --- это $x_l=e^{\tfrac{2i \pi l}{n}}$, $l\in \ovl{0,n-1}$, они без кратностей. Многочлен $g(x)=1$ восстанавливается по своим значениям в точках  $e^{\tfrac{2i \pi l}{n}}$ по формуле Лагранжа
$$1=\sum_{l=0}^{n-1} \frac{ \prod_{i\neq l} (x-x_i)}{nx_l^{n-1}}.$$
Тогда
$$\frac{1}{x^{n}-1}=\frac{\sum_{l=0}^{n-1} \frac{ x_l\cdot \prod_{i\neq l} (x-x_i)}{n}}{x^{n}-1}= \sum_{l=0}^{n-1} \frac{x_l}{n(x-x_l)}.$$



\section{Степенные ряды как производящие функции}


\dfn[Линейное рекуррентное соотношение] Будем говорить, что последовательность $x_n$ удовлетворяет линейному рекуррентному соотношению $k$-го порядка, если существуют числа $a_0,\dots,a_{k}$, что $a_k,a_0\neq 0$ и
$$a_k x_{n+k}+a_{k-1}x_{n+k-1}+\dots+a_0x_n=0$$
\edfn


\rm Вообще говоря, ничто не мешает считать, что начальные коэффициенты $a_0,\dots,a_s=0$. Просто это означает, что до номера $k+s$ последовательность может быть любой, а после $k+s$ начинает удовлетворять соотношению с коэффициентами $a_{s+1},\dots,a_k$.
\erm

\dfn[Производящая функция] Пусть дана последовательность $a_n$, $n\geq 0$. Производящей функцией для последовательности $a_n$ назовём формальный степенной ряд $f(x)=\sum_{i=0}^{\infty} a_ix^i$.
\edfn



\thrm Ряд из $K[[x]]$ является рядом некоторой правильной дроби $\frac{f(x)}{g(x)}$, тогда и только тогда, когда его коэффициенты  удовлетворяют линейному рекуррентному соотношению. Более того, порядок наименьшего линейного рекуррентного соотношения, которому удовлетворяют коэффициенты, равен степени знаменателя в несократимой дроби.
\ethrm

\proof
Заметим, что, ряд Лорана  несократимой дроби $\frac{f}{g}$ лежит в $K[[x]]$ тогда и только тогда, когда $g(x)\ndi x$. Если $g(x)\ndi x$, то $g(x)$ обратима в $K[[x]]$, откуда и вся дробь лежит в $K[[x]]$. Обратно, если $g \di x$, и $\frac{f}{g}$ лежит в $K[[x]]$, то $f\ndi x$ и тогда $f$   обратимо в $K[[x]]$ откуда $g(x)^{-1} \in K[[x]]$, что не возможно, потому что свободный член $g$ равен 0.

Пусть $q(x)$ есть ряд правильной несократимой дроби $\frac{f(x)}{g(x)}$. Тогда $q(x)$ удовлетворяет соотношению $g(x)q(x)=f(x)$. Мы уже один раз выписывали соотношение на коэффициенты $g(x)$, когда $f(x)$ был равен 1. Поступим аналогично. Пусть $g(x)=b_nx^n+\dots +b_0$, $f(x)=a_mx^m+\dots +a_0$, а $q(x)=c_0+c_1x+\dots$. Тогда имеем уравнения на коэффициенты $q(x)$

$$ \sum_{j=0}^{n} b_j c_{i-j} =a_i .$$
Выражение справа равно 0 при $i>m$. Выражение слева при $i\geq n$ всегда имеет $n$ слагаемых. Функция $g(x)$ не может делиться на $x$ так как иначе мы не получили бы элемент из $K[[x]]$ или дробь была бы сократима. Тогда $b_0\neq 0$. Таким образом, при $i\geq n$ получаем рекуррентное соотношение на $c_j$

$$ b_0 c_{j+n}+b_1 c_{j+n-1}+\dots + b_n c_j=0.$$
Обратно, пусть $c_j$ удовлетворяют рекуррентному соотношению
$$ b_0 c_{j+n}+b_1 c_{j+n-1}+\dots + b_n c_j=0, \text{ где } b_0,b_n \neq 0.$$

Тогда возьмём в качестве $g(x)= b_n x^n+\dots+b_0$. Как теперь найти $f(x)$? Вспомним условие на коэффициенты и положим
$$a_i= \sum_{j=0}^{n} b_{j} c_{i-j}, \text{ где } i\in \ovl{0,n}.$$
Это и есть коэффициенты $f(x)$. Допустим дробь $\frac{f}{g}$ сократима. Тогда по уже доказанному она удовлетворяет соотношению меньшего $\deg g$ порядка.
\endproof

Рассмотрим простейший пример. Какая рациональная функция соответствует последовательности удовлетворяющей соотношению $z_{n+1}=\lambda z_n$, $z_0=1$? Эта последовательность имеет вид $z_n=\lambda^n$. Ей соответствует ряд $$1+ \lambda x+\dots + \lambda^nx^n+\dots .$$
Это ряд для функции
$$\frac{1}{1-\lambda x}.$$
Значит функции
$$\frac{1}{x-\lambda}=\frac{-1}{\lambda}\frac{1}{1-\frac{x}{\lambda}}$$
соответствует последовательность $z_n=-{\frac{1}{\lambda}^{n+1}}$, то есть некоторая геометрическая прогрессия.

А что соответствует ${\frac{1}{(1-\lambda x)^k}}$, где $k\geq 2$? Вспомним про производную. Заметим, что $$\frac{d^{k-1}}{dx^{k-1}}\frac{1}{1-\lambda x}=\frac{\lambda^{k-1} (k-1)!}{(1-\lambda x)^k}.$$
Переписывая получаем
$$\frac{1}{(1-\lambda x)^k}= \frac{1}{\lambda^{k-1} (k-1)!}\frac{d^{k-1}}{dx^{k-1}}\frac{1}{1-\lambda x}=  \sum_{n=0}^{\infty} C_{n+k-1}^{k-1} \lambda^{n}x^{n}.$$

\crl Пусть последовательность комплексных чисел $z_n$ удовлетворяет соотношению $a_k z_{n+k}+a_{k-1}z_{n+k-1}+\dots+a_0z_n=0$. Пусть многочлен $p(x)=a_k x^k+\dots +a_0$ имеет корни $\lambda_1$ кратности $k_1$, $\ldots$, $\lambda_l$ кратности $k_l$. Тогда последовательность $z_n$ имеет вид
$$ p_1(n)\lambda_1^n+\dots+p_l(n)\lambda_l^n,$$
где $p_i$ многочлен степени не выше $k_i$.
\proof
Рассмотрим многочлен $g(x)=a_0x^k+\dots+a_k$. Заметим, что $$g(x)=x^k p\left(\frac{1}{x}\right)= x^k\prod\left(\frac{1}{x}-\lambda_i\right)^{k_i}= \prod (1-\lambda_ix)^{k_i}.$$
Последовательность $z_n$ имеет производящую функцию вида
$$\frac{h(x)}{g(x)}.$$
Разложим её на простейшие над $\mb C$. Получим
$$\frac{h(x)}{g(x)}=\sum_{i=1}^l \sum_{0\leq j < k_i} \frac{b_{ij}}{(1-\lambda_ix)^j}.$$
Каждое слагаемое является производящей функцией для последовательности вида $p_{ij}(n)\lambda_i^n$, $\deg p_i < k_i$. Осталось просуммировать при одинаковом $i$.
\endproof
\ecrl

\dfn Многочлен $a_kx^k+\dots +a_0$ называется характеристическим многочленом линейной рекуррентной последовательности.
\edfn

Рассмотрим пример: пусть $f_n$ --- последовательность чисел Фибоначчи. Она удовлетворяет рекуррентному соотношению $f_{n+2}-f_{n+1}-f_n=0.$ Такой последовательности соответствует многочлен $g(x)=-x^2-x+1$ и $f(x)=x$. Рассмотрим дробь $F(x)=\frac{x}{-x^2-x+1}$  и разложим её в сумму простейших. Корни знаменателя это $\lambda_1=\frac{-1+\sqrt{5}}{2}$ и $\lambda_2=\frac{-1-\sqrt{5}}{2}$. Получим
$$F(x)= -\left(\frac{\lambda_1}{\sqrt{5}(x-\lambda_1)}-\frac{\lambda_2}{\sqrt{5}(x-\lambda_2)}\right).$$
Представим каждое слагаемое в виде ряда и получим формулу
$$f_n= \frac{1}{\sqrt{5}}(\ffi^{n-1}-\ovl{\ffi}^{n-1}),$$
где $\ffi=\frac{1+\sqrt{5}}{2}$, а $\ovl{\ffi}=\frac{1-\sqrt{5}}{2}$

Пусть последовательность $a_{n+2}=4a_{n+1}-4a_n$ начинается с $a_1=2$, $a_0=0$. Найдём общую формулу. Рассмотрим дробь $$A(x)=\frac{2x}{4x^2-4x+1}=\frac{2x}{(2x-1)^2}.$$
Как найти разложение в ряд? Заметим, что $$\frac{2}{(2x-1)^2}= \frac{d}{dx}\frac{-1}{2x-1}.$$
Вспомним, как считается производная для рядов. Тогда
$$A(x)=\sum_{n=0} (n+1) 2^{n+1} x^{n+1}=\sum_{n=0} n2^nx^n.$$




Заметим, что если $z_n$ --- комплексная последовательность, удовлетворяющая линейному рекуррентному соотношению, то её производящая функция $f(x)=\frac{h(x)}{g(x)}$ имеет конечный набор комплексных точек, в которых она не определена. Более того, мы даже знаем эти комплексные точки --- это корни $g(x)$, то есть обратные к корням характеристического многочлена. Общая философия, которая за этим стоит такая --- поведение последовательности определяется  <<особыми точками>> её производящей функции, то есть точками на комплексной плоскости, куда эта  функция не может быть продолжена, например, её полюсами.

\chapter{Линейная алгебра}

\setcounter{zad}{0}
\setcounter{lem}{0}
\setcounter{thm}{0}
%\setcounter{defn}{0}
\setcounter{cor}{0}

\section{Системы линейных уравнений и метод Гаусса}

\dfn Пусть $R$ -- кольцо. Тогда системой $m$ линейных уравнений от $n$ неизвестных называется набор условий

$$\begin{cases}
a_{11}x_1+\dots + a_{1n}x_n=b_1\\
\vdots \\
a_{m1}x_1+\dots+a_{mn}x_n=b_m
\end{cases},$$
где $a_{ij}, b_i \in R$.
\edfn

 Совершенно понятно, что система линейных уравнений определяется однозначно числами $a_{ij}$ и $b_i$. Эти числа удобно организовывать в матрицы.

\dfn Матрица размера $m\times n$ над кольцом $R$ -- это набор чисел проиндексированных двумя индексами $a_{ij}$ $i\in \ovl{1,m}$, $j\in \ovl{1,n}$. Множество всех матриц размера $m\times n$ над кольцом $R$ обозначается как $M_{m\times n}(R)$. Обычно матрицы я буду обозначать заглавными буквами, например $A$. Тот факт, что матрица $A$ имеет размер $m\times n$ будем записывать как $A\in M_{m\times n}(R)$.
\edfn

\dfn Матрица системы линейных уравнений называется матрица $A\in M_{m\times n}$, заполненная коэффициентами этой системы -- то есть числами $a_{ij}$. Матрица размера $m\times (n+1)$ содержащая дополнительно столбец $b_1,\dots, b_m$ называется расширенной матрицей системы. Мы будем отчёркивать столбец $b$, чтобы выделить его особую роль и будем обозначать расширенную матрицу системы как $(A|b)$.
\edfn

\dfn Пусть $A$ -- матрица $M_{m\times n}(K)$ составленная из элементов $a_{ij}$, а $x \in K^n$ -- столбец с компонентами $x_j$. Определим произведение матрицы $A$ на столбец $x$, как элемент $Ax \in K^m$, заданный по правилу
$$(Ax)_i=\sum_{j=1}^n a_{ij}x_j.$$
 \edfn

Таким образом матрица $A \in M_{m\times n}(K)$ задаёт отображение $K^n \to K^m$. А систему уравнений с матрицей $A$ и столбцом $b$ можно переписать как
$$Ax=b.$$

От каждой системы нас прежде всего интересует множество её решений. Поэтому логично ввести определение:

\dfn Две системы линейных уравнений называются эквивалентными, если множества их решений совпадают.
\edfn

Как для данной системы линейных уравнений можно построить эквивалентную? Прежде всего заметим, что если есть два уравнения, то по ним можно построить много новых, а именно, пусть имеют $\lambda$ и $\mu$ из $R$. Тогда сложив два уравнения с коэффициентами $\lambda$ и $\mu$ получаем третье
$$\begin{cases}
a_1x_1+\dots+a_nx_n=c\\
b_1x_1+\dots+b_nx_n=d
\end{cases} \Rightarrow (\lambda a_1+\mu b_1)x_1+ \dots +(\lambda a_n+ \mu b_n)x_n= \lambda c+\mu d.$$
Понятно, что если набор $(x_1,\dots,x_n)$ -- решение первых двух, то и нового тоже.

Получать новые уравнения мы научились, остался вопрос про эквивалентные системы. Введём определение элементарных преобразований.

\dfn Пусть дана система уравнений
 $$\begin{cases}
a_{11}x_1+\dots + a_{1n}x_n=b_1\\
\vdots \\
a_{m1}x_1+\dots+a_{mn}x_n=b_m
\end{cases},$$
Элементарным преобразованием первого типа над этой системой линейных уравнений называется следующая операция. Рассмотрим уравнения с номерами $i$ и $j$, где $i\neq j$ и элемент $\lambda \in K$. Тогда прибавим  $i$-ое уравнение к $j$-ому с коэффициентом $\lambda$ и поместим результат на место $j$-го уравнения.
\edfn

\rm Очевидно, что решение новой системы содержит решения старой. Однако верно и наоборот, так как старая система получается из новой аналогичным прибавлением $i$-ого уравнения к $j$-ому, но с коэффициентом $-\lambda$.
\erm

\dfn Элементарным преобразованием второго типа называется преобразование меняющее местами $i$-ое и $j$-ое уравнения местами. Элементарным преобразованием третьего типа называется преобразование, домножающие $i$-ое уравнение на коэффициент $\lambda \in R^*$.
\edfn

\rm Элементарное преобразование третьего типа приводит к эквивалентной системе так как есть обратное преобразование -- домножение на $\lambda^{-1}$. Для преобразования второго типа эквивалентность тривиальна.
\erm

Нам будет удобно вместо системы линейных уравнений работать с её упрощённой записью -- матрицей этой системы. Поэтому логично перевести понятия элементарных преобразований на язык матриц.

\dfn Элементарным преобразованием строк первого типа над матрицей $A$ называется прибавление к $j$-ой строчке матрицы $A$ её $i$ строки с некоторым коэффициентом $\lambda$. Элементарным преобразованием второго типа называется перестановка $i$-ой и $j$-ой строк в матрице $A$. Преобразованием третьего типа называется домножение $i$-ой строчки на обратимый элемент $\lambda \in R^*$.
\edfn

Нас в основном пока будет интересовать случай $R=K$ -- поле. Обсудим метод Гаусса решения систем линейных уравнений. Он заключается в том, чтобы с помощью элементарных преобразований перевести систему уравнений к эквивалентной системе, так, чтобы вид последней был как можно более простым. Сформулируем это.

\dfn Будем говорить, что матрица $A$ имеет ступенчатый вид, если каждая новая строчка начинается с большего количества нулей, чем предыдущая. Говоря строго, для $i$-ой строки номер столбца в котором стоит первый ненулевой элемент строки строго больше, чем аналогичный  номер у $i-1$ строки, если только строка не целиком состоит из нулей.
\edfn

Утверждение, которое стоит за методом Гаусса можно сформулировать следующим образом:

\thrm Любую матрицу над полем $K$ можно перевести элементарными преобразованиями к ступенчатому виду. Более того, можно считать, что для каждой строки первый её ненулевой элемент равен 1 и в столбце над ним стоят нули.
\ethrm
Предъявим индукционный алгоритм для получения ступенчатого вида:\\
{\bf Случай 1:} Элемент $a_{11}\neq 0$. Тогда прибавим ко всем остальным строкам первую с коэффициентами $-\frac{a_{i1}}{a_{11}}$. Получится матрица у которой в перво столбце стоят нули, кроме первой позиции. Вычеркнем первый столбик и первую строчку и продолжим по индукции.\\
{\bf Случай 2:} Элемент $a_{11}=0$, но в $i$-ой строчке стоит ненулевой элемент. Поменяем строку с номером $i$ с первой строкой и продолжим, как в случае 1.\\
{\bf Случай 3:} Весь первый столбец нулевой. Тогда вычеркнем первый столбец и продолжим по индукции.


Указанные преобразования очевидно приводят матрицу к ступенчатому виду. Способ добиться нулей над первыми не нулевыми элементами называется обратным ходом метода Гаусса.


Прежде всего заработаем единицы в первых ненулевых элементах строк $a_i$ поделив на $a_i^{-1}$.

Затем посмотрим на последнюю ненулевую строку -- скажем строку $k$, первый  столбец с ненулевым элементом в которой имеет номер $j_k$, и прибавим её ко всем строкам выше с коэффициентом $-a_{lj_k}$ для $l$-ой строки. После чего перейдём к следующей строке.\\



\noindent {\bf Метод Гаусса}
Приступим к решению системы линейных уравнений. Приведём расширенную матрицу системы к ступенчатому виду. Рассмотрим последнюю ненулевую строчку. Если её ненулевой элемент находится в самом последнем отчёркнутом столбце, то решений нет, потому, что эта строчка соответствует уравнению $0x_1+\dots+0x_n=b_1\neq 0$, которое, как ни крути, решений не имеет.

Если же такого не происходит, то разделим переменные на два класса -- те, в чьём столбце есть первый ненулевой элемент какой-то строки и те, в чьём столбце такого элемента нет -- обозначим последние за $x_{i_1},\dots,x_{i_n}$. Тогда выбрав любые значения для $x_{i_1},\dots,x_{i_n}$ из оставшихся уравнений мы однозначно восстановим значения всех остальных переменных. Более того, значения остальных переменных представляются в виде значений многочленов первой степени от  $x_{i_1},\dots,x_{i_n}$. Так выглядит стандартное описание всех решений линейного уравнения, которое выдаёт метод Гаусса.

\rm Метод Гаусса подразумевает работу со строчками в определённой порядке, в частности, перестановка строк делается только в экстренных случаях. Но, в принципе, никто не запрещает для удобства переставлять строчки и прибавлять их друг к другу в произвольном порядке -- лишь бы вид системы в конце позволял проанализировать множество решений.
\erm

\exm\\
Допустим мы хотим наладить некоторую поисковую систему. Что это значит? Поисковая система индексирует страницы в сети и то, какая страница на какую ссылается. Иными словами, поисковая система видит ориентированный граф $G$, вершины которого -- это страницы в сети, а рёбра проводятся, если один сайт ссылается на другой.

Что же должна сделать поисковая система? Ей неплохо было назначить каждому сайту его важность. То есть необходима функция из множества вершин графа в вещественные числа  $W\colon G \to \mb R$. Важность сайта зависит от того, насколько много на него ссылаются. Это приводит  к следующей системе уравнений:
$$w_i=\sum_{j\to i} \frac{1}{d_j^{out}}w_j,$$
где $d_j^{out}$ исходящая степень вершины $j$.

Приведу другой пример, в котором систему линейных уравнений можно увидеть не сразу. Точнее, рассмотрим набор целых чисел $a=-5250$, $b=-140$, $c=-1050$, $d=-1470$. Вопрос: можно ли найти такие $k_1,k_2,k_3,k_4$ не все чётные, что
$a^{k_1}b^{k_2}c^{k_3}d^{k_4}$ есть квадрат целого числа? Вообще-то это довольно сложная задача. Но числа которые я здесь взял немного специальные. Я знаю их разложение на множители:
$$a=2\cdot 3\cdot 5^3\cdot 7, \,\,b=-2^2\cdot 5\cdot 7, \,\, c=-2\cdot 3\cdot 5^2 \cdot 7,\,\, d=-2\cdot 3\cdot 5\cdot 7^2.$$

Целое число является квадратом, тогда и только тогда, когда оно положительно и любое простое входит в него в чётной степени. Получаем, что $a^{k_1}b^{k_2}c^{k_3}d^{k_4}$ есть квадрат, тогда и только тогда, когда
$$\begin{cases} k_2+k_3+k_4=0 \mod 2,\\
k_1+2k_2+k_3+k_4=0 \mod 2,\\
k_1+k_3+k_4=0 \mod 2,\\
3k_1+k_2+2k_3+k_4=0 \mod 2,\\
k_1+k_2+k_3+2k_4=0 \mod 2,\\
\end{cases}.$$
Похоже, что это система уравнений над $\mb Z/2$. Сразу видно, что на $k_i$ можно смотреть по модулю 2. Составим матрицу системы.

$$A=\pmat 0&0&1&1\\
1&0&1&1\\ 1&0&1&1 \\
1&1&0 &1 \\
1&1&1&0 \epmat.$$

Начнём применять элементарные преобразования. Замечу, что столбец $b$ нулевой и его можно опустить.
$$\pmat
0&0&1&1\\
1&0&1&1\\
1&0&1&1 \\
1&1&0&1 \\
1&1&1&0 \epmat \stackrel{\substack{(3)+(2), (4)+(2),\\ (5)+(2), (2) \to (1),\\ (2) \to (4), (3) \to (5)}}{\sim}
\pmat
1&0&1&1\\
0&1&0&1\\
0&1&1&0 \\
0&0&1&1\\
0&0&0&0 \epmat \stackrel{\substack{(3)+(2), (4)+(3)}}{\sim} \pmat
1&0&1&1\\
0&1&0&1\\
0&0&1&1 \\
0&0&0&0\\
0&0&0&0 \epmat
\stackrel{\substack{(1)+(3)}}{\sim} \pmat
1&0&0&0\\
0&1&0&1\\
0&0&1&1 \\
0&0&0&0\\
0&0&0&0 \epmat
.$$
Ответ да можно. Возьмём $k_4=1$, тогда $k_2,k_3=1$, $k_1=0$.

Подобная задача возникает при поиске разложения числа $n$ на множители. Точнее, нетривиальное решение сравнения $x^2=y^2(\mod n)$ (то есть $x\neq \pm y (\mod n)$) даёт разложение числа $n$. Для получения такого разложения можно взять набор не очень больших простых $b_1,\dots,b_k$ и считать  $x \to x^2$ для случайного $x$. Если $k$ большое, то с высокой вероятностью $z=x^2 (\mod n)$ есть произведение $b_1^{\alpha_1}\dots b_k^{\alpha_k}$. Оставим только такие $z_i$. Подберём $d=\prod z_i^{\eps_i}$, чтобы $d=y^2$ в $\mb Z$. Тогда  $y^2=d=\prod {x_i^{\eps_i}}^2 (\mod n)$. Вопрос, почему это эффективный алгоритм относится к разделу аналитической теории чисел и пока вне нашей досягаемости.

Казалось бы, мы научились решать произвольную систему линейных уравнений -- что же ещё можно спросить? Я поставлю несколько вопросов, на которые отвечу в дальнейшем. Вот первый и основной из них:\\
1) Предположим, что две системы эквивалентны, например, потому что получились одна из другой перестановкой строк. Верно ли, что ответ для общего решения ответ в методе Гаусса будет содержать одинаковое число независимых параметров? А что будет, если одна система из другой получается заменой переменных?\\
2) В методе Гаусса видно, что решающую роль играет матрица системы. Вопрос: как по матрице системы определить, для каких $b$ система будет разрешима?\\
3) Мы интуитивно догадываемся, что обычно решение системы из $n$ уравнений и $m$ неизвестных описывается $m-n$ параметрами (если это число не отрицательно, конечно). Однако это не всегда так. Вопрос -- найти критерии, когда это выполнено, а когда нет. Это очень полезно, например, в задаче про поисковик. Ведь в этой задаче $n$ уравнений на $n$ неизвестных, но она, очевидно, имеет нулевое решение. Хочется верить, что оно не единственное.




\section{Векторные пространства}

Наша задача понять, какие ситуации приводят к линейным уравнениям.

\dfn[Векторное пространство]
Векторным пространством над полем $K$ называется множество $V$ вместе с отображениями $+\colon V\times V \to V$ и $\cdot \colon K \times V \to V$, удовлетворяющее свойствам:\\
1) $V, +$ является абелевой группой\\
2) $\forall v \in V$ верно, что $1\cdot v=v$\\
3) $\forall v \in V$, $\forall \lambda, \mu \in K$ верно, что $(\lambda+\mu)\cdot v= \lambda\cdot v + \mu \cdot v$.\\
4) $\forall u,v \in V$, $\forall \lambda \in K$ верно, что $\lambda\cdot(u+v)= \lambda\cdot u + \lambda \cdot v$.\\
5) $\forall v \in V$ $\forall \lambda, \mu \in K$ верно, что $(\lambda\mu)\cdot v= \lambda\cdot(\mu \cdot v)$.
\edfn

\dfn Пусть $R$ -- кольцо. Тогда модулем (левым) над $R$ называется множество $M$ вместе с отображениями $+\colon M\times M \to M$ и $\cdot \colon R \times M \to M$, удовлетворяющее свойствам 1)-5) из определения векторного пространства
\edfn

Я немного подожду с примерами модулей над кольцами, чтобы стало понятно, насколько теория с модулями сложнее, чем теория над полем, а пока лишь приведу примеры векторных пространств.

\exm\\
0) Само поле $K$ вместе со сложением и умножением.\\
1) Пространство столбцов $K^n$. Умножение и сложение покомпонентное.\\
2) Обобщая. Пространство матриц $M_{m\times n}(K)$.\\
3) Пусть $X$ -- множество. Рассмотрим множество всех функций  из $K$ в $X$ , то есть $K^X$. Это векторное пространство над полем $K$ с поточечным сложением и умножением.\\
4) Рассмотрим множество непрерывных вещественнозначных функций на отрезке $[0,1]$. Это векторное пространство над $\mb R$.\\
5) Рассмотрим множество последовательностей над полем $K$, удовлетворяющих заданному линейному рекуррентному соотношению $a_k x_{n+k}+\dots+a_0x_n=0$. Это векторное пространство над $K$.\\
6) Рассмотрим множества всех многочленов $K[x_1,\dots,x_n]$, всех рациональных функций $K(x_1,\dots, x_n)$, рядов $K[[x_1,\dots,x_n]]$. Это векторные пространства относительно естественного сложения и умножения на элементы $K$.\\
7) Пусть $A$ -- абелева группа, такая, что любой ей элемент имеет порядок $p$, для фиксированного простого числа $p$. Тогда на $A$ существует и единственна структура векторного пространства над $\mb Z/p$.\\
8) Пусть $L$ -- расширение поля $K$. Тогда $L$ является векторным пространством над $K$. В частности $\mb C$ вектроное пространство над $\mb R$ и над $\mb Q$. Аналогично $\mb R$ -- векторное пространство над $\mb Q$.\\
9) Рассмотрим фактор $K[x_1,\dots,x_n]/I$ -- это векторное пространство над полем $K$.\\

Отметим простейшие свойства элементов векторных пространств.
\lm Пусть $V$ -- векторное пространство над полем $K$. Тогда выполнено:\\
1) $\forall v  \in V$ выполнено, что $0\cdot v =0$.\\
2) Если для некоторых $\lambda \in K$ и $v \in V$ верно равенство $\lambda v =0$, то либо $\lambda=0$, либо $v=0$.\\
3) $\forall v  \in V$ выполнено, что $(-1)\cdot v=-v$.\\
\elm






\dfn[Подпространство] Пусть $V$ -- векторное пространство над полем $K$. Подмножество $U\subseteq V$ называется подпространством $V$, если\\
1) $U$ -- подгруппа $V$.\\
2) $\forall \lambda \in K$, $\forall u \in U$ верно, что $\lambda u \in U$.\\
По другому говоря,  операции на $V$ можно сузить на $U$, с тем, чтобы $U$ стало векторным пространством относительно этих операций.
\edfn

\exm\\
1) Рассмотрим множество столбцов из $K^n$, у которых первая координата равно 0. Это подпространство в пространстве столбцов.\\
2) Рассмотрим множество многочленов, которые делятся на данный многочлен $p(x)K[x]$. Это подпространство в $K[x]$. \\
3) Рассмотрим множество непрерывных на отрезке $[0,1]$ функций, принимающих значение $0$  в точках $0, \frac{1}{2}, 1$. Это подпространство в $C[0,1]$.\\
4) Рассмотрим множество многочленов степени не выше $n$ от одной переменной $$K[x_1,\dots, x_k]_{\leq n}=\{ f \in K[x_1,\dots, x_k]\,|\, \deg f\leq n\}.$$ Это подпространство в  $K[x_1,\dots,x_n]$.\\
5) Рассмотрим множество правильных дробей $\frac{f}{g}\in K(x)$. Это подпространство в $K(x)$.\\
6) Множество решений уравнения $Ax=0$, где матрица $A \in M_{m \times n}(K)$ образует подпространство в $K^n$.\\




Как и в жизни элементы пространства обычно не интересны сами по себе, а только во взаимоотношении с окружающей их действительностью.

\dfn(Линейная комбинация) Линейной комбинацией векторов $v_1,\dots, v_n$ с коэффициентами $\lambda_1, \dots, \lambda_n$, называется элемент $v\in V$
$$v=\lambda_1 v_1 +\dots + \lambda_n v_n.$$
Если хотя бы один из элементов $\lambda_1,\dots, \lambda_n $ не равен 0, то говорят, что линейная комбинация нетривиальна.
\edfn

\lm Рассмотрим набор $v_1,\dots,v_n \in V$. Пусть $w_1=\mu_{11}v_1+\dots+\mu_{1n}v_n$, $\dots$, $w_m= \mu_{m1}v_1+\dots+\mu_{mn}v_n$. Рассмотрим набор $\lambda_1,\dots, \lambda_m$. Тогда вектор $w=\sum \lambda_i w_i$ является линейной комбинацией набора $v_i$.
\elm


\dfn[Линейная зависимость] Набор векторов $v_1,\dots,v_n$ называется линейно зависимым, если 0 является нетривиальной линейной комбинацией $v_1,\dots, v_n$, то есть существуют  $\lambda_1, \dots, \lambda_n \in K$ не все равные 0, что
$$0=\lambda_1v_1+\dots+\lambda_n v_n.$$
\edfn

\dfn[Линейная независимость] Набор векторов $v_1,\dots,v_n$ называется линейно независимым, если он не является линейно зависимым, то есть если $\lambda_1, \dots, \lambda_n \in K$ такие, что $$0=\lambda_1v_1+\dots+\lambda_n v_n, \text{ то $\lambda_1=\dots=\lambda_n=0$}.$$
\edfn



\rm Будем говорить, что набор линейно независим, если независим каждый его конечный поднабор.
\erm

\exm \\
0) Набор из одного нуля линейно зависим.\\
1) Пусть $v_1$ и $v_2$ два вектора из $V$. Они линейно зависимы тогда и только тогда, когда они пропорциональны.\\
2) Рассмотрим пространство $K^n$ и набор столбцов
$$e_1=\pmat 1\\0\\ \vdots\\ 0\epmat,\, \dots, e_n=\pmat 0\\ 0 \\ \vdots \\ 1 \epmat.$$
Это линейно независимая система векторов. \\
3) Аналогично в пространстве матриц $M_{m \times n}(K)$ имеется набор матриц $e_{ij}$ вида
$$ e_{ij}=
\bordermatrix{
 & &j&& \cr
 &0&\cdots&\cdots&0\cr
 &\vdots&\ddots && \vdots\cr
i&\vdots& 1 & \ddots& \vdots\cr
 &0&\cdots& \cdots&0
}
$$\\
4) Все мономы $x^{\alpha}$ в кольце  $K[x_1,\dots,x_n]$ линейно независимы\\
5) Набор  $\frac{1}{x-\lambda}$ линейно независим.\\
6) Любой поднабор линейно независимого набора линейно независим.\\


\thrm[О линейной зависимости линейных комбинаций.] Пусть $u_1,\dots,u_m$ и $v_1,\dots,v_n$ два набора векторов. При этом все вектора $u_i$ являются линейными комбинациями $v_j$, то есть $u_i=\sum_{j=1}^n \lambda_{ij}v_j$. Тогда, если $m>n$, то $u_i$ обязательно линейно зависимы.
\ethrm
\proof Индукция по $n$. $n=1$. Все вектора $u_i$ кратны $v_1$ и пропорциональны друг другу.

Будем поступать как в методе Гаусса. Запишем $u_i$ в виде
$u_i=\sum_{j=1}^n \lambda_{ij}v_j$. Если для некоторого индекса $j$ все $\lambda_{ij}=0$, то можно воспользоваться предположением индукции.

Рассмотрим вектор $u_i$, что $\lambda_{in}\neq 0$. Тогда перейдём к набору $$u_s'=u_s - \frac{\lambda_{sn}}{\lambda_{in}}u_i= \sum_{j=1}^{n-1} \mu_{sj} v_j, \,\,\,\, s\neq i.$$
Это набор из $m-1$ элемента, которые суть линейные комбинации $v_1,\dots,v_{n-1}$. Этот набор линейно зависим по индукционному предположению, то есть существуют $\nu_{s}$ не все равные нулю, что
$$0=\sum \nu_s u_{s}'= -\left(\sum\nu_s\frac{\lambda_{sn}}{\lambda_{in}}\right)u_i +\sum \nu_s u_s.$$
Заметим, что не все коэффициенты при $u_s$ равны нулю. Таким образом мы получили нетривиальную линейную зависимость.
\endproof




\dfn Пусть $X$ подмножество векторного пространства $V$. Тогда подпространство, порожденное $X$ -- это наименьшее подпространство, содержащее $X$. Обозначается оно $\lan X \ran$. Это подпространство так же называют линейной оболочкой $X$.
\edfn


\lm Пусть $X\subseteq V$. Тогда линейная оболочка $\lan X \ran$ существует и описывается как множество всех линейных комбинаций элементов из $X$.
$$\lan X \ran = \{ v\in V\,|\, \exists x_1, \dots,x_n \in X, \text{ и } \lambda_1,\dots,\lambda_n \in K, \text{ что } v=\lambda_1x_1+\dots+\lambda_nx_n\}$$
\proof Аналогично лемме про идеалы.
\endproof
\elm

\dfn Будем говорить, что набор $v_{\alpha} \in V$ $\alpha \in I$ является порождающим для $V$, если $\lan \{v_{\alpha}\}_{\alpha \in I}\ran= V$. Иными словами, для любого $v \in V$ существуют $v_{\alpha_1},\dots,v_{\alpha_n}$  и $\lambda_1,\dots,\lambda_n$, что $v=\sum \lambda_i v_{\alpha_i} $.
\edfn

\dfn[Базис] Набор векторов $v_{\alpha}$, $\alpha \in I$ называется базисом пространства $V$, если он является порождающей и линейно независимой системой векторов в $V$. \edfn

\exm\\
1) Набор векторов $e_i \in K^n$ является базисом. Этот базис называют стандартным.\\
2) Набор $e_{ij}$ является базисом $M_{m\times n}(K)$.\\
3) Мономы $x^{\alpha}$ являются базисом $K[x_1,\dots,x_n]$.\\
4) Элементы $1,x, \dots, x^n, \dots$ и $\frac{1}{(x-\lambda)^n}$ по всем $\lambda \in \mb C$ являются базисом в $\mb C(x)$.\\
5) $1,i$ -- базис $\mb C$ над $\mb R$.






\lm[Переформулировки] Пусть $e_1,\dots,e_n$ -- набор элементов пространства $V$.
Тогда следующие свойства эквивалентны: \\
1) $e_1,\dots,e_n$ -- базис $V$.\\
2) $e_1,\dots,e_n$ -- минимальный по включению среди порождающих $V$ наборов векторов.\\
3) Для любого $v\in V$ существуют единственные $\lambda_1,\dots,\lambda_n \in K$, что $v=\lambda_1e_1+\dots+\lambda_ne_n$. Такие элементы $\lambda_1,\dots,\lambda_n$ называются координатами вектора $v$ в базисе $e$.
4) $e_1,\dots,e_n$ -- максимальный по включению набор среди линейно независимых наборов векторов из $V$.
\proof 1)  в 2). Пусть набор не минимален. Тогда выкинем векторок. Например, $v_1$. Но тогда $v_1=\sum_{i\geq 2}\lambda_i v_i$. Это даёт нетривиальную линейную зависимость.

2) в 3). Существование разложения следует из порождаемости. Пусть $\sum \lambda_i v_i =\sum \mu_i v_i$. Тогда либо все $\mu_i=\lambda_i$, либо, скажем, $\mu_1-\lambda_1 \neq 0$. Тогда  $v_1=\frac{1}{\mu_1-\lambda_1}\sum_{i\geq 2} (\mu_i-\lambda_i) v_i$. Тогда из системы можно выкинуть $v_1$ и всё равно будет порождающая система.

3) в 4). Единственность разложения для 0 означает линейную независимость. Покажем максимальность. Пусть система не максимальна. Тогда к ней можно добавить вектор $v\neq 0$ и она останется независимой. Но такого не может быть, так как $v=\sum \lambda_i v_i$, что очевидно даёт линейную зависимость.

4) в 1). Пусть система не порождает $V$. Тогда добавим к ней вектор $v \in V\setminus \lan v_i \ran$. Если система зависима, то есть $\lambda v+ \sum \lambda_i v_i=0$, то либо $\lambda=0$ и тогда получаем, что $v_i$ зависимы, что не так, либо $v=-\lambda^{-1} \sum \lambda_i v_i$, что противоречит определению $v$. Таким образом, система $v,v_1,\dots, v_n$ независима, что противоречит максимальности.
\endproof
\elm

\dfn Пространство называется конечно порождённым, если существует конечная порождающая система $e_1,\dots,e_n$ для $V$.
\edfn

\thrm[Теорема о дополнении до базиса] Пусть $V$ -- конечно порождённое векторное пространство. Тогда любой набор линейно независимых векторов $(u_{\alpha})_{\alpha \in I}$ можно дополнить до базиса при помощи элементов заданного конечного порождающего множества $v_1,\dots,v_n$.
\proof Будем добавлять элементы из порождающего множества к линейно независимой системе до тех пор, пока полученный набор будет оставаться линейно независимым.
Индукция по числу элементов из порождающего множества не лежащих в линейной оболочке $\lan u_{\alpha}\ran$. База. Если все элементы задействованы, то
$$ V=\lan v_1,\dots,v_n \ran \subseteq \lan u_{\alpha}\ran \subseteq V.$$
Следовательно, система $u_{\alpha}$ является порождающей, и  является базисом.

Рассмотрим порождающий набор $v_1,\dots,v_n$. Пусть элемент $v_i$ не лежит в $\lan u_{\alpha}\ran$. Рассмотрим новую систему $\{ u_{\alpha}\}\cup \{v_i\}$. Эта система линейно независима и, очевидно, в её линейной оболочке лежит больше векторов из порождающей системы. Тогда её можно дополнить до базиса.
\endproof
\ethrm

\crl В любом конечно порождённом пространстве есть конечный базис.
\proof Пустое множество линейно независимо. Дополним его до базиса при помощи конечной порождающей системы.
\endproof
\ecrl


\thrm[Теорема о равномощности базисов] Пусть $V$ -- конечно порождённое пространство. Тогда любые два базиса $V$ конечны и равномощны.
\proof Можно считать, что один базис $e_1,\dots,e_n$ конечен. Рассмотрим другой базис $f_{\alpha}$. Тогда по теореме о линейной зависимости линейных комбинаций число независимых элементов в $f_{\alpha}$ не более $n$. Пусть их $m\leq n$. Тогда заметим, что $e_j$ выражаются через $f_i$. Тогда $n\leq m$.
\endproof
\ethrm

\crl Размер любого базиса не больше размера любой порождающей системы и не меньше размера любой независимой системы векторов.
\ecrl

\rm Верны бесконечномерные версии этих теорем. А именно, любую линейно независимую систему можно дополнить до базиса при помощи элементов любой порождающей системы. Любое пространство имеет базис и два разных базиса одного и того же пространства равномощны.
\erm

\dfn[Размерность] Пусть $V$ -- векторное пространство. Размерностью $V$ называется мощность какого-нибудь базиса $V$.
\edfn

\dfn Пространство $V$ называется конечномерным, если в нём есть конечный базис. В противном случае пространство $V$ называется бесконечномерным. Класс конечномерных пространств совпадает с классом конечно порождённых пространств.
\edfn

\thrm Любое подпространство $U$ конечномерного пространства $V$ конечномерно и $\dim U \leq \dim V$ и равенство достигается только в случае $U=V$.
\ethrm
\proof Пусть в пространстве $U$ нет конечного базиса. Тогда рассмотрим линейно независимую систему $u_1,\dots,u_m$ в $U$ самого большого размера. Такая есть, так как линейно независимая система в $U$ лежит в $V$ и, следовательно, ограничена размерностью $V$. Если $\lan u_1,\dots,u_m \ran =U$, то теорема доказана. Иначе рассмотрим вектор $u_{m+1}\in U$, что $u_{m+1}\notin \lan u_1,\dots,u_m \ran$. Новая система $u_1,\dots,u_m,u_{m+1}$ линейно независима и больше предыдущей. Противоречие. Значит в $U$ есть конечный базис. Этот базис есть линейно независимая система внутри $V$ и, как мы уже отмечали, имеет размер меньший, чем у базиса $V$ и равный, только при $U=V$ (иначе можно нетривиально дополнить).
\endproof

\lm Пусть размерность $V$ равна $n$, а $f_1,\dots, f_n \in V$ набор векторов с одним из условий:\\
1)  $f_1,\dots, f_n$ -- линейно независимая система\\
2) $f_1,\dots, f_n$ -- порождает $V$.\\
Тогда $f_1,\dots, f_n$ -- базис $V$.
\elm




\exm\\
1) Размерность пространства $M_{m\times n}(K)$ равна $mn$. Базисом является $e_{ij}$. Такой базис называется стандартным\\
2) Размерность пространства $K[x]_{\leq n}=n+1$. Базис $1,x,\dots,x^n$.\\
3) Размерность $\mb C$ над $\mb R$ равна 2.\\
4) Размерность $K[x]/p(x)=\deg p(x)$.\\

Как определить по системе векторов $v_1,\dots,v_k$ из $V$, что она линейно независима? Пусть заранее известен $e_1,\dots,e_n$ базис $V$. Разложим  $v_j=\sum a_{ij} e_i$. Тогда сумма $0=\sum \lambda_j v_j=\sum \sum \lambda_j a_{ij} e_i$ обнуляется тогда и только тогда, когда $\lambda_i$ решение системы уравнений $A\lambda =0$. Итого, вектора линейно независимы тогда и только тогда, когда у этой системы нет нетривиальных решений и зависимы иначе.

Как определить, что система порождающая? Для этого необходимо и достаточно получить любой базисный элемент $e_i$, то есть добиться существования линейной комбинации $\sum \lambda_iv_i - e_j =0$ по всем $j$. Это эквивалентно тому что система уравнений с расширенной матрицей $(A|e_i)$ имеет решение (в последнем случае $e_i$ -- это столбец из стандартного базиса).

Приведём пример использования понятия линейной зависимости. Пусть $\alpha \in \mb R$ корень многочлена с рациональными коэффициентами $x^3-x-1=0$. Вопрос, а верно ли, что $\beta=\alpha^2$ тоже удовлетворяет какому-то уравнению с рациональными коэффициентами (такие числа называют алгебраическими), и, если да, то какому.

Для этого рассмотрим пространство $V= \lan 1,\alpha, \alpha^2, \dots, \ran$. Заметим, что это пространство порождено $1,\alpha,\alpha^2$. Покажем, что это его базис. Для этого заметим, что многочлен $x^3-x-1$ неприводим над  $\mb Q$ и если бы была нетривиальная линейная зависимость $a_2 \alpha^2+ a_1 \alpha+ a_0=0$, то $x^3-x-1$ имел бы общий делитель с многочленом $a_2 x^2+ a_1 x+ a_0$, что противоречит неприводимости.

Теперь заметим, что элементы $1, \beta, \beta^2, \beta^3$ лежат в $V$ и, следовательно линейно зависимы. Тогда нулевая линейная комбинация $b_3 \beta^3+b_2 \beta^2+b_1 \beta + b_0=0$ даёт желаемое уравнение. Найдём её. Для этого заметим, что $\beta^2=\alpha(\alpha+1)$, а $\beta^3=(\alpha+1)^2$. Строим матрицу
$$A =\pmat
1&0&0&1\\
0&0&1&2\\
0&1&1&1
\epmat
.$$
Заменяем матрицу на эквивалентную
$$\pmat
1&0&0&1\\
0&1&0&-1\\
0&0&1&2
\epmat$$
и находим решение $-1,1,-2,1$. Итого $\beta$ удовлетворяет уравнению
$$\beta^3-2\beta^2+\beta-1.$$

\section{Линейные отображения}

\dfn[Линейное отображение] Пусть $U,V$ -- два векторных пространства над полем $K$. Отображение $L\colon U \to V$ называется линейным, если\\
1) $\forall a,b \in U$ верно, что $L(a+b)=L(a)+L(b)$.\\
2) $\forall a \in U$, $\lambda \in K$ верно, что $L(\lambda a)=\lambda L(a)$.
\edfn

\dfn Линейное отображение называется моно-, эпи-, изоморфизмом, если оно инъективно, сюръективно, биективно. Линейное отображение $V \to V$ называется эндоморфизмом пространства $V$, а обратимый эндоморфизм $V$ -- автоморфизмом $V$.
\edfn

\exm\\
0) Рассмотрим пространство $K^n$ и набор элементов $a_1,\dots,a_n$. Определим отображение $K^n \to K$ по правилу $$x \to (a_1,\dots, a_n) \cdot x= \sum a_ix_i.$$
Это отображение линейно.\\
1) Пусть задана матрица $A \in M_{m\times n}(K)$. Тогда отображение $K^n \to K^m$, заданное как $x \to Ax$ линейно.\\
2) Пусть задан многочлен $p(x) \in K[x]$. Тогда отображение $K[x] \to K[x]$ $f(x)\to p(x)f(x)$ линейно.\\
3) Пусть $\lambda \in K$. Тогда отображение $f(x) \to f(\lambda)$ линейно.\\
4) Пусть $p(x) \in K[x]$ -- многочлен. Тогда отображение $K[x] \to K[x]$, заданное как  $f(x)\to f(p(x))$ линейно.\\
5) Отображение взятия производной $f \to f'$ линейно во всех пространствах, где оно определено (например, как отображение из $K[x] \to K[x]$).\\
6) Отображение сопряжения $\ovl{\cdot} \colon \mb C \to \mb C$ линейно над $\mb R$, но не линейно над $\mb C$.\\
7) Пусть $g(x,y)$ непрерывная функция из $[0,1]^2 \to \mb R$. Тогда отображение $f(x) \to \int_{0}^1 f(y)g(x,y)dy$ линейно  как отображение $C([0,1])\to C([0,1])$ .\\




Итак, благодаря примерам, мы поняли, что отображение $K^n \to K^m$, заданное с помощью матрицы $A$ это частный случай линейного отображения. Таким образом, решение системы линейных уравнений можно понимать как нахождение прообраза элемента при линейном отображении. В дальнейшем мы увидим, что общий случай можно свести  и разными способами к указанному частному случаю. Но прежде посмотрим, на общие качественные свойства линейных отображений.



\lm[Базовые свойства]
0) Пусть $V$ -- векторное пространство, а $\lambda \in K$. Тогда отображение $x \to \lambda x$ является линейным отображением. \\
1) Пусть $L_1$ и $L_2$ -- два линейных отображения $V \to W$. Тогда $L_1+L_2$ -- их поточечная сумма -- тоже линейное отображение.\\
2) Пусть $\mu \in K$ и $L\colon V \to W$ -- линейное отображение. Тогда $\mu L$ -- тоже линейное отображение.\\
3) Пусть $L_1\colon V_1\to V_2$, а $L_2\colon V_2 \to V_3$. Тогда $L_1\circ L_2$ -- линейное отображение $V_1 \to V_3$.\\
4) Пусть $L\colon V \to W$ линейное отображение. Тогда обратное к нему линейно.\\
5) Для композиции линейных отображений выполнена дистрибутивность. А именно, если $L_1,L_2 \colon V \to W$, а $L_3 \colon W \to U$, а $\lambda_1, \lambda_2 \in K$, то
$$(\lambda_1 L_1+ \lambda_2 L_2 ) \circ L_3 = \lambda_1 (L_1 \circ L_3)+ \lambda_2 (L_2 \circ L_3).$$\\
6) Так же, если $L_1 \colon V \to W$, а $L_2,L_3 \colon W \to U$, и $\lambda_1, \lambda_2 \in K$, то
$$ L_1 \circ (\lambda_1 L_2+ \lambda_2 L_3 ) = \lambda_1 (L_1 \circ L_2)+ \lambda_2 (L_1 \circ L_3).$$
7) Пусть $L\colon V \to W$. Тогда $\Ker L$ и $\im L$   -- это подпространства $V$ и $W$ соответственно.
\elm
\proof
\endproof

\crl Множество всех отображений $\Hom(U,V)$ является векторным пространством.
\proof Необходимо дополнительно проверить, что сложение ассоциативно, коммутативно, любой элемент имеет противоположенный и что умножение на скаляр ассоциативно. Эти свойства верны потому что они верны в $V$.
\endproof
\ecrl






Основная теорема, которая говорит про устройство линейных отображений, следующая:
\begin{thm}
Пусть $V_1$, $V_2$ --- векторные пространства над полем $K$. Пусть $e_1,\dots,e_n$ - базис $V_1$, а $f_1,\dots,f_n$ --- набор каких-то векторов из $V_2$. Тогда существует единственное отображение $L\colon V_1 \to V_2$, что $L(e_i)=f_i$. Пусть вектор $v\in V_1$ раскладывается по базису $v=\sum \lambda_i e_i$, то $L(v)=\sum \lambda_i f_i$.
\proof Очевидно, что отображение должно быть задано такой формулой и это показывает единственность.
Осталось показать, что оно линейно. Действительно, если два вектора $v=\sum \lambda_i e_i$ и $u=\sum \mu_i e_i$, то $u+v= \sum (\lambda_i +\mu_i) e_i$ есть разложение для суммы. Осталось посчитать $L(u+v)$ и раскрыть скобки.
\endproof
\end{thm}

\crl Если $L$ переводит некоторый базис $V$ в базис $W$, то $L$ обратимо. Обратно, если линейное отображение $L \colon V \to W$ является изоморфизмом, то $L$ переводит любой базис $V$ в базис $W$.
\proof
Пусть $e_1,\dots,e_n$ -- базис $V$ и $L(e_1),\dots,L(e_n)$ -- базис $W$. Тогда есть единственное линейное отображение $T \colon W \to V$, которое переводит $L(e_i)$ в $e_i$. Заметим, что композиция $U\circ L$ переводит $e_i \to e_i$. Тогда по единственности такая композиция тождественна. Аналогично композиция в другом порядке тождественна. Тогда $T$ -- обратное отображение к $L$.
 В другую сторону. Пусть $e_1,\dots,e_n$ -- базис $V$. Покажем, что $L(e_1),\dots,L(e_n)$ -- базис $W$.  Пусть есть линейная зависимость  $$0=\lambda_1 L(e_1)+\dots+\lambda_n L(e_n).$$
Тогда
$$0=L^{-1}(0)=L^{-1}(L(\lambda_1 e_1+\dots+\lambda_n e_n))=\lambda_1 e_1+\dots+\lambda_n e_n.$$
Тогда все $\lambda_i$ равны 0. Покажем, что $L(e_i)$ порождают $W$. Рассмотрим вектор $w\in W$. Тогда $$w=L(v)=L\left(\sum \lambda_i e_i\right)=\sum \lambda_i L(e_i).$$
\endproof
\ecrl

Это позволяет нам установить следующее

\crl Два конечномерных пространства $V$ и $W$ изоморфны тогда и только тогда, когда $\dim V=\dim W$. В частности $V\cong K^{\dim V}$. Более того, задание базиса равносильно заданию изоморфизма $V\to K^{\dim V}$.
\proof
Рассмотрим в качестве $W$ пространство $K^n$ и стандартный базис $e_1,\dots,e_n$. Тогда базису $f_1,\dots,f_n$ в $V$ соответствует линейное отображение переводящее $f_i \to e_i$. Это отображение берёт вектор $v \in V$, раскладывает его как $v=\sum_{i=1}^n \lambda_i f_i$ и переводит его в столбец координат $\lambda_i$. Обратно, если есть изоморфизм $L\colon V \to K^n$, то можно рассмотреть обратное отображение $L^{-1}$. Тогда $L^{-1}(e_i)$ -- это базис $V$.
\endproof
\ecrl

\dfn
Изоморфизм между пространством $V$ и пространством $K^n$ называется линейной системой координат на пространстве $V$. Мы знаем что любая линейная система координат происходит из некоторого базиса. Отображение, сопоставляющее вектору $v$ его $i$-ую координату, называется $i$-ой координатной функцией.
\edfn

Сделаем отступление про модули над кольцом. Мы видели, что все векторные пространства изоморфны пространствам столбцов. Изоморфизм строится благодаря наличию базиса.
Сейчас я приведу примеры модулей и мы увидим, что модули могут иметь самый разнообразный облик. Итак


\exm\\
1) Пусть $R$ -- кольцо. Тогда множество $R^n$ является модулем над $R$ вместе с покоординатным сложением и умножением (и даже имеет базис, то есть порождающую линейно независимую систему). Такой модуль называется свободным модулем ранга $n$\\
2) Пусть $M$ -- абелева группа. Тогда на $M$ существует единственная структура $\mb Z$ модуля.\\
3) Пусть $I$ идеал кольца $R$. Тогда можно ограничить умножение на элементы из $R$ со всего $R$ до  $I$ -- это и задаёт структуру модуля на $I$ (быть идеалом это то же, что быть подмодулем в $R$ при естественном определении этого слова).

Верно ли что каждый модуль изоморфен модулю вида $R^n$? Пусть $R=\mb Z$. Тогда модуль $\mb Z/p$ точно не изоморфен $\mb Z$, потому что он конечен. Но можно заметить, что любой идеал $I \leq \mb Z$ изоморфен $\mb Z$. Действительно, $I$ имеет вид $n\mb Z$. Изоморфизм осуществляется умножением на $n$.

Может быть так происходит всегда? Ответ -- нет. Покажем, что идеал $(x,y) \leq K[x,y]$ имеет не менее двух порождающих его элементов, но при этом между любыми из этих элементов есть нетривиальная зависимость.

Прежде всего найдём зависимость, потому что это легче. Пусть есть два элемента из $I=(x,y)$ -- скажем, $f,g$. Тогда $0=fg+(-f)g$. Просто они лежат внутри кольца $R$, какой хотите идеал берите.
Пусть теперь $(x,y)= (g)$. Тогда степень $g$ обязана быть равной 1, чтобы получить в качестве кратного элемент $x$. Пусть $g= \alpha x+ \beta y$, $\alpha, \beta \in K$. Но тогда единственные многочлены степени $1$ кратные $g$ это $\lambda g$, $\lambda \in K$. Это одномерное пространство в $\lan x, y\ran $ -- двумерном над $K$. Итого, все элементы не получить. Такие дела.  \\



Вернёмся к векторным пространствам. {\color{red} Внимание!} Далее я всегда буду предполагать, что все пространства конечномерны. Можно ли как-то попроще разобраться с нашими модельными примерами конечномерных пространств --  пространствами $K^n$?

\crl
Все линейные отображения $L\colon K^n \to K^m$ имеют вид $L(x)=Ax$, где $A$ --- матрица $m\times n$
\proof Если задано линейное отображение $L$, то по нему определяется матрица $A$ составленная из столбцов $L(e_i)$, где $e_i$ -- это стандартный базис $K^n$. Тогда отображения $x \to L(x)$ и $x \to Ax$ оба переводят $e_i \to L(e_i)$, то есть совпадают на базисе и, значит, совпадают везде.
\endproof
\ecrl


\dfn Пусть $U \leq V$. Рассмотрим факторгруппу $V/U$. Введём на ней структуру векторного пространства по следующему правилу $\lambda \cdot \ovl{v}=\ovl{\lambda v}$.
\edfn

Можно заметить, что верна теорема
\thrm[Об изоморфизме] Пусть $V$ и $W$ два векторных пространства, а $L\colon V \to W$ -- линейное отображение. Тогда $V/\Ker L \cong \Im L$.
\ethrm

\dfn Рангом линейного отображения $L\colon V \to W$ называется размерность его образа $\rank L= \dim \im L$. Рангом матрицы $A\in M_{m\times n}(K)$ называется ранг соответствующего ей линейного отображения $K^n \to K^m$. Иными словами ранг матрицы это размерность пространства, порождённого столбцами этой матрицы.
\edfn


\begin{thm}
Пусть $L\colon V \to W$ --- линейное отображение между конечномерными пространствами. Тогда
$$\dim V= \dim \Ker L +  \dim \im L= \dim \Ker L +\rank L.$$
\end{thm}
\proof Выберем $e_1,\dots, e_k$ -- базис $\Ker L$ и дополним его элементами $e_{k+1},\dots,e_n$ до базиса всего $V$. Я утверждаю, что $L(e_{k+1},\dots,L(e_n))$ являются базисом образа. Действительно, образ $L$ порождён $L(e_1),\dots, L(e_n)$, но первые $k$ элементов этого набора нули. Поэтому их можно исключить, что означает, что $L(e_{k+1}), \dots, L(e_n)$ порождают образ $L$.

Пусть сумма $\sum_{i=k+1}^n \lambda_i L(e_i)=0$. Тогда элемент $\sum_{i=k+1}^n \lambda_i e_i $ лежит в ядре $L$. Отсюда получаем, что имеет место равенство $$\sum_{i=k+1}^n \lambda_ie_i = \sum_{i=1}^k \mu_i e_i$$
которое приводит к нулевой линейной комбинации всех $e_i$. Но $e_i$ -- базис $V$ и поэтому все коэффициенты, в частности, $\lambda_i$ равны нулю.

Итого, размерность ядра равна $k$, размерность образа равна $n-k$, что в сумме и даёт $n$.
\endproof

\crl Пусть $U \leq V$. Тогда $\dim V/U= \dim V - \dim U$.
\ecrl

\dfn Коразмерностью пространства $U \leq V$ называется число $\dim V/U$.
\edfn

\crl Для любого подпространства $U \leq K^n$ коразмерности $d$ существует линейное отображение $L\colon K^n \to K^d$, такое, что $U=\Ker L$. Иными словами, такое подпространство задаётся $d$ уравнениями.
\ecrl
\proof Рассмотрим цепочку линейных отображений $K^n \to K^n/U \stackrel{\sim}{\longrightarrow} K^d$.
\endproof

\crl[Принцип Дирихле для линейных отображений] Пусть $V$ и $W$ два пространства размерности $n$ и $L \colon V \to W$ -- линейное отображение между ними. Тогда $L$ -- сюръективно тогда и только тогда, когда $L$ -- инъективно.
\ecrl




\section{Линейные уравнения и качественное описание их решений}
Итак, основную задачу линейной алгебры можно обозначить следующим образом: пусть $L\colon V \to W$ -- линейное отображение. Требуется для конкретного $y\in V$ описать множество решений уравнения $Lx=y$ и, с другой стороны, описать все $y\in W$, для которых есть решение есть.

Попробуем качественно ответить на эти вопросы. Прежде всего множество $y$, для которых уравнение $Lx=y$ разрешимо есть, как мы знаем, подпространство $W$. Его базис можно найти следующим образом -- взять базис $V$ -- $e_1,\dots,e_n$, рассмотреть порождающий набор для образа $L(e_1),\dots,L(e_n)$ и выбрать из него базис образа.

Задача описания прообраза для элемента $0\in W$ по определению есть задача описания ядра $L$, которую мы разберём чуть позже. Пусть дан некоторый $y\in \im L$. Тогда все решения уравнения $Lx=y$ образуют класс $x_0+\Ker L$.



\dfn Аффинным подпространством векторного пространства $V$ называется подмножество вида $U+v_0$, где $U$ -- линейное подпространство $V$, а $v_0\in V$, или, что тоже самое -- смежный класс относительно некоторого линейного подпространства.
\edfn

\lm Пусть $W$ -- аффинное подпространство $V$. Тогда существует единственное подпространство $U$, что $W=U+v_0$. Вектор $v_0$ однозначно определён с точностью до  элементов из $U$.
\elm
\proof Рассмотрим множество $U'=\{w-v\,|\, w,v \in W\}$. Очевидно оно однозначно определяется множеством $W$. Пусть $W=U+v_0$. Необходимо показать, что $U=U'$. Прежде всего включение $U'\subseteq U$ получается исходя из того, что $x-v_0-y+v_0=x-y \in U$. В другую сторону. Пусть $x\in U$. Тогда $x=(x+v_0)-v_0\in U'$. Так как $W$ -- это смежный класс по $U$, то в качестве $v_0$ можно взять любой элемент из $W$. Но все они отличаются на элемент из $U$ по определению.
\endproof

\dfn Пусть $W=U+v_0$ аффинное подпространство $V$. Тогда определим $\dim W=\dim U$.
\edfn

Сформулируем полученное в виде теоремы.

\thrm Пусть $L\colon V \to W$ -- линейное отображение и $\dim V= n$, а $\dim W=m$. Тогда множество решений  $Lx=y$ либо пусто, либо образует аффинное подпространство вида $x_0+ \Ker L$ размерности $$\dim \Ker L = n - \rank L\geq n -m.$$
\ethrm

 Рассмотрим модельный случай, а именно, когда $L$ -- это некоторое отображение $K^n \to K^m$, которое, как мы знаем, задаётся некоторой матрицей $A \in M_{m\times n}(K)$. Тогда задача описать все решения уравнения $Ax=y$ равносильна решению системы линейных уравнений, которую  мы уже умеем решать методом Гаусса.

Прежде всего ответим на вопрос: верно ли, что число параметров в методе Гаусса равно размерности соответствующего аффинного подпространства?

\thrm Пусть дана система линейных уравнений $Ax=b$ c непустым множеством решений $W$. Тогда число параметров, задающих общее решение системы после использования метода Гаусса, равно размерности соответствующего пространства решений.
\ethrm
\proof Пусть переменные в методе Гаусса $x_{i_1}, \dots,x_{i_s}$ -- любые, а остальные выражаются через них следующим образом
$$x_{i}=t_i+ \sum c_{ij} x_{i_j}, \text{ при всех $i \notin \{i_1,\dots, i_k\}$}.$$
Тогда набор $x_{i_1}=\dots=x_{i_k}=0, \,\, x_i=t_i, \text{ при всех $i \notin \{i_1,\dots, i_s\}$}$ является решением системы. Обозначим этот столбец за $v_0$. Вычитая  $v_0$ произвольного решения системы $x$ мы должны получить произвольный элемент ядра. С другой стороны, координаты $x-v_0$ определяются по $x_{i_1}, \dots,x_{i_s}$  по формуле
$$x_{i}=\sum c_{ij} x_{i_j}, \text{ при всех $i \notin \{i_1,\dots, i_s\}$}.$$
 Заметим, что тогда мы получили взаимооднозначное линейное отображение $K^s \to \Ker A$, то есть изоморфизм. Следовательно, $\dim \Ker A = s$, что и есть число параметров.
\endproof




Как же найти частное решение $x_0$ системы $Ax=y$? Для этого надо решить систему методом Гаусса и подставить конкретные значения для свободных переменных. Проще всего подставить нули, что мы и сделали при доказательстве теоремы про число параметров. Как теперь описать ядро? Метод Гаусса даёт параметрическое описание ядра. Нас же может интересовать базис ядра. Заметим, что подобное параметрическое описание даёт  изоморфизм $K^s \to \Ker A$. При изоморфизме базис переходит в базис, следовательно взяв стандартный базис $K^s$ в качестве его образов получим базис ядра. Разберём пример:

Пусть $$A=\pmat
1 & 2 & -1 & 2\\
0& 1& 0 & 1\\
1& 0 & -1 & 0\epmat, \,\,b= \pmat -1\\ -1\\ 1 \epmat.$$
Решаем систему
$$\pmat
1 & 2 & -1 & 2 & -1\\
0& 1& 0 & 1 & -1\\
1& 0 & -1 & 0 & 1\epmat \sim
\pmat 1& 0 & -1 & 0 & 1 \\
0& 1& 0 & 1 & -1\\
0 & 2 & 0 & 2 & -2
\epmat \sim \pmat 1& 0 & -1 & 0 & 1 \\
0& 1& 0 & 1 & -1\\
0 & 0 & 0& 0 & 0
\epmat.$$
Параметра два -- это $x_3,x_4$. Берём $x_3=x_4=0$. Тогда $x_2=-1$, $x_1=1$. Итого $$v_0=\pmat 1\\ -1\\0 \\ 0 \epmat.$$
Ищем базис ядра. Для этого переходим к системе
$$\pmat
1& 0 & -1 & 0 & 0 \\
0& 1& 0 & 1 & 0\\
0 & 0 & 0& 0 & 0
\epmat.$$
Подставляем стандартные вектора из $K^2$. Берём $x_3=1$, $x_4=0$. Тогда $x_2=0,x_1=1$. Берём $x_3=0$, $x_4=1$. Тогда $x_1=0,x_2=-1$. Итого базис состоит из двух векторов
$$v_1=\pmat 1\\0\\1\\0 \epmat, \, v_2=\pmat 0 \\ -1 \\ 0 \\ 1 \epmat.$$
Общее решение описывается как
$$\pmat 1\\ -1\\0 \\ 0 \epmat+ \lan \pmat 1\\0\\1\\0 \epmat, \pmat 0 \\ -1 \\ 0 \\ 1 \epmat \ran.$$

\section*{Дополнительно: больше слов про аффинность и немного другое доказательство про число параметров в методе Гаусса}

\dfn Пусть $U$ и $V$ векторные пространства. Тогда отображение $T\colon U \to V$ называется аффинным, если существует линейное отображение  $L$ и вектор $v_0 \in V$, что $T(x)=L(x)+v_0$.
\edfn

\rm Отображение $L$ и вектор $v_0$
определены однозначно Прежде всего $v_0=T(0)$. Далее $L(x)=T(x)-v_0$. Говорят, что отображение $L$ -- это дифференциал $T$ и обозначают $L=dT$. Это имеет прямое отношение к дифференциалу из математического анализа.
Если $T \colon U \to V$ аффинное с дифференциалом $L$, то $\im T$ аффинное и $\dim \im T= \dim \im L$, так как $T(U)=L(U)+v_0$.
\erm


\begin{thmm} Пусть $T \colon U \to V$ инъективное аффинное отображение. Тогда $\dim U = \dim \im T$.
\end{thmm}
\proof Если отображение $T$ инъективно, то и его дифференциал инъективен ( иначе бы $T$ склеивало те же точки, что и $L$). Тогда $\dim U = \dim L(U)= \dim T(U)$.
\endproof

\crl Пусть дана система линейных уравнений $Ax=b$ и множество её решений $W$ не пусто. Тогда число параметров задающих общее решение системы после использования метода Гаусса равно размерности $W$.
\proof Пусть переменные в методе Гаусса $x_{i_1}, \dots,x_{i_s}$ -- любые, а остальные выражаются через них следующим образом
$$x_{i}=t_i+ \sum c_{ij} x_{i_j}, \text{ при всех $i \notin \{i_1,\dots, i_k\}$}.$$
Это задаёт аффинное взаимооднозначное соответствие между $K^s \to W$. Тогда $s=\dim W$.
\endproof
\ecrl







\section{Матрица линейного отображения}
Итак, у нас есть модельная задача: пусть $A \colon K^n \to K^m$ линейное отображение и элемент $y \in K^m$. Необходимо описать прообраз элемента $y$, то есть решить систему линейных уравнений $Ax=y$. Про эту задачу мы понимаем всё.
Наша текущая задача -- научиться сводить общую задачу к модельной. Однако тут возникает нюанс -- это можно сделать разными способами. Точнее:

\dfn
Пусть $V_1$, $V_2$ - векторные пространства над полем $K$ с базисами $e_1,\dots, e_n$ и $f_1,\dots, f_m$ соответственно. Пусть $L\colon V_1\to V_2$ - линейное отображение. Рассмотрим диаграмму:

\begin{center}
\begin{tikzpicture}
\node (A) at (1, 1) {$V_1$};
\node (B) at (0, 0) {$K^n$};
\node (C) at (3, 1) {$V_2$};
\node (D) at (4, 0) {$K^m$};
\path[->,font=\scriptsize,>=angle 60]
(A) edge node[above]{$L$} (C)
(A) edge node[above,rotate=45]{$\sim$} (B)
(C) edge node[above,rotate=-45]{$\sim$} (D);
\path[->,font=\scriptsize,>=angle 45]
(B) [bend left] edge node[below]{$A$}  (D);
\end{tikzpicture}
\end{center}

Сквозная композиция $f \circ L \circ e^{-1}$ задаёт линейное отображение из $K^n\to K^m$, то есть матрицу $A$. Матрица $A$ называется матрицей линейного отображения $L$ в базисах $e_1,\dots, e_n$ и $f_1,\dots, f_m$.

Это даёт следующий рецепт для вычисления матрицы $A$ -- $i$-ый столбец матрицы $A$ состоит из координат $L(e_i)$ в базисе $f_1,\dots, f_m$.
\edfn


\exm \\
0) Пусть $V$ векторное пространство с базисом $e_1,\dots,e_n$. Тогда матрица тождественного отображения $\id \colon V \to V$ из базиса $e$ в базис $e$ есть единичная матрица
$$E_{n}=\pmat
1&&\\
&\ddots& \\
& &1
\epmat.$$
1) Рассмотрим произвольный базис $V$ $v_1,\dots,v_n$. Тогда  матрица отображения $V\to V$, заданного как  $x \to \lambda x$  из этого базиса  в себя равна
$$ \lambda E_n=\pmat \lambda & & \\
 & \ddots &\\
&  & \lambda
\epmat.$$
2) Рассмотрим пространство $V=K[x]_{\leq 3} $, где $\chr K \neq 2,3$ и отображение $L\colon V \to K^3$, заданное как $f \to (f(-1),f(0),f(2))$. Его матрица в стандартных базисах имеет вид
$$\bordermatrix{
    &1& x&x^2&x^3\cr
e_1\, &1&-1&1  &-1 \cr
e_2\, &1& 0&0  &0  \cr
e_3\, &1& 2&4  &8
}.$$
Найдём ядро этого отображения. Для этого сначала решим однородную систему
$$\pmat 1&-1&1  &-1 \\
1& 0&0  &0  \\
1& 2&4  &8
\epmat \sim
\pmat 1& 0&0  &0  \\
0&-1&1  &-1 \\
0& 2&4  &8 \epmat  \sim
\pmat 1& 0&0  &0  \\
0&-1&1  &-1 \\
0& 0&6  &6 \epmat$$
Размерность ядра матрицы единица. Оно порождено вектором $\pmat 0\\ -2\\-1\\1\epmat $, которому соответствует многочлен $$x^3-x^2-2x=x(x+1)(x-2).$$ Что, в общем, и ожидалось исходя из наших знаний про многочлены.\\
3) Рассмотрим линейное отображение из пространства $K[x]_{\leq n} \to K[x]_{\leq n}$ заданное по правилу $f(x) \to f(x+1)$. Найдём его матрицу. Рассмотрим базис $1,\dots, x^n$ справа и слева. Получим матрицу $a_{i,j}=C_{j-1}^{i-1}$
$$
\pmat
1& 1& \dots & 1  & 1 \\
& 1  & \dots & C_{n-1}^{n-2}& C_{n}^{n-1}\\
 &  & \ddots &\vdots &\vdots \\
& &  & 1  & C_{n}^{1} \\
 &  & &  &1
\epmat
$$
4) А ещё для того же отображения можно выбрать базисы $1,x,\dots,x^n$ и $1,x+1,\dots,(x+1)^n$. Получится матрица
$$E_{n+1}=\pmat
1&&\\
&\ddots& \\
& &1
\epmat.$$
Видно, что матрица линейного отображения очень чувствительна к замене базиса.

\lm Пусть $L_1,L_2\colon U \to V$ и $e$ -- базис $U$, а $f$ -- базис $V$. Тогда матрица $L_1+L_2$ в базисах $e$ и $f$ есть сумма матриц $L_1$ и $L_2$. Аналогично про домножение на скаляр.
\elm
\proof Пусть $L_1 e_j= \sum A_{ij}f_i$, а $L_2=\sum B_{ij}f_i$. Тогда $L_1(e_j)+L_2(e_j)=\sum (A_{ij}+B{ij})f_i$, что и доказывает требуемое.
\endproof

\crl Пусть пространства $U$ и $V$ имеют размерности $n$ и $m$. Допустим выбраны базисы $e_1,\dots,e_m$ и $f_1,\dots,f_m$ пространств $U$ и $V$ соответственно. Тогда сопоставление линейному отображению $L \in \Hom(U,V)$ его матрицы в указанных базисах есть изоморфизм
$$\Hom(U,V) \stackrel{\sim}{\longrightarrow} M_{m\times n}(K).$$
\ecrl





Пусть есть две матрицы $A\in M_{n\times m}(K)$ и $B\in M_{m\times l}(K)$. Какая матрица соответствует композиции линейных отображений, построенных по $A$ и $B$?

\dfn
Определим произведение $C=A\cdot B$ следующим образом:
$$C_{ij}= \sum_{1\leq k\leq m} A_{ik} B_{kj}$$
Иными словами надо взять $i$-ую строку $A$, $j$-ый столбец $B$, и их перемножить.
\edfn

\rm Можно смотреть на произведение матриц немного по другому. Точнее, $j$-ый столбец $AB$ получается как произведения $j$-го столбца $B$ на матрицу $A$. Это тоже самое, что взять линейную комбинацию столбцов матрицы $A$ с коэффициентами из $j$-го столбца матрицы $B$. Аналогично $i$-ая строка $AB$ это линейная комбинация строк $B$ с коэффициентами из $i$-ой строки $A$. Произведение матриц ассоциативно, так как ассоциативна композиция линейных отображений.
\erm

\thrm Пусть $U$, $V$, $W$ пространства с базисами $e_1,\dots,e_n$, $f_1,\dots,f_m$, $g_1,\dots,g_l$. Пусть $L\colon U \to V$ и $T\colon V \to W$ два линейных отображения. Пусть $A$ и $B$ матрицы $T$ и $L$ в парах базисов $v$,$w$  и $u$, $v$. Тогда матрица $T \circ L$ в базисах $u$, $w$ равна произведению $AB$.
\proof По определению это свойство верно, если речь  идёт о композиции линейных отображений между $K^n$ в стандартных базисах. Матрица композиции $T \circ L$ по определению соответствует линейному отображению
$$g\circ T \circ L \circ e^{-1}=(g \circ T \circ f^{-1}) \circ (f \circ L \circ e^{-1})=AB.$$
\endproof
\ethrm

Какая матрица соответствует изоморфизму? Та, которая осуществляет изоморфизм пространств $K^n$ и $K^m$.



\dfn Матрица $A \in M_{m\times n}$ называется обратимой, если существует $B \in M_{n\times m}$, что $AB=E_m$, $BA=E_n$.  Матрица $B$ называется обратной к $A$ и обозначается $A^{-1}$.
\edfn



\crl Если матрица $A \in M_{n\times m}$ обратима, то $n=m$. Обратная матрица единственна.
\proof Обратимая матрица осуществляет изоморфизм пространства $K^n$ и $K^m$.  Но если пространства изоморфны, то их размерности равны, то есть $n=m$.
\endproof
\ecrl

\crl Квадратная $A\in M_n(K)$ обратима тогда и только тогда, когда $\rank A =n$. Обратимые матрицы так же называются невырожденными.
\proof Следует из принципа Дирихле для линейных отображений.
\endproof
\ecrl

\dfn Пусть $V$ -- пространство. Определим полную линейную группу $\GL(V)$ как множество всех изоморфизмов из $V\to V$ относительно композиции. Если $V=K^n$, то для $\GL(K^n)$ имеется специальное обозначение: $\GL_n(K)$. Элементы из $\GL_n(K)$ -- это обратимые матрицы.
\edfn

\exm\\
0) Обратная к единичной матрице -- это единичная матрица.\\
1) Очевидно отображение $L \colon f(x) \to f(x+1)$ на пространстве $K[x]_{\leq n}$ имеет обратное $U \colon f(x)\to f(x-1)$. Матрица первого отображения -- $A$ нам известна. Матрица второго находится аналогично и равна
$$
B=\pmat
1& -1& \dots & (-1)^{n-1}  & (-1)^n \\
& 1  & \dots & (-1)^{n-2}C_{n-1}^{n-2}& (-1)^{n-1}C_{n}^{n-1}\\
 &  & \ddots &\vdots &\vdots \\
& &  & 1  & -C_{n}^{1} \\
 &  & &  &1
\epmat
$$
Тогда $A\cdot B=E_{n+1}$. Что же это значит? Посмотрим, что даёт это очевидное равенство. Перемножим последний столбец матрицы $B$ и первую строчку $A$. Получим
$$0=\sum_{i=0}^{n}(-1)^{i}C_{n}^{i}.$$
Посмотрим, что получится, если взять вторую строчку $A$ при $n \geq 2$
$$0=\sum_{i=0}^{n}(-1)^{i}iC_{n}^{i}.$$
Вообще тождества, происходящие из очевидных равенств с многочленами -- очень широкая степь.\\

Настала пора ответить на следующий вопрос: что происходит с матрицей линейного отображения, если мы заменим базисы пространств?



\dfn Рассмотрим пространство $V$ размерности $n$ и два базиса $e_1,\dots,e_n$ -- старый и $f_1,\dots,f_n$ -- новый. Имеем связанную с ними диаграмму.
\begin{center}
\begin{tikzpicture}
\node (A) at (1, 1) {$V$};
\node (B) at (0, 0) {$K^n$};
\node (C) at (-1, 1) {$K^n$};
\path[->,font=\scriptsize,>=angle 60]
(A) edge node[above left]{$f$} (B)
(A) edge node[above]{$e$} (C);
\path[->,font=\scriptsize,>=angle 45]
(C) [bend left] edge node[below left]{$\color{red} {C_e^f}$}  (B);
\end{tikzpicture}
\end{center}
Здесь $e$ и $f$ обозначают координатные отображения  для соответствующих базисов. Так как $u$ обратимо, то можно определить отображение $C_{e}^f=f\circ e^{-1}$. Матрица соответствующая $C_{e}^f$ называется матрицей замены из базиса $e$ в базис $f$.
\edfn


Это отображение устроено следующим образом: оно берёт столбец $x$ и сопоставляет ему вектор $v=\sum x_i e_i$, у которого ровно такие координаты, а затем считает его координаты в новом базисе. То есть матрица $C$ -- это способ пересчитать координаты вектора из старого базиса в новый. Точнее

\rm Если есть вектор $v$ и его координаты в базисе $e_1,\dots,e_n$ это столбец $x$, а координаты в базисе $f_1,\dots,f_n$ -- это столбец $y$, то верно соотношение
$$y=Cx.$$
\erm

Как найти матрицу $C$? Для этого рассмотрим вектор $i$-ый вектор из стандартного базиса $K^n$ и посмотрим, что с ним происходит. Он переходит в вектор $e_i$, а теперь нужно разложить $$e_i= \sum_{j=1}^n C_{ji} f_j.$$
Тогда коэффициенты $C_{ji}$ и будут составлять $i$-ый столбец матрицы $C$, что и объясняет выбранную для них индексацию.
Это соотношение удобно записать в такой форме
$$(e_1,\dots,e_n)=(f_1,\dots,f_n)\cdot C$$
или
$$\pmat e_1 \\ \vdots\\ e_n \epmat =C^{\top} \pmat f_1 \\ \vdots \\ f_n \epmat $$

\dfn Пусть $C\in M_{m \times n}(K)$. Тогда транспонированной к $C$ называется матрица $C^{\top} \in M_{n \times m}(K)$ с элементами $C^{\top}_{ij}=C_{ji}$.
\edfn


\rm Имеют место следующие соотношения $C_f^e= (C_e^f)^{-1}$ и $C_f^gC_e^f= C_e^g$. Действительно $C_f^e=e\circ f^{-1}= (f\circ e^{-1})^{-1}= (C_f^e)^{-1}$. Далее $C_f^gC_e^f= g \circ f^{-1} \circ f \circ e^{-1}= g \circ e^{-1}= C_e^g$.
В частности, матрица замены базиса всегда обратима. Любая обратимая матрица есть матрица замены из заданного базиса в какой-то.
\erm



Теперь можно разобраться с ситуацией про замену базиса. Пусть даны пространства $V$, $W$ и линейное отображение $L\colon V\to W$. Рассмотрим в пространстве $V$ два базиса $e_1,\dots, e_n$ (старый) и $e_1',\dots, e_n'$ (новый). Аналогично в $W$ --- $f_1,\dots, f_m$ (старый) и $f_1',\dots, f_m'$ (новый). Нарисуем диаграмму


\begin{center}
\begin{tikzpicture}
\node (A) at (1, 1) {$V$};
\node (B) at (0, 0) {$K^n$};
\node (C) at (3, 1) {$W$};
\node (D) at (4, 0) {$K^m$};
\node (E) at (-1, 1) {$K^n$};
\node (F) at (5, 1) {$K^m$};
\path[->,font=\scriptsize,>=angle 60]
(A) edge node[above]{$\color{red} L$} (C)
(A) edge node[above left]{ $e'$} (B)
(C) edge node[above right]{$f'$} (D)
(A) edge node[above]{ $e$} (E)
(C) edge node[above]{ $f$} (F);
\path[->,font=\scriptsize,>=angle 45]
(B) [bend left] edge node[below]{$\color{red} {A'}$}  (D)
(E) [bend left] edge node[above]{$\color{red}{A}$ } (F)
(E) [bend left] edge node[below left]{$\color{red}{C}$ } (B)
(F) [bend right] edge node[below right]{$\color{red}{D}$}  (D);
\end{tikzpicture}
\end{center}

\thrm
В указанных предположениях матрицу $A'$ линейного отображения $L$ в базисах $e'$ и $f'$ можно выразить через матрицу $A$ и матрицы замены $C$ и $D$ следующим образом
$$A'=DAC^{-1}.$$
\ethrm
\proof $$A'=f'\circ L\circ {e'}^{-1}= f'\circ f^{-1} \circ f\circ L\circ e^{-1}\circ e\circ{e'}^{-1}= D A C^{-1}$$
\endproof




\section{Свойства ранга}

Наша ближайшая задача -- классифицировать все линейные отображения. Для этого мы посмотрим на их численную характреистику -- ранг. Ранг линейного отображения отвечает за его обратимость и за размерность множества решений уравнения $Lx=y$. Исследуем свойства ранга.

\lm Пусть $S \colon V \to W$, $T \colon W \to U$, тогда $$\rank T \circ S \leq \min(\rank T, \rank S).$$
\proof Для начала покажем, что если есть $L\colon V_1 \to V_2$ и $V' \leq V_1$, то $\dim L(V') \leq \dim V'$. Действительно $$\dim V'= \dim \im L|_{V'} + \dim \Ker L|_{V'} \geq \dim L(V').$$
Теперь $\im T \circ S \subseteq  \im T$, поэтому $\rank T \circ S \leq \rank T$.
Далее $\rank T \circ S= \dim \im T \circ S = \dim T (S(V)) \leq \dim S(V)= \rank S.$ Что и требовалось.
\endproof
\elm

\lm Пусть $S, T \colon V \to W$ -- два линейных отображения. Тогда $$\rank (T + S) \leq \rank T + \rank S.$$
\elm
\proof Для начала покажем, что, если $V_1,V_2 \leq V$, то $\dim V_1 +V_2 \leq \dim V_1 + \dim V_2$. Действительно, если $e_1,\dots, e_k$ базис  $V_1$, и $f_1,\dots, f_l$ базис $V_2$, то $e_1,\dots, e_k, f_1, \dots, f_l$ порождает $V_1+V_2$. Осталось заметить, что размер базиса меньше размера любой порождающей системы.
Теперь заметим, что $\im (S+T) \leq \im S + \im T$. Тогда $\rank (S+T) \leq \dim (\im S + \im T) \leq \rank S + \rank T$.
\endproof

\lm[Ранг не меняется при подкрутке на изоморфизмы] Пусть $L \colon U_1 \to V_1$ -- линейное отображение, а $G \colon U_1 \to U_2 $ и $H \colon V_1 \to V_2$ изоморфизмы. Тогда $\rank (H \circ L \circ G^{-1}) = \rank L$.
\elm
\proof Для начала покажем, что, если некоторое линейное отображение $T\colon V \to W$ обратимо, то  $\dim T(U)= \dim U$ для любого $U \leq V$. Действительно, если $T$ обратимо, то оно инъективно и его сужение на $U$ тоже инъективно. Тогда $\dim U = \dim T(U) + \dim \Ker T|_U= \dim T(U)$.
Заметим, что если $G$ обратимо, то  $G^{-1}$ тоже и, следовательно, сюръективно. Тогда $G^{-1}(X)=U$.  Теперь $\rank L= \dim L(U)= \dim H(L(U))= \dim H(L(G^{-1}(X))= \rank H \circ L \circ G^{-1}$.
\endproof

\thrm Пусть $U_1, U_2$ и $V_1,V_2$ векторные пространства над полем $K$, причём  $\dim U_1=\dim U_2$ и $\dim V_1 =\dim V_2$. Пусть $L \colon U_1\to V_1$ и $T \colon U_2 \to V_2$ -- линейные отображения. Тогда условие $\rank L=\rank T$ равносильно существованию изоморфизмов
$$G \colon U_1 \to U_2 \text{ и }  H \colon V_1 \to V_2, \text{ что }  H \circ L \circ G^{-1}=T$$
\ethrm
\proof Для того, чтобы построить $H$ и $G$ нужно выбрать подходящие базисы пространств. Рассмотрим базис $e_1,\dots, e_k$ пространства $\Ker T$ и дополним его элементами $e_{k+1},\dots, e_n$ до базиса $U_1$. Так как $\dim \Ker L = \dim \Ker T$ (благодаря равенству рангов), то аналогично можно построить базис $f_1,\dots, f_k$ ядра $\Ker T$ и дополнить его $f_{k+1},\dots, f_n$ до базиса $U_2$.

Построим линейное отображение $G\colon U_1 \to U_2$ по правилу $G(e_i)=f_i$. Это обратимое линейное отображение.

Теперь заметим, что образы $L(e_{k+1}), \dots, L(e_n)$ являются базисом $\im L$ и, аналогично, $T(f_{k+1}), \dots, T(f_n)$  являются базисом $\im T$. Тогда дополним их элементами $g_1,\dots,g_l$ и $ g_1',\dots, g_l'$ до базисов $V_1$ и $V_2$. Тогда  зададим $H(L(e_i))=T(f_i)$ и $H(g_i)=g_i'$.

Проверим теперь, что $H \circ L \circ G^{-1}=T$. Рассмотрим базисный элемент $f_i$. Тогда $G^{-1}(f_i)=e_i$. Далее, если $i<k$, то  $H(L(G^{-1}(f_i)))= H(L(e_i))=0=T(f_i)$. В оставшемся случае $H(L(G^{-1}(f_i)))= H(L(e_i))=T(f_i)$ по выбору $H$.
\endproof


Напомню, что ранг матрицы -- это ранг соответствующего линейного отображения. Иными словами, это размерность пространства, порождённого столбцами матрицы.

\crl Для любой матрицы $A \in M_{m \times n}$ ранга $r$ существуют обратимые матрицы $C$ и $D$, что $CAD^{-1}$ имеет вид
$$\pmat E_r& 0\\
0&0\\
\epmat.$$
\ecrl


Однако, в методе Гаусса удобно смотреть на строчки, чем на столбцы. Это замечание приводит нас к определению строчного ранга.

\dfn Пусть $A\in M_{m\times n}(K)$. Определим $\rank_{row}$ -- cтрочный ранг матрицы $A$ как размерность пространства, порождённого её строками.
\edfn

\rm Очевидно равенство $\rank_{row} A= \rank A^{\top}$.
\erm

Поэтому исследуем свойства операции транспонирования матрицы.

\lm Пусть $A,B$ -- матрицы. Тогда\\
0) ${A^{\top}}^{\top}=A$.\\
1) $(A+\lambda B)^{\top}= A^{\top}+\lambda B^{\top}$.\\
2) $(AB)^{\top}=B^{\top}A^{\top}$.\\
3) $(A^{\top})^{-1}= (A^{-1})^{\top}$.
\elm
\proof Покажем второй пункт. Имеем
$$(AB)^{\top}_{ij}= (AB)_{ji}=\sum_k A_{jk}B_{ki}=\sum_k B^{\top}_{ik}A^{\top}_{kj}=(B^{\top}A^{\top})_{ij}.$$
Теперь покажем пункт 3. Пусть $AA^{-1}=E_n$. Тогда $E_n=E_n^{\top}=(AA^{-1})^{\top}= (A^{-1})^{\top}A^{\top}$. Аналогично из равенства $A^{-1}A=E_n$ следует $E_n=A^{\top}(A^{-1})^{\top}$, что завершает доказательство.
\endproof

\rm Заметим, что, вообще говоря, проверять второе условие не обязательно. Действительно, предположим, что $BA=E_n$, где $A,B$ квадратные матрицы размера $n$. Покажем, что тогда $B=A^{-1}$. Заметим, что матрица $A$ имеет нулевое ядро так как в противном случае матрица $E_n$ имела бы нетривиальное ядро. Тогда матрица $A$ обратима по принципу Дирихле, потому что обратимо соответствующее линейное отображение. Тогда $B=BAA^{-1}=E_nA^{-1}=A^{-1}$.
\erm

\thrm Пусть $A\in M_{m \times n}(K)$. Тогда ранг по строчкам и ранг по столбцам совпадают.
\ethrm
\proof Представим матрицу $A$ в виде произведения
$$A=C \pmat E_r & 0\\
0 & 0 \epmat D,$$
где $C \in \GL_m(K)$, а $D \in \GL_n(K)$, а $r$ -- ранг матрицы. Тогда
$$A^{\top}=D^{\top} \pmat E_r & 0\\
0 & 0 \epmat C^{\top}.$$
Так как ранг не меняется при домножении на обратимые матрицы, то $\rank A^{\top} = \rank \pmat E_r & 0\\
0 & 0 \epmat = r = \rank A$.
\endproof

Есть и другое определение -- минорный ранг.

\dfn Пусть $A \in M_{m\times n}(K)$ и  даны два множества индексов $I=\{i_1< \dots< i_k\}\subseteq \ovl{1,m}$, $J=\{j_1< \dots < j_l\}\subseteq \ovl{1,n}$. Подматрицей $A_{I,J} \in M_{k\times l}$ определим так, чтобы ${A_{I,J}}_{s,t}=a_{i_s,j_t}$.
\edfn

\dfn[Минорный ранг] Пусть $A \in M_{m\times n}(K)$. Минорный ранг матрицы $A$ -- это наибольшее такое $k$, что существует невырожденная квадратная подматрица $A_{I,J}$ размера $k$ внутри $A$.
\edfn

\thrm Ранг матрицы равен её минорному рангу.
\ethrm
\proof Пусть $A$ -- матрица ранга $r$. Тогда в матрице $A$ есть $r$ линейно независимых столбцов $v_1,\dots,v_r$. Составим из них матрицу $A'$. Ранг $A'$ равен $r$. Теперь, так как строчный и столбцовый ранги совпадают, то в $A'$ есть $r$ линейно независимых строк. Они дают крадратную невырожденную подматрицу в $A'$ и следовательно в $A$.
\endproof


\begin{comment}
\rm Пусть есть матрица составленная из столбцов $v_1,\dots,v_n$. Что значит, что её минорный ранг равен $k$? По определению это значит, что можно выбрать $k$ столбцов и $k$ стандартных базисных векторов $e_{i_1},\dots,e_{i_k}$ так, чтобы проекция на подпространство $U=\lan e_{i_1},\dots,e_{i_k}\ran$ дала базис этого пространства. Таким образом любое подпространство размерности $k$ изоморфно проецируется на некоторое координатное подпространство размерности $k$.
\erm
\end{comment}

Сейчас перед нами стоит вопрос, как вычислить ранг матрицы. При решении системы линейных уравнений нам пригодились элементарные преобразования. Покажем, что они не влияют на ранг матрицы.

Для этого покажем, что элементарное преобразование можно реализовать домножив исходную матрицу на некоторую обратимую матрицу. Точнее, рассмотрим матрицу
$$ E_{ij}(\lambda)=E_n +\lambda e_{ij}=
\bordermatrix{
 & &j&& \cr
 &1&\cdots&\cdots&0\cr
 &\vdots&\ddots && \vdots\cr
i&\vdots& \lambda & \ddots& \vdots\cr
 &0&\cdots& \cdots&1
}.
$$\\
Домножение на матрицу $E_{ij}(\lambda)$ приводит к тому, что матрица
$$A'=E_{ij}(\lambda)A.$$
получается из матрицы $A$ прибавлением $j$-ой строки к $i$-ой с коэффициентом $\lambda$.
Действительно, домножение матрицы $A$ на матрицу $e_{ij}$ выделяет из матрицы $A$ $j$-ую строку и записывает её на позицию $i$.

\dfn Матрица $E_{ij}(\lambda)$ называется матрицей элементарного преобразований первого типа.
\edfn



Что соответствует замене местами двух строк матрицы $A$? Для этого надо поставить $i$-ую строчку в позицию $j$ и наоборот, а остальные строки оставить на месте. Итого имеем матрицу
$$T_{(ij)}=\,\bordermatrix{
 & & &j& & i& \cr
 &1& & & & & &0\cr
 & &\ddots & & & & \cr
j\,& && 0 & & 1 & &\cr
\,& &&  & &  & &\cr
i\,& && 1 & &0 & &\cr
 & & & & & & \ddots& \cr
 &0& & & & & &1
}=E_n- e_{ii}-e_{jj}+e_{ij}+e_{ji}.
$$


\dfn Матрица $T_{(ij)}$ называется матрицей элементарного преобразований второго типа или матрицей транспозиции.
\edfn

А что вообще происходит при перестановке строк? Каждой перестановке $\sigma \in S_n$ однозначно соответствует матрица $T_{\sigma}$ заданная правилом
$$(T_{\sigma})_{ij}= \begin{cases}
1, \text{ если $\sigma(j)=i$}\\
0, \text{ иначе }
\end{cases}.$$


\dfn Матрица $T_{\sigma}$ называется матрицей перестановки $\sigma$.
\edfn


\rm Заметим, что произведение $T_{\sigma} T_{\tau}=T_{\sigma\tau}$. Это даёт гомоморфизм из группы перестановок в группу $\GL_n(K)$. Любая матрица, у которой в каждом столбце ровно один единичный элемент, а все остальные 0 является матрицей перестановки. Можно написать аналогичное условие для строк.
\erm

Наконец, преобразованию третьего типа соответствует матрица
$$ D_{i}(\lambda)=
\bordermatrix{
 & &&i&& \cr
 &1&&&&0\cr
 &&\ddots &&& \cr
i&&  & \lambda && \cr
&&  &  &\ddots& \cr
 &0&& &&1
}.
$$

\dfn Матрица $D_{i}(\lambda)$, $\lambda \in K^*$ называется матрицей элементарного преобразований третьего типа.
\edfn

\rm Заметим, что $E_{ij}^{-1}(\lambda)= E_{ij}(-\lambda)$ и $D_i(\lambda)= D_i(\lambda^{-1})$, а $T_{ij}^{-1}=T_{ij}$.
\erm

Прежде чем пойти дальше сформулируем давно напрашивающиеся определения про преобразования столбцов, которые нам пригодятся в дальнейшем.

\dfn Определим элементарные преобразования столбцов матрицы $A\in M_{n \times m}(K)$ следующим образом:\\
1) Преобразование первого типа соответствует домножению
$$A \to A E_{ij}(\lambda),$$
которое приводит к тому, что $i$-ый столбец прибавляется к $j$-ому с коэффициентом $\lambda$.\\
2) Преобразование столбцов второго типа -- это перестановка $i$-го  и $j$-го столбца местами, что соответствует домножению на $T_{(ij)}$
$$A \to A T_{(ij)}.$$
3) Преобразование третьего типа -- это домножение на $D_i(\lambda)$ -- умножение $i$-го столбца на $\lambda \in K^*$.
\edfn

\thrm
Элементарные преобразования строк и столбцов не меняют ранг матрицы.

\ethrm
\proof Преобразования строк и столбцов соответствуют домножениям на обратимые матрицы и поэтому не меняют ранг.
\endproof

Приведём пример использования свойства ранга. Для этого посмотрим на пример с поисковой системой. Пусть $G$ граф на $n$ вершинах. Тогда рассмотрим квадратную матрицу $P(G)$ размера $n$
$$P(G)_{ij}= \begin{cases} \frac{1}{d_j^{out}}, \text{ если есть ребро $j \to i$} \\
1, \text{ если $i=j$ и из вершины $i$ нет исходящих рёбер}\\
0, \text{ иначе }
\end{cases}.$$
Эта матрицу еще называется матрицей случайного блуждания на графе $G$. Смысл состоит в том, что она моделирует следующую ситуацию -- если мы с вероятностями $w_1,\dots,w_n$ находимся в вершинах графа и на следующем шаге хотим из $j$-ой вершины перейти в $i$-ую по ребру графа, при этом проход по каждому ребру равновероятен, то после такого действия новый набор вероятностей будет иметь вид $P(G)w$. Матрица $P$ обладает тем свойством, что сумма элементов в каждом столбце равна 1.

Теперь систему для нахождения <<важностей>> вершин графа можно переписать в виде $E_n w= P(G)w$ или
$$(E_n - P(G))w=0.$$
Нам хочется показать, что у этой системы есть нетривиальное решение, то есть, что ранг матрицы $E_n- P(G)$ строго меньше $n$. Для этого можно показать, что ранг
$$E_n - P(G)^{\top}$$
меньше $n$, а это легко сделать -- достаточно заметить, что матрица $E_n - P(G)^{\top}$  имеет столбец $(1,1,\dots, 1)^{\top}$ в своём ядре. Заметьте -- это не помогает найти решение исходной системы, а лишь доказывает его существование.






\section{Прямая сумма и фактор}

\dfn[Прямая сумма и прямое произведение] Рассмотрим набор векторных пространств $U_{\alpha}$ над полем $K$,  где элементы $\alpha$ пробегают множество индексов $I$. Определим прямое произведение пространств из семейства $U_{\alpha}$, как множество
$$\prod_{\alpha \in I} U_{\alpha}$$
вместе с операциями $$(u_{\alpha})_{\alpha\in I}+ (v_{\alpha})_{\alpha\in I}=(u_{\alpha}+v_{\alpha})_{\alpha\in I},\,\,\, \lambda \cdot (u_{\alpha})_{\alpha\in I}= (\lambda \cdot u_{\alpha})_{\alpha\in I}.$$
То есть операции покомпонентные. Определим прямую сумму набора $U_{\alpha}$ как подпространство в $\prod_{\alpha \in I} U_{\alpha}$, заданное следующими условиями
$$\bigoplus_{\alpha \in I } U_{\alpha}=\left\{ (u_{\alpha})_{\alpha \in I} \,|\, \text{ все, кроме конечного числа компонент $u_{\alpha}$ равны 0}\right\}.$$
\edfn

\rm Прямая сумма конечного числа пространств совпадает с прямым произведением. Обычно мы будем использовать значок прямой суммы $U_1\oplus \dots \oplus U_n$.
\erm


\rm $\dim U_1\oplus \dots \oplus U_n= \sum \dim U_i$.
\erm

Применим это нехитрое соображение.


\lm[Формула Грассмана] Пусть $U$ и $V$ подпространства некоторого $W$ тогда имеет место соотношение
$$\dim U +\dim V = \dim U\cap V + \dim (U+V).$$

\proof Рассмотрим последовательность отображений
$$U\cap V \stackrel{v\to (v,-v)}{\longrightarrow} U\oplus V \stackrel{(u,v)\to u+v}{\longrightarrow} W$$
Заметим, что образ отображения $L$ из $U\oplus V \to W$ это в точности $U+V$. Когда пара элементов $(u,v)$ лежит в ядре $L$? Когда $u=-v$. То есть лежит в образе $U \cap V$. Так как отображение из $U\cap V$ инъективно, то $U\cap V \cong \Ker L$. Тогда $\dim U\cap V = \dim \Ker L $. Осталось применить соотношение про связь размерности ядра и образа
$$\dim U+ \dim V = \dim U\oplus V= \dim(U+V)+\dim U\cap V.$$
\endproof
\elm

\dfn Пусть $W$ -- векторное пространство, а $U_1,\dots,U_n$ -- его подпространства. Будем говорить, что пространство $W$ раскладывается в прямую сумму $U_i$, если отображение $U_1\oplus \dots \oplus U_n \to W$ заданное по правилу $(u_1,\dots,u_n) \to u_1 +\dots+u_n$ является изоморфизмом.
\edfn

\thrm
Пусть $W$ -- векторное пространство, а $U_i$ его подпространства. Тогда следующие условия эквивалентны\\
1) $W$ раскладывается в прямую сумму $U_i$ \\
2) $\forall w \in W$, существуют единственные $u_i \in U_i$  $w=u_1+\dots+u_n$\\
3) $U_i\cap \sum_{j\neq i} U_j= \{0\}$ и $\dim U_1 +\dots+ \dim U_n =\dim W$\\
\ethrm

Сделаем ещё несколько уточнений для случая двух подпространств:

\lm Пространство $W$ раскладывается в прямую сумму $U$   и $V$ тогда и только тогда, когда существует оператор $L \colon W \to W$, что $L^2=L$ и $U=\im L$, а $V=\Ker L$. Более того, такой оператор единственен.
\elm

\dfn Оператор, удовлетворяющий тождеству $L^2=L$ называется проектором.
 Любой проектор однозначно определяет пару $U$ и $V$. Проектор, соответствующий такой паре подпространств, называется проекцией на $U$ вдоль $V$.
\edfn

\rm Для любого подпространства $U \leq W$ существует пространство $V$, что $W=U\oplus V$.

  Пусть $U,V\leq W$ подпространства и верно, что $W=U\oplus V$. Рассмотрим каноническую проекцию $\pi \colon W \to W/U$. Тогда $\pi|_{V} \colon V \to W/U$ является изоморфизмом.
\erm

Последним замечанием станет следующие: пусть $L\colon V \to W$ -- линейное отображение, а пространства $V$ и $W$ разложены в прямую сумму $V=U_1\oplus U_2 $ и $W=U_1'\oplus U_2'$. Рассмотрим базисы  $U_1, U_2$ и $U_1', U_2'$. Тогда матрица $L$ в этих базисах имеет вид
$$ \pmat A& B \\
C & D
\epmat,
$$
где матрица $A$ -- это матрица $pr_{U_1'}\circ L|_{U_1}$, $B=pr_{U_1'}\circ L |_{U_2}$ и т.д. Такую запись матрицы $L$ будем называть блочной. Если $C=0$, то матрица называется блочной верхнетреугольной. Если и $C=0$ и $B=0$, то матрица называется блочно диагональной.


\dfn[Пространство натянутое на множество]
Пусть $X$ множество. Рассмотрим множество функций $\lan X \ran = \{ f\colon X \to K\,|\, \text{ $f=0$ всюду, кроме конечного числа точек} \}$.
\edfn

Это множество, очевидно, является подпространством в пространстве всех функций. Более того, с каждой точкой $x\in X$ можно связать функцию $$\delta_x(y)=\begin{cases} 1, \text{ если  $x=y$} \\
0, \text{ иначе }
\end{cases}$$
Набор таких функций образует базис $\lan X \ran$. Вместо сложного обозначения $\delta_x$, если не возникает противоречий, можно просто говорить про точку $x$. Тогда все элементы $\lan X \ran $ можно записать в виде $\lambda_1 x_1+\dots +\lambda_n x_n$. Про такое выражение говорят, что это формальная линейная комбинация элементов из $X$. Такой подход позволяет посмотреть на некоторые задачи, как на задачи из линейной алгебры, что в некоторых ситуациях приносит пользу, так как для задачи из линейной алгебры обычно имеют решение. Вот например:

Рассмотрим граф $G$ с множеством вершин $V$ и множеством рёбер $E$.

Пусть на множестве рёбер задана ориентация. Определим отображение $\lan E \ran \to \lan V \ran$ заданное на базисе следующим образом:

$$\text{ ребро $e$ идущее из вершины $u$ в вершину $v$ переходит в $v-u$}.$$

Матрица этого отображения имеет специальное название.

\dfn Определим матрицу инцидентности $B(G)$ размера $n\times m$, $n$ -- число вершин, а $m$ число рёбер $G$, так, что
$$b_{ij}=\begin{cases} 1, \text{ если ребро $j$ входит в вершину $i$}\\
-1, \text{ если ребро $j$ исходит из вершины $i$}\\
0, \text{ иначе }
\end{cases}.$$
\edfn

Рассмотрим для простоты $K=\mb Z/2$. Тогда ориентация не нужна, так как никакого знака нет. Более того -- теперь любому элементу из $\lan E \ran $ однозначно соответствует подграф -- если коэффициент 1, то ребро есть, если 0, то нет. Посчитаем теперь некоторую характеристику матрицы $B$

\lm Размерность ядра матрицы инцидентности  равна $m-n+c$, где $m$ -- количество рёбер, $n$ -- количество вершин, $c$ -- количество компонент связности графа.
\elm
\proof Посчитаем коразмерность образа матрицы $B$ -- для этого мы придумаем уравнения на образ. Рассмотрим множество компонент связности  $C = \{ K_1,\dots,K_c\}$ графа $G$.

Построим отображение $\ffi \colon \lan V \ran \to \lan C \ran$ заданное на базисе по правилу $v \to K_i$ , если $v\in K_i$. Это отображение сюръективно. Я утверждаю, что $\Ker \ffi$ -- это в точности $\Im B$. % Странно, ffi же просто отображение из множества в множество. Ясно, что имелось в виду, но надо написать получше. К.М.: прошу прощения, но слово "надо" здесь неуместно!

Формальная сумма $\sum a_v v$ лежит в ядре $\ffi$ тогда и только тогда, когда для любой компоненты связности $\sum_{v\in K_i} a_v=0$. Очевидно, образ $B$ удовлетворяет этим  уравнениям (достаточно посмотреть на образ ребра).

Покажем, что любой элемент $x= \sum_{v \in V} a_v v \in \lan V \ran$ с таким свойством лежит в $\Im B$. Сделаем это индукцией по числу вершин с ненулевым $a_v$. Действительно, в компоненте связности $K_i$ не может быть одной вершины $v$, что $a_v\neq 0$ так как сумма весов в её компоненте связности была бы не нулевая.
Пусть есть хотя бы две вершины $u$ и $w$ в одной компоненте связности с $a_{u},a_{w}\neq 0$. Проведём путь $e_1\dots e_k$ из вершины $u \to w$. Тогда посмотрим на сумму $a_{u}(\sum e_i)$. Имеем $$B(a_{u}(\sum e_i))= a_{u}w - a_u u.$$
Заметим, что $a_{u}w - a_uu +x$ имеет по крайней мере на одну нулевую координату больше (обнулился коэффициент в вершине $u$ и, возможно, в $w$, остальные не поменялись) и, следовательно, по индукционному предположению лежит в $\Im B$. Но тогда и $x \in \Im B$ так как от этой суммы он отличается на $B(a_{u}(\sum e_i))$.

Теперь $\dim \Ker B = m - \dim \Im B= m - (n-c)= m-n +c$.
\endproof

\dfn Граф называется эйлеровым, если множество его рёбер можно представить в виде дизъюнктного объединения циклов. Это эквивалентно тому, что степень каждой вершины чётна.
\edfn

\crl Граф $G$ содержит $2^{m-n+c}$ эйлеровых подграфов (считая пустой).
\proof Элемент $x \in \lan E \ran $ лежит в $\Ker B$ (над $\mb Z/2$) тогда и только тогда, когда сумма в каждой вершине равна нулю, то есть когда в соответствующем подграфе степень любой вершины чётна.
\endproof
\ecrl


\section{Определитель}

Есть ли какая-нибудь численная характеристика, которая позволяет сказать, что матрица $A\in M_n(K)$ обратима? Напомню, что обратимость матрицы $A$ равносильна линейной независимости её столбцов. Если посмотреть на картинку над $\mb R$, то видно, что вектора линейно независимы тогда и только тогда, когда объём параллелепипеда на них натянутого отличен от нуля.

\dfn Пусть $V$ -- векторное пространство размерности $n$ над $\mb R$, тогда для набора  $v_1,\dots,v_n \in V$ определим параллелепипед
$$D(v_1,\dots,v_n)=\left\{\sum_{i=1}^n \lambda_i v_i\,|\, \text{ где } \lambda_i\in [0,1]\right\}.$$
\edfn


Обозначим для краткости $\Vol(v_1,\dots,v_n)= \Vol (D(v_1,\dots,v_n))$. Более того, работая с набором векторов из пространство $K^n$ я буду отождествлять его с матрицей $M_n(K)$. 

Попробуем понять есть ли возможность как-то аксиоматизировать понятие объёма параллелепипеда, так, чтобы оно работало над произвольным полем и для произвольного пространства. А заодно узнаем как его посчитать.

Отображение $\Vol \colon M_n(K) \to K$ типа объёма естественно удовлетворяет следующим свойствам...\\
0) Объём единичного кубика это $\Vol(E)=1$.\\
1) При растяжении одного вектора объём меняется пропорционально $\Vol(\dots,\lambda v,\dots)=|\lambda|\Vol(\dots,v,\dots)$.\\
2) Исходя из принципа Кавальери $\Vol(\dots,v,\dots,u,\dots)=\Vol(\dots,v,\dots,u+\lambda v,\dots)$.\\
3) Если в наборе есть два одинаковых вектора, то $\Vol(\dots,v, \dots, v,\dots)=0$.\\
Для общего пространства $V$ условие типа 0) не имеет смысла, так как нет возможности выбрать какой-то канонический базис в $V$.

Безусловно, модуль числа в свойстве 1) нет возможности определить над произвольным полем. Да и вообще ни о какой положительности, как для объёма, речи нет. Вместо свойства 1) хочется потребовать\\
1') $\omega(\dots,\lambda v,\dots)=\lambda \omega(\dots,v,\dots)$.\\
Заметим, что если мы находимся над $\mb R$ и  для отображения $\omega$ выполнены свойства 0), 1'), 2), то для отображения $|\omega|$ выполнены свойства 0),1),2). 

Наконец, с алгебраической точки зрения свойство 2) не самое удобное. Вместо него мы рассмотрим свойство\\
2') $\omega(\dots,u+v,\dots)=\omega(\dots,u,\dots)+\omega(\dots,v,\dots)$.


\begin{defn}
Отображение $\omega \colon  V^l\to K $ называется полилинейным, если для любого $1\leq i\leq l$
$$\omega(v_1,\dots,v_i+\lambda u_i,\dots, v_l)= \omega(v_1,\dots,v_i,\dots, v_l)+\lambda\omega(v_1,\dots,u_i,\dots, v_l).$$
Здесь одновременно зашифрованы свойства типа 1')  и  2').
\end{defn}

\lm Для полилинейного отображения $\omega \colon V^l \to K $ выполнено, что, если $e_1,\dots,e_n$ базис $V$, то
$$\omega(v_1,\dots,v_l)=\sum_{i_1,\dots,i_n} \omega(e_{i_1},\dots,e_{i_l}) \prod_{j=1}^l a_{i_j,j}$$
где $a_{i,j}$ -- это $i$-ая координата $v_j$.\\
\elm

\begin{defn}
Полилинейное отображение $\mu \colon V^l\to K $ называется антисимметричным или кососимметричным, если
$$\omega(v_1,\dots,v,\dots,v,\dots, v_l)=0.$$ и симметричным, если
$$\omega(v_1,\dots,v_i,\dots,v_j,\dots, v_l)=\omega(v_1,\dots,v_j,\dots,v_i,\dots, v_l)$$
\end{defn}



Теперь надо разобраться с естественностью рассматриваемых понятий и как-то отметить их свойства.
\lm Пусть $V$ -- пространство размерности $n$. Для кососимметрического полилинейного отображения $\mu \colon V^l \to K $ выполнено, что \\
1) $\omega(\dots,u,\dots,v,\dots)= -\omega(\dots,v,\dots,u,\dots)$.\\
2) В случае, если $\chr K \neq 2$, то из  свойства, описанного в пункте 1), следует кососимметричность.\\
3) Для любой перестановки $\sigma \in S_l$ выполнено, что $\omega(v_{\sigma(1)},\dots,v_{\sigma(l)})= \sgn(\sigma) \omega(v_1,\dots,v_n)$.\\
4)  $\omega(\dots,v,\dots,u,\dots)=\omega(\dots,v,\dots,u+\lambda v,\dots)$.\\
5) Пусть $l=n$. Для набора векторов $v_1,\dots,v_n$ и базиса $e_1,\dots,e_n$ выполнено, что
$$\mu(v_1,\dots,v_n)=\omega(e_1,\dots,e_n)\sum_{\sigma \in S_n} \sgn(\sigma)\prod_{j=1}^n a_{\sigma(j),j}=\omega(e_1,\dots,e_n)\sum_{\sigma \in S_n} \sgn(\sigma)\prod_{i=1}^n a_{i,\sigma(i)},$$
где $a_{i,j}$ -- это $i$-ая координата $j$-ого вектора.
\proof Докажем первое утверждение. Распишем 
\begin{align*} 
0&=\omega(\dots,u+v,\dots,u+v,\dots)=\\&=\omega(\dots,u,\dots,u,\dots)+\omega(\dots,v,\dots,u,\dots)+\omega(\dots,u,\dots,v,\dots)+\omega(\dots,v,\dots,v,\dots)=\\ &=\omega(\dots,v,\dots,u,\dots)+\omega(\dots,u,\dots,v,\dots),
\end{align*}
что и доказывает утверждение.\\
Теперь, если $\chr K\neq 2$, то переставляя одинаковые столбцы получаем $$\omega(\dots,v,\dots,v,\dots)=-\omega(\dots,v,\dots,v,\dots).$$
То есть $2\omega(\dots,v,\dots,v,\dots)=0$. Осталось поделить на 2.\\
По определению знака перестановки $$\sgn(\sigma)=(-1)^{k}, \text{ где $k$ -- число транспозиций в разложении $\sigma$}$$
откуда получаем, что применить $\sigma$ это тоже самое, что применить $k$ транспозиций, то есть изменить знак $k$ раз, что и требовалось.\\
Докажем 4). $$\omega(\dots,v,\dots,u+\lambda v,\dots)=\omega(\dots,v,\dots,u,\dots)+\lambda\omega(\dots,v,\dots, v,\dots)=\omega(\dots,v,\dots,u,\dots).$$
Для доказательства свойства 5) воспользуемся общим выражением для полилинейной формы.
$$\omega(v_1,\dots,v_n)=\sum_{i_1,\dots,i_n} \omega(e_{i_1},\dots,e_{i_n}) \prod_{j=1}^n a_{i_j,j}.$$
Если два индекса совпадают, то $\omega(e_{i_1},\dots,e_{i_n})=0$, а вместе с ним и всё слагаемое. Остаются только наборы с разными $i_{\alpha}$ которые однозначно соответствуют перестановкам $\sigma(k)=i_k$. Теперь заметим, что $\omega(e_{\sigma(1)},\dots,e_{\sigma(n)})=\sgn(\sigma)\omega(e_1,\dots,e_n)$, что доказывает первое равенство. \\
Теперь $$\prod_{j=1}^n a_{\sigma(j),j}=\prod_{i=1}^n a_{i,\sigma^{-1}(i)}.$$ Если вместо $\sigma$ поставить сумму по $\sigma^{-1}$, то с одной стороны сумма не поменяется, а с другой стороны по тождеству выше, можно будет перекинуть $\sigma$ на другой индекс.
\endproof
\elm







\begin{defn}
Пусть $n=\dim V$. Антисимметричная полилинейная форма $\omega \colon V^n \to K $ называется формой объёма на $V$. Если такая форма не равна 0, то будем говорить, что она невырождена.
\end{defn}



До этого момента мы говорили про объекты которых, возможно, просто не существует. Настала пора предъявить для них конструкцию и доказать её единственность.

\dfn  Определителем $\det$ называется отображение $\det \colon M_n(K) \to K$, такое, что $$\det(A)=\sum_{\sigma \in S_n} \sgn(\sigma)\prod_{1\leq i\leq n} a_{i\sigma(i)}.$$
\edfn

\dfn Пусть $e_1,\dots,e_n$ базис пространства $V$. Определим отображение $\Vol_e \colon V^n \to K$, такое, что
$$\Vol_e(v_1,\dots,v_n)=\det(e(v_1),\dots, e(v_n)),$$
где $e\colon V \to K^n$ -- отображение сопоставления координат.
\edfn

\begin{thm} Следующие свойства верны:\\
1) Определитель является формой объёма на $K^n$, такой, что $\det E=1$.\\
2) Если $V$ -- пространство размерности $n$, то любая форма объёма на $V$ имеет вид $$\omega=\omega(e_1,\dots,e_n)\Vol_e.$$
В частности, если есть два базиса $e$ и $f$, то $\Vol_{f}=\det(C^f_{e}) \Vol_{e}$.\\
3) Пространство форм объёма одномерно.\\
4) Для любой невырожденной формы объёма $\omega$ верно утверждение: $\omega(v_1,\dots,v_n)=0$ тогда и только тогда, когда $v_1,\dots,v_n$ линейно зависимы.
\proof Определитель является полилинейной функцией столбцов, так как  каждое слагаемое содержит только одну координату из каждого столбца матрицы $A$. Осталось показать, что если в матрице $A$ пара столбцов одинакова, то её определитель равен нулю.
Пусть равны столбцы $k$ и $l$. Рассмотрим транспозицию $\tau=(kl)$. Тогда все перестановки разбиваются на классы $A_n$ и $A_n\tau$. Выпишем теперь сумму
$$\sum_{\sigma \in S_n} \sgn(\sigma)\prod_{j=1}^n a_{\sigma(j)j} = \sum_{\sigma \in A_n} \prod_{j=1}^n a_{\sigma(j)j}- \sum_{\sigma\tau \in A_n\tau} \prod_{j=1}^n a_{\sigma(\tau(j))j}.$$
Заметим, что $a_{\sigma(\tau(j))j}=a_{\sigma(j)j}$, если $j\neq k,l$. Но при  $j=k$, благодаря тому что $\tau(k)=l$  равенству столбцов, имеем $a_{\sigma(\tau(l))k}=a_{\sigma(l)k}=a_{\sigma(l)l}$. Аналогично $a_{\sigma(\tau(l))l}=a_{\sigma(k)k}$, что означает, что произведения для одной и той же $\sigma$ в правой и левой части суммы одинаковы.\\

Далее, используя предыдущую лемму, получаем $\omega(v_1,\dots,v_n)=\omega(e_1,\dots,e_n)\Vol_e(v_1,\dots,v_n)$. Осталось связать $\Vol_e$ и $\Vol_f$ для разных базисов. Для этого надо вычислить $\Vol_f(e_1,\dots,e_n)$. Для этого надо построить матрицу из координат столбцов $e_i$ в базисе $f$ и посчитать её определитель. Но это матрица $C_{e}^f$. Итого $\Vol_{f}=\det(C^f_{e}) \Vol_{e}$. \\
Все формы пропорциональны $\Vol_e$ для некоторого базиса $e$.
Пусть теперь $v_1,\dots,v_n$ набор векторов. Тогда, если $v_1,\dots,v_n$ линейно зависимы, то существует нетривиальная линейная комбинация $\sum \lambda_i v_i=0$. Пусть $\lambda_1\neq 0$, тогда можно считать, что $\lambda_1=1$. Тогда, прибавляя к первому аргументу остальные с соответствующими коэффициентами, получаем набор с нулевым первым вектором. $\omega(0,\dots)$ с одной стороны 0, а с другой равен определителю исходной матрицы.
Обратно, если $v_1,\dots,v_n$ независимы, то $\omega=\omega(v_1,\dots,v_n)\Vol_v$. Так как $\omega \neq 0$, то коэффициент $\omega(v_1, \dots,v_n)\neq 0$.
\endproof
\end{thm}




\lm Для определителей квадратных матриц верны следующие свойства:\\
0) $\det(A)=\det(A^{T})$.\\
1) Определитель не меняется при элементарных преобразованиях первого типа для строк и столбцов. При смене строк местами меняется знак определителя. При домножении строки на $\lambda$, определитель домножается на $\lambda$.\\
2) $\det(AB)=\det(A)\det(B)$.\\
3) $\det \left(\begin{matrix}A & B\\
0 & C \\
\end{matrix}\right)= \det(A)\det(C)$.\\
4) Определитель верхнетреугольной или нижнетреугольной матрицы равен произведению диагональных элементов.\\
5) $\det (A^{-1})=(\det A)^{-1}.$\\
6) $\det \colon \GL(V) \to K^*$
является гомоморфизмом групп.
\proof Свойство 0) мы уже доказали. Из него сразу же следует свойство 1). Докажем 2).\\
Заметим, что форма $B \to \det AB$ линейна по столбцам $B$. Тогда это форма объёма и следовательно она пропорциональна форме $B \to\det B$. Для того, чтобы найти коэффициент пропорциональности, подставим $B=E_n$. Имеем $\det AE_n=\det A= c \det E_n=c$. Откуда для любого $B$ получаем
$$\det AB=\det A \det B.$$\\
Покажем 3). Сначала посчитаем $$\det \pmat E_n& B\\
0& E_m\epmat.$$
Заметим, что с помощью элементарных преобразований 1-го типа из неё можно сделать $E_{n+m}$. Тогда определитель равен 1.
Теперь заметим, что форма $$A \to \det \pmat A & B\\ 0& E_m\epmat $$ форма объёма на $K^n$. Как и раньше подставив $A=E_n$ получаем что коэффициент пропорциональности с $A \to \det A$ равен 1.\\
Теперь отображение $$C \to \det \pmat A& B\\ 0 & C \epmat $$  есть форма объёма, по строчкам $C$. Подставляя $B=E_m$ находим коэффициент пропорциональности $\det A$, что и даёт
$$\det \pmat A&B\\ 0&C \epmat = \det A \det C.$$
 Применяя предыдущее условие несколько раз приходим к пункту 4).\\
Беря равенство $AA^{-1}=E_n $ и считая определитель получаем $\det A \det A^{-1}=E_n$, что завершает доказательство 5). Пункт 6) следует из 2).
\endproof
\elm



С вычислительной точки зрения формула для определителя бесполезна -- в ней $n!$ слагаемых. Но она даёт некоторые важные следствия. Например, то, что определитель есть однородный многочлен степени $n$ с целыми коэффициентами от элементов матрицы (следовательно, понятие определителя можно ввести над любым коммутативным кольцом по этой формуле).\\


\exm\\
Приведём пример, вычислив определитель Вандермонда.

$$\det \pmat 1 & \dots & 1\\
\lambda_1 & \dots & \lambda_n\\
\vdots &&\vdots\\
\lambda_1^{n-1}& \dots & \lambda_n^{n-1} \epmat= \prod_{i>j}(\lambda_i-\lambda_j).$$
% FIXME: не плохо бы привести здесь доказательство.


Мы стартовали с понятия объёма параллелепипеда в $\mb R^n$. Отображение $A \to |\det A|$ удовлетворяет всем предполагаемым свойствам объёма. Может ли тем не менее объём параллелепипеда считаться по-другому? \\
Нет. Так как мы знаем, что модуль определителя и объём одинаково изменяются при элементарных преобразованиях. Точнее:

\lm Пусть дано отображение $\Vol \colon M_n(\mb R) \to \mb R $, такое что \\
1) $\Vol(E_n)=1$\\
2) $\Vol(\dots,u+\lambda v,\dots,v,\dots)=\Vol(\dots,u,\dots,v,\dots)$\\
3) $\Vol(\dots,\lambda v,\dots)=|\lambda|\Vol(\dots,v,\dots)$\\
Тогда $\Vol(A)=|\det A|$.
\proof Покажем, что эти свойства однозначно позволяют вычислить $\Vol(A)$. Так как отображение $A\to |\det A|$ тоже им удовлетворяет, то это гарантирует их совпадение.  Действительно, если матрицы $A$ вырождена, то элементарными преобразованиями первого типа можно сделать так, чтобы один столбец $A$ стал нулевым. По третьему свойству $$\Vol(v_1,\dots,0,\dots,v_n)=\Vol(v_1,\dots,0\cdot 0,\dots,v_n)=0\cdot\Vol(v_1,\dots,0,\dots,v_n)=0.$$
Если же $A$ невырождена, то элементарными преобразованиями первого типа её можно привести к диагональному виду не меняя определителя. Теперь вынося по свойству 3) диагональные элементы приводим матрицу $A$ к единичному виду. Осталось воспользоваться свойством 1).
\endproof
\elm




Теперь дадим важное в дальнейшем определение:

\dfn Пусть $V$ -- пространство. Тогда линейное отображение $L\colon V \to V$ называется оператором на пространстве $V$. Пусть $e_1,\dots, e_n$ базис $V$. Тогда матрицей оператора $L$ в базисе $e$ называется матрица $L_e^e$ (базисы с разных концов взяты одинаковыми).
\edfn

Это определение удобно, потому что при фиксированном базисе композиции операторов $L_1\circ L_2$ соответствует произведение их матриц $A_1A_2$ в базисе $e$. Если бы базисы были выбраны не согласовано, этого бы не было. В частности оператору $L^n$ соответствует матрица $A^n$.

\dfn Пусть $L\colon V \to V$ -- линейный оператор. Тогда определим $\det L=\det A$, где $A$ -- матрица оператора в каком-то базисе.
\edfn

\rm Линейный оператор обратим тогда и только тогда, когда $\det L \neq 0$.
\erm

Это определение требует с одной стороны требует проверки корректности, а с другой пояснения. Корректность получается из соображения, что при смене базиса $e\to f$ матрица $A$ заменяется на $C_e^fA (C_e^f)^{-1}$. Но
$$\det C_e^fA (C_e^f)^{-1}= \det C_e^f \det A (\det C_e^f)^{-1}= \det A.$$



Теперь проинтерпретируем полученное определение. Рассмотрим какую-нибудь невырожденную форму объёма $\omega$ на пространстве $V$. Рассмотрим следующее отношение
$$\frac{\omega(L(e_1),\dots,L(e_n))}{\omega(e_1,\dots,e_n)}.$$
Если $L$ не обратим, то это 0. Если обратим, то $L(e_1),\dots,L(e_n)$ базис и $$\omega(L(e_1),\dots,L(e_n))=\omega(e_1,\dots,e_n) \det L_e^e.$$
Откуда всё выражение просто равно $\det L_e^e=\det L$. В частности не зависит ни от выбора формы объёма, ни от выбора базиса.
Это означает, что определитель оператора есть коэффициент изменения объёма параллелепипеда (вне зависимости от формы объёма и выбора параллелепипеда).\\



Возвращаясь к вещественным числам, заметим, что понятие определителя выросло из понятия объёма и немного его переросло. А именно, для набора элементов из пространства $\mb R^n$ можно посчитать не только объём, но ещё и некоторый знак. О смысле этого знака и пойдёт сейчас речь.

\dfn Будем говорить, что два базиса пространства $V$ над $\mb R$ одинаково ориентированы, если матрица перехода между ними имеет положительный определитель.
\edfn

\rm Отношение одинаковой ориентированности есть отношение эквивалентности.
\erm

\dfn Выбор одного из классов эквивалентности базисов вещественного векторного пространства $V$ называется заданием ориентации.
\edfn

\dfn Пусть $V$ -- векторное пространство над $\mb R$. Будем говорить, что линейный оператор $L\colon V \to V$ сохраняет ориентацию, если $\det L>0$ и не сохраняет, если $\det L<0$.
\edfn

\lm Сохраняющее ориентацию отображение переводит одинаково ориентированные базисы в одинаково ориентированные.
\proof Пусть $e$ -- базис. Если матрица $L$ это $A$, то $A$ -- это матрица перехода от $Le$ к $e$, что и доказывает утверждение.
\endproof
\elm

\exm\\
1) Симметрия относительно прямой в $\mb R^2$ меняет ориентацию.\\
2) Поворот сохраняет ориентацию.\\
3) Центральная симметрия меняет или нет ориентацию в зависимости от размерности пространства.


\dfn Определим группу операторов $\SL(V)=\{ L\colon V \to V \,|\, \det L=1\}$. Если $V$ -- вещественное векторное пространство, то это операторы, которые сохраняют понятие объёма и  выбор ориентации пространства.
\edfn



\section{Явные формулы в линейной алгебре}

Заметим, что так как определитель линеен по столбцам, то $\det (\dots,v,\dots)= \sum c_i v_i$, как функция от $v_i$ -- координат  столбца $v$. Мы сейчас найдём эти коэффициенты $c_i$.

\dfn Минором порядка $k$ матрицы $A_{m\times n}(K)$ называется определитель некоторой квадратной подматрицы $M_{I,J}=\det A_{I,J}$, где $|I|=|J|=k$. Если $A\in M_n(K)$, то алгебраическим дополнением элемента $a_{ij}$ называется $A^{ij}=(-1)^{i+j} M_{\ovl{i},\ovl{j}}$.
\edfn



\lm При разложении по $j$-ому столбцу имеет место формула  $$\det(A)=\sum_{i=1}^n a_{ij} A^{ij}.$$
\proof Прежде всего установим эту формулу в простейшем случае $$A=\pmat 1& *\\
0& A_{\ovl{1},\ovl{1}}\epmat. $$
Это следствие вычисления определителя блочной матрицы. Итак, пусть фиксирован столбец $j$. Как же теперь найти коэффициенты при $a_{ij}$ в разложении. Для этого надо вместо $j$-го столбца поставить стандартные базисный $e_i$ и посчитать определитель. Сделаем это. Напишем матрицу
$$B= \bordermatrix{
 & &       &  j &      & \cr
 & a_{11} && 0 &  & a_{n1}    \cr
 & \vdots& & \vdots& & \vdots\cr
 i &a_{i1}&\dots & 1 &\dots & a_{in} \cr
 & \vdots& & \vdots& & \vdots\cr
 & a_{1n} &  & 0 &      &a_{nn} } $$
и переставим по циклу столбцы, чтобы $j$-ый столбец стал первым, и $i$-ая строчка стала первой. Такое преобразование изменит знак определителя на $(-1)^{i-1+j-1}=(-1)^{i+j}$. А в правом нижнем блоке будет стоять нужная подматрица $A$.
\endproof
\elm
Это свойство называется разложением  определителя матрицы  по столбцу. Аналогичная формула верна и для разложения по строке.


В прошлом разделе мы научились по некоторой формуле определять обратима матрица или нет. Это полезно в различных задачах с параметром, где метод Гаусса напрямую не применим для вычисления определителя. Можно поставить новый вопрос: а можно ли написать формулу для решения системы линейных уравнений при некоторых ограничениях? Ответ на  этот вопрос дают формулы Крамера.

\thrm[Формула Крамера]
Пусть дана система линейных уравнений $Ax=b$ с квадратной матрицей $A$ над полем $K$. Если матрица $A$ --- обратима, то единственное решение этой системы имеет вид $$x_i=\frac{\Delta_i}{\Delta}, \text{ где } \Delta=\det A, \text{ а } \Delta_i \text{ --- определитель матрицы, полученной из  $A$ заменой $i$-го столбца на столбец $b$}.$$
\proof Пусть $x$ такой, что $Ax=b$. За $v_1,\dots,v_n$ обозначим столбцы $A$. Тогда $b=x_1v_1+\dots+x_nv_n$.  Теперь посчитаем определитель $\delta_i$. Это определитель матрицы $(v_1,\dots,v_{i-1},b,v_{i+1},\dots,v_n)$. Подставим $b=x_1v_1+\dots+x_nv_n$ и раскроем скобки. Останется ровно одно слагаемое с $x_iv_i$. Итого
$$\det (v_1,\dots,v_{i-1},b,v_{i+1},\dots,v_n)= x_i \det(v_1,\dots,v_{i-1},v_i,v_{i+1},\dots,v_n)=x_i\det A.$$
\endproof
\ethrm

Итак мы научились явно решать систему линейных уравнений. Заметим, в свою очередь, что столбец $u_j$ в матрице $A^{-1}$ есть решение уравнения $Au_j=e_j$. Тогда по формуле Крамера $$(A^{-1})_{ij}=\frac{\det(v_1,\dots,v_{i-1}, e_j,v_{i+1},\dots,v_n)}{\det A}=\frac{A^{ji}}{\det A}.$$
Это приводит нас к определению:

\dfn
Присоединённой матрицей к матрице $A$ называется матрица $(\Adj A)_{ij}= A^{ji}$ где $A^{ij}$ --- алгебраическое дополнение элемента $a_{ij}$.
\edfn

\thrm
Пусть $A\in M_n(K)$. Тогда имеет место соотношение
$$\Adj A \cdot A= A\cdot \Adj A= \det(A)\cdot E.$$
\proof Для определённости будем доказывать $A\cdot \Adj A= \det(A)\cdot E$. Заметим, что если матрица $A$ обратима, то это уже у нас в кармане. Что же делать для необратимых $A$?
Оказывается, что вместо того, чтобы пытаться как-то свести к тождеству для всех матриц над тем же полем $K$ проще доказать это тождество над любым кольцом. Точнее, заметим, что это тождество равносильно равенству нулю некоторых целочисленных многочленов от коэффициентов матрицы. Это тождество имеет смысл над любыми кольцами. Тогда, если мы доказали это тождество для кольца $R$ и матрицы $B$, и есть гомоморфизм $\ffi \colon R \to S$, то это тождество верно для матрицы $\ffi(B)$ над $S$.

Рассмотрим кольцо $\mb Z[a_{11}, \dots, a_{nn}]$ многочленов от $n^2$ независимых переменных. Над этим кольцом есть матрица $A$, что $A_{ij}=a_{ij}$. Эта матрица называется общей матрицей. Потому что для любой матрицы $B$ над кольцом $R$ есть гомоморфизм $\ffi \colon \mb Z[a_{11},\dots,a_{nn}]\to R $, что $\ffi(A)=B$. Иными словами в эту матрицу можно подставить любые значения.

Таким образом, достаточно доказать наше тождество для одной матрицы $A$. Для этого вложим кольцо
$$\mb Z[a_{11},\dots,a_{nn}] \to Q(\mb Z[a_{11},\dots,a_{nn}]).$$ Тогда достаточно доказать тождество для матрицы $A$ в $Q(\mb Z[a_{11},\dots,a_{nn}])$. Для этого покажем, что матрица $A$ невырождена. Заметим, что её определитель -- это ненулевой многочлен $$\sum_{\sigma \in S_n} \sgn(\sigma) \prod_{i=1}^n a_{i\sigma(i)} \in Q(\mb Z[a_{11},\dots,a_{nn}]),$$ который мы обычно и называем определителем. Раз определитель $A$ не ноль, то $A$ -- невырожденная матрица над полем $Q(\mb Z[a_{11},\dots,a_{nn}])$ и, следовательно для неё тождество выполнено. Что и требовалось.
\endproof
\ethrm

В этом доказательстве мы пользовались возможностью перейти к полю частных и тем, что общая матрица обратима хотя и не над кольцом $\mb Z[a_{11},\dots,a_{nn}]$, но в его поле частных. Морально, это означает, что  полиномиальное тождество верное для всех значений коэффициентов, кроме тех которые задаются условием $p= 0$ верно всегда, потому что таких решений $p=0$ <<мало>> -- ведь общая матрица им не удовлетворяет.

\zd Покажите, что над произвольным кольцом $\det(AB)=\det A \det B$.
\ezd

\zd Предыдущее утверждение можно доказать и руками расписав формулу разложения по столбцу для матриц вида $(v_1,\dots, v_k,\dots,v_k,\dots, v_n)$ где на позиции $i$ поставлен столбик $v_k$. Сделайте это.
\ezd


\section{Алгебры}

{\bf {\color{red} Внимание!!!}} Начиная с этого момента под словом кольцо я буду понимать ассоциативное кольцо с единицей. Коммутативно кольцо или нет теперь придётся уточнять.

Начнём издалека. Если есть два пространства $V_1$ и $V_2$ размерностей $n$ и $m$, то имеет место изоморфизм векторных пространств $\Hom(V_1,V_2)\cong M_{m\times n}(K)$. Ничего лучше не сказать, потому что никаких других структур на этих пространствах в общем случае нет. Однако если $V=V_1=V_2$, то пространство $\End(V)$ является ещё и кольцом относительно композиции.  Структура векторного пространства и кольца связаны (кроме дистрибутивности) ещё одним соотношением:
$$\forall \lambda \in K \text{ и } L_1, L_2 \in \End(V) \text{ верно, что } (\lambda L_1)\circ L_2= \lambda (L_1 \circ L_2)=L_1 \circ(\lambda L_2).$$


\dfn(Алгебра над коммутативным кольцом) Пусть $R$ -- коммутативное кольцо. Кольцо $S$ вместе с отображением $R \times S \to S$ называется алгеброй, если \\
1) $\forall r \in R$, $\forall u,v \in S$ $r(uv)=(ru)v=u(rv)$.\\
2) $S$ является $R$ модулем, то есть \\
2.1) $\forall r_1,r_2 \in R$ $\forall u \in S$ $r_1(r_2u)=(r_1r_2)u$.\\
2.2) $\forall r_1,r_2 \in R$ $\forall u \in S$ $r_1u+r_2u=(r_1+r_2)u$.\\
2.3) $\forall r \in R$ $\forall u,v \in S$ $ru+rv=r(u+v)$.\\
2.4)  $\forall u\in S$ $1\cdot u=u$.
\edfn



Указанное выше определение алгебры годится не только для ассоциативных колец, однако условие ассоциативности, которое мы всегда накладываем даёт возможность пересказать определение алгебры на другом языке.

\dfn[Центр кольца] Пусть $S$ -- кольцо. Его центром называется подкольцо
$$Z(S)=\{x\in S\,|\, \forall y \in S \,\, xy=yx\}.$$
\edfn

Докажем некоторую теорему, которая позволит пояснить почему те или иные кольца являются алгеброй над другими.


\thrm Задание на кольце $S$ структуры $R$-алгебры над коммутативным кольцом $R$ равносильно заданию гомоморфизма $R \to Z(S)$.
\proof
Пусть $S$ -- это $R$-алгебра. Определим отображение $R\to S$ по формуле $r \to r\cdot 1$. Это гомоморфизм. Покажем, что мы попали в центр. Действительно
$$s(r\cdot 1)= r\cdot (s1)=r\cdot s$$
С другой стороны,
$$(r\cdot 1)s=r\cdot(1s)=r\cdot s.$$
Обратно, если задан гомоморфизм $\ffi \colon R \to S$, то определим $r\cdot s=\ffi(r)s$. Это умножение задаёт структуру $R$-алгебры.
\endproof
\ethrm

\exm \\
1) Множество всех эндоморфизмов $\End(V)=\Hom(V,V)$ векторного пространства $V$ над полем $K$ является ассоциативной алгеброй с единицей над полем $K$.\\
2) Пусть $R$ и $S$ -- коммутативные кольца. Тогда задание гомоморфизма $R \to S$ равносильно заданию на $S$ структуры $R$-алгебры. \\
3) Любое кольцо -- алгебра над $\mb Z$.\\
4) Кольцо $R[x_1,\dots,x_n]/I $ -- алгебра над $R$.\\
5) $\mb Z/p$ является $\mb Z/p^2$ алгеброй, но не наоборот.\\





\dfn Отображение $f \colon S_1 \to S_2$, где $S_1$ и $S_2$ являются $R$-алгебрами,
называется гомоморфизмом алгебр, если $f$ -- гомоморфизм колец и $f(rs)=rf(s)$ для всех $r \in R$.
\edfn



\rm Кольцо эндоморфизмов $\End(V)$  изоморфно, как алгебра, алгебре матриц $M_{\dim V}(K)$.
\erm

\rm Кольцо эндоморфизмов $\End(V)$ некоммутативно, если $\dim V \geq 2$. Группа $\GL(V)=\End(V)^*$ некоммутативна при тех же условиях.
\erm

\zd Найдите центр кольца $M_n(K)$.
\ezd






\thrm[Теорема типа Кэли] Любая алгебра $A$, конечномерная над полем $K$ вкладывается в $\End_K(A)$.
\proof Пусть $x\in A$. Тогда рассмотрим отображение $L_x \colon A \to A$, заданное по правилу $L_x(z)=xz$. Тогда $L_x\circ L_y=L_{xy}$. Так же $L_x+L_y=L_{x+y}$, а $L_1=\id_A$. Заметим, что, если $x\neq 0$, то $L_x$ не нулевое, потому что $L_x(1)=x\neq 0$.
\endproof
\ethrm

Мы с вами отлично помним, что если коммутативные кольца $R$ и $S$ связаны гомоморфизмом $f\colon R \to S$, иными словами, если на $S$ задана структура $R$-алгебры, то имеет смысл говорить про решения в $S$ системы полиномиальных уравнений, заданных на $S$. Вопрос -- существенна ли здесь коммутативность? Ответ: да, но иногда нет.

\rm
Пусть $K$ -- поле, $A$ -- алгебра над $K$. Тогда задание элемента $y \in A$ определяет единственный гомоморфизм $K$-алгебр $\ffi \colon K[x]\to A$, что $\ffi(x)= y$. Однако в общем случае это  гомоморфизм колец не задающий на $A$ структуру $K[x]$ алгебры. Образ многочлена $p$ естественно обозначать $p(y)$.
\erm

Однако не всё так просто, если переменных больше.

\rm Пусть $a,b$ два элемента алгебры $A$, которые не коммутируют между собой. Тогда не существует гомоморфизма $K[t_1,t_2]$ переводящего $t_1\to a$, и $t_2 \to b$.
\erm

\zd Является ли это необходимым условием? Каков ответ для $K[x_1,\dots,x_n]$?
\ezd




\thrm Любой элемент $y$ конечномерной алгебры $A$ удовлетворяет уравнению $p(y)=0$, где $p(x)\in K[x]$, $p(x)\neq 0$.
\proof Рассмотрим набор элементов $1,y,\dots, y^{\dim A}$. Они линейно зависимы. Следовательно, их нетривиальная линейная комбинация равна нулю, что и даёт уравнение.
\endproof
\ethrm


\dfn Ядро гомоморфизма $K[x] \to A$, переводящего $x \to y$ является идеалом в $Ann_y\leq K[x]$. Его элементы называют аннуляторами $y$. Если этот идеал не 0, то его образующую со старшим коэффициентом 1 называют минимальным многочленом элемента $y\in A$ и обозначают $\mu_y(x)$.
\edfn

Мы только что показали, что в конечномерной алгебре у любого элемента есть минимальный многочлен. Получим из этого простое следствие:

\thrm Любой элемент $y$ конечномерной алгебры $A$ над полем $K$ либо обратимым, либо делитель нуля ( с любой стороны).
\proof Рассмотрим минимальный многочлен $\mu_y(x)= x^n+ \dots+a_0$. Пусть $a_0$ свободный член $p(y)$. Если $a_0=0$, то $$y(y^{n-1}+
\dots + a_1)=(y^{n-1}+
\dots + a_1)y=0.$$ Благодаря минимальности оба выражения отличны от 0, что показывает, что $y$ делитель нуля с любой стороны. Если же $a_0\neq 0$, то $$y^{-1}=\frac{1}{-a_0}(y^{n-1}+
\dots + a_1).$$ В частности, обратный элемент в конечномерной алгебре всегда есть многочлен от исходного. Одновременно оба условия выполенены быть не могут так как, если $yb=0$ и $y^{-1}y=1$, то $b=y^{-1}yb=0$.
\endproof
\ethrm








\section{Линейные операторы}

Рассмотрим последовательность $x_{n+2}=x_{n+1}+x_{n}$, $x_0=a,x_1=b$. Как посчитать $x_{1000}=?$. Для того, чтобы воспользоваться рекуррентой, надо сделать 1000 операций. Можно ли меньше? С одной стороны вы знаете ответ -- надо найти характеристический многочлен и посчитать его корни, а потом свести всё к вычислению геометрической прогрессии.  Однако в этом случае для получения точного ответа придётся возиться с  иррациональными корнями. Попробуем сделать по другому. Заменим наше соотношение системой
$$ \begin{cases} x_{n+1}=x_n+y_n \\
y_{n+1}=x_{n}
\end{cases}.$$
Перепишем её в следующем виде
$$ \pmat x_{n+1}\\ y_{n+1}\epmat = A \pmat  x_{n}\\ y_n \epmat, \text{ где } A=\pmat 1& 1\\ 1& 0 \epmat.$$
Тогда  $x_{1000}$ это первая координата столбца $A^{1000} \pmat b\\ a\epmat $. Итого достаточно просто возвести матрицу в 1000 степень. Заметим, что это частный случай общей задачи: имеется последовательность $x_n$ из $K^m$ удовлетворяющая соотношению $x_{n+1}=Ax_n$. Требуется найти какой-то её член. Например -- дан  набор состояний $s_1,\dots,s_k$ и даны вероятности перехода между состояниями $a_{ij}$ за один шаг. Вопрос: что произойдёт с системой после $n$ шагов?
Итак, вопрос можно ли как-то в общем виде упростить возведение матрицы в степень? Замечу, что это вопрос напрямую связан со  свойством произведения матриц, или, более инвариантно,   операторов на пространстве $V$. Ближайшей нашей темой будет разбирательство с новой структурой -- пространством с оператором.

\dfn Два пространства с $V_1$ и $V_2$ с операторами $L_1$ и $L_2$ эквивалентны, если существует изоморфизм векторных пространств $C \colon V_1 \to V_2$, что $L_2=C \circ L_1 \circ C^{-1}$.
\edfn


\rm Две матрицы $A, B \in M_n(K)$ эквивалентны как операторы на $K^n$, тогда и только тогда, когда существует матрица $C \in \GL_n(K)$, что $A=CBC^{-1}$. Часто такие матрицы называются подобными. Эквивалентные операторы имеют подобные матрицы при любом выборе базиса.
\erm

\rm Два оператора эквивалентны тогда и только тогда, когда в некоторых базисах их матрицы одинаковы.
\proof Пусть $L_2=C \circ L_1 \circ C^{-1}$. Пусть $e_1,\dots,e_n$ базис $V_1$, а $C(e_1),\dots,C(e_n)$ базис $V_2$. В указанных базисах матрица $C$ единичная, что даёт равенство матриц $L_1$ и $L_2$.
\endproof
\erm


\dfn Пусть $V$ -- пространство с оператором $L$. Пусть $U\leq V$. Тогда $U$ называется инвариантным подпространством, если $L(U) \leq U$.
\edfn

\rm Это условие позволяет сузить оператор $L$ с $V$ на $U$.
\erm

Посмотрим на простейший случай, когда инвариантное пространство одномерно. Пусть оно порождено вектором $v$. Тогда условие инвариантности перепишется как $L(v) = \lambda v$ для некоторого $\lambda \in K$.


\dfn Пусть $V$ -- пространство с оператором $L$. Тогда вектор $0\neq v\in V$ называется собственным вектором c с собственным числом $\lambda$ относительно оператора $L$, если $Lv=\lambda v$.
\edfn

\exm \\
1) Рассмотрим пространство  последовательностей и оператор сдвига $S(x)_n= x_{n+1}$. Тогда собственный вектор -- это геометрическая прогрессия.\\
2) Рассмотрим тот же контекст. Тогда несложно увидеть, что подпространство последовательностей, удовлетворяющих линейному рекуррентному соотношению с постоянными коэффициентами является инвариантным относительно оператора сдвига.\\
3) Рассмотрим алгебру $K[x]/p(x)q(x)$. Тогда подпространство многочленов делящихся на $p(x)$ является инвариантным относительно оператора $f(x) \to x f(x)$ домножения на $x$.\\
4) Подпространство многочленов степени меньшей или равной $n$ инвариантно относительно оператора дифференцирования.\\
5) Если $p(x)$ многочлен, то $\Ker p(L)$ инвариантно относительно $L$.\\
6) Пусть $v \in V$. Тогда  $V'=\lan v, Lv, L^2v,\dots \ran$ является инвариантным пространством, порождённым $v$.\\


\lm Пусть $U\leq V$ -- подпространство, а $L \colon V \to V$ -- линейный оператор. Тогда $U$ инвариантно относительно $L$ тогда и только тогда, когда в базисе $e_1,\dots,e_k,e_{k+1},\dots,e_n$, где $e_1,\dots,e_k$ базис $U$ матрица оператора имеет блочно диагональный вид
$$\pmat A&B\\
0&C \epmat$$
\proof Образ $L(e_i)$ при $i \leq k$ лежит в $U$ и раскладывается по базису $U$, то есть по первым $k$ векторам.
\endproof
\elm


\rm У нас снова всплыли блочные матрицы и на этот раз нам необходимо обсудить как перемножаются матрицы в таком виде. Общая формулировка выглядит так. Если есть две матрицы
$$A=\pmat A_{11} & A_{12}\\
A_{21}& A_{22}
\epmat \text{ и } B=\pmat B_{11} & B_{12}\\
B_{21}& B_{22}
\epmat $$
которые перемножаемы, плюс ко всему размеры $A_{ik}$ и $B_{kj}$ согласованы, то тогда
$$AB= \pmat A_{11}B_{11}+ A_{12}B_{21} & A_{11}B_{12}+ A_{12}B_{22}\\
A_{21}B_{11}+ A_{22}B_{21}& A_{21}B_{12}+ A_{22}B_{22}\epmat. $$
То есть матрицы перемножаются по блокам.
\erm

Как найти инвариантные подпространства? Например, собственные векторы.

\dfn Определим характеристический многочлен оператора $L$ как $\chi_L(t)=\det(tE_n - A)$, где $A$ -- матрица $L$ в некотором базисе.
\edfn

\lm Характеристический многочлен корректно определён.
\elm
\proof Пусть $A$ матрица оператора $L$ в базисе $e$, $A'$ -- в базисе $f$, а $C$ матрица перехода из $e\to f$. Тогда $A'=CAC^{-1}$. Рассмотрим $\chi_A(t)$, как элемент $K(t)$. Тогда $C$ -- тоже матрица над $K(t)$ и
$$\det(tE-A')=\det(C(tE-A)C^{-1})=\det(C)\det(tE-A)\det C^{-1}=\det(tE-A).$$
Раз эти выражения равны как элементы $K(t)$, то и как элементы $K[t]$.
\endproof

\lm Элемент $\lambda \in K$ является собственным числом оператора $L$ тогда и только тогда, когда $\lambda$ корень $\chi_L(t)$.
\proof $\lambda$ собственное число тогда и только тогда, когда есть ненулевой $v$, что $Lv=\lambda v$ тогда и только тогда, когда $(\lambda \id - L)v=0$ тогда и только тогда, когда матрица этого оператора вырождена тогда и только тогда, когда $\det (\lambda E- A) =0$.
\endproof
\elm


Коэффициенты многочлена $\chi_L(t)$ являются инвариантами оператора $L$. Давайте посмотрим на них чуть внимательнее.
\rm $\det L = (-1)^{n} \chi_L(0)$
\erm

\dfn Пусть $A$ -- матрица размера $n$, тогда $$\Tr A=\sum_{i=1}^n a_{ii}.$$
\edfn

\rm След матрицы $A$ это $-a_{n-1}$, где $\chi_A(t)= \sum a_{i}t^i$. В частности, можно определить след  оператора $L$ как след его матрицы.
\erm

Имеют место следующие свойства следа

\lm След обладает следующими свойствами: \\
1) $\tr CAC^{-1}= \tr A$ для обратимой матрицы $C$.\\
2) След равен сумме собственных чисел с учётом кратностей, как корней характеристического многочлена (над любым полем, где характеристический многочлен раскладывается на линейные множители).\\
3) $\tr AB = \tr BA$ для квадратных $A$ и $B$.\\
4) $\tr A= \tr A^{\top}$.\\
5) $\tr( A + \lambda B) = \tr A + \lambda \tr B$.\\
\proof
1) следует из инвариантности. 2) Многочлен $\chi(t)=\prod (t-\lambda_i)$. Осталось раскрыть скобки. 3) Если расписать, то равенство эквивалентно
$$\sum_{i,k} a_{ik}b_{ki}=\sum_{j,l} b_{jl}a_{lj}$$
Которое верно, особенно если взять $k=j$ и $i=l$. 4),5) ясно. \endproof
\elm

\rm Аналогично определитель равен произведению собственных чисел с учётом кратностей.
\erm

\dfn Кратность собственного числа $\lambda$ оператора $L$ как корня $\chi_L(t)$ называется его алгебраической кратностью.
\edfn









\dfn Оператор называется диагонализуемым,  если в некотором базисе $V$ его матрица диагональна.
\edfn

\lm Матрица оператора $L$ в базисе $v_1,\dots,v_n$ диагональна тогда и только тогда, когда все $v_i$ -- собственные вектора $L$. В этом случае на диагонали матрицы стоят собственные числа оператора $L$.
\proof Если $Lv_i=\lambda_iv_i$, то в $i$ столбце $\lambda_i$ стоит на диагонали, а остальное -- $0$.
\endproof
\elm

\lm Пусть $v_1,\dots,v_n$ собственные вектора $L$. Если при этом для каждого с.ч. $\lambda$ набор всех векторов $v_{i_1}\dots,v_{i_k}$ с одинаковыми $\lambda=\lambda_{i_1}=\dots=\lambda_{i_k}$  линейно независим, то тогда и все $v_1,\dots,v_n$ линейно независимы.
\proof Пусть есть линейная комбинация $$\sum c_i v_i=0,$$
состоящая из минимального числа векторов (пусть они нумеруются от 1 до $k$). Тогда так же
$$0=L\left(\sum c_i v_i\right)= \sum c_i \lambda_i v_i$$
Умножая на $\lambda_1$ и вычитая получаем $$0=\sum_{i=2}^k c_i(\lambda_1-\lambda_i)v_i$$
Если в последней сумме есть хоть одно ненулевое слагаемое, то приходим к противоречию. Обратное бывает только если $\lambda_i=\lambda_1$ для всех $i$. Но в этом случае мы можем воспользоваться независимостью из условия леммы.
\endproof
\elm

\dfn Алгебраической кратностью собственного числа $\lambda$ называется его кратность как корня характеристического многочлена. Геометрической кратностью $\lambda$ называется размерность $\Ker \lambda \id - L$.
\edfn

\thrm Пусть $K$ -- алгебраически замкнутое поле.  Тогда оператор $L$ диагонализуем в том и только том случае, когда для всякого собственного числа $\lambda$ его алгебраическая кратность равна его геометрической кратности. Более того, алгебраическая кратность собственного числа не меньше его геометрической кратности.
\ethrm
\proof Пусть $L$ диагонализуем, то $\dim \Ker \lambda \id - L$ совпадает с количеством  $\lambda$ на диагонали, что равно алгебраической кратности собственного числа.

Обратно. Сумма алгебраических кратностей $k_i$ собственных чисел равна $$\sum k_i=\deg \chi_L(t)=n=\dim V.$$
По условию для различных корней $\lambda_i$, можно выбрать $k_i$ линейно независимых векторов $v_{i1},\dots,v_{ik_i}$. Если объединить эти наборы, то по предыдущей лемме они будут независимы. Их ровно $n$ штук. Следовательно это базис.
Если есть $k_i$ линейно независимых собственных векторов, то их можно дополнить до базиса. У полученной матрицы оператора будет блочный вид, в первом блоке которого будет стоять матрица $\lambda_iE_{k_i}$. Следовательно характеристический многочлен делится на $(t-\lambda_i)^{k_i}$.
\endproof



\crl[Критерий диагонализуемости] Пусть $K$ -- алгебраически замкнутое поле. Если характеристический многочлен не имеет кратных корней, то оператор $L$ диагонализуем.
\proof В этом случае алгебраическая кратность наименьшая возможная.
\endproof
\ecrl






\section{Жорданова форма}

В этом разделе я буду предполагать, что  поле $K$ алгебраически замкнуто. Ровно те же результаты можно получить, предполагая, что поле $K$ содержит корни характеристического многочлена рассматриваемого оператора. Как я уже обмолвился в этом разделе мы получим полную классификацию операторов над алгебраически замкнутым полем $K$, то есть для каждого класса эквивалентности мы построим некоторую каноническую модель и научимся сравнивать оператор с его моделью.



\lm Пусть $L$ -- оператор на пространстве $V$, а  многочлен $g(t)=p(t)q(t)$ аннулятор $L$, причём $(p(t),q(t))=1$. Тогда подпространство $V$ раскладывается в прямую сумму
$$V = \Ker p(L)\oplus \Ker q(L).$$
\elm
\proof Рассмотрим линейное разложение $1=a(t)p(t)+b(t)q(t)$. Тогда любой вектор $v$  представим в виде
$$v=a(L)p(L)v+ b(L)q(L)v.$$
Тогда $q(L)a(L)p(L)v=0=p(L)b(L)q(L)v$. Следовательно, $V= \Ker p(L)+\Ker q(L)$. Покажем, что ядра пересекаются по нулю. Пусть $v\in \Ker p(L) \cap \Ker q(L)$. Тогда $v=a(L)p(L)v+ b(L)q(L)v=0$.
\endproof

\dfn Пусть $p(x)$ -- неприводимый многочлен над $K[x]$. Пространство $V$ с оператором $L$ называется $p$-примарным, если существует $\alpha \in \mb N$, что $p^{\alpha}$ аннулирует $V$.
\edfn

\dfn Пусть $V$ -- пространство с оператором $L$, а $V'$ -- инвариантное подпространство. Тогда определим оператор $\ovl{L}$ на $V/V'$ следующим образом:
$$\ovl{L}(\ovl{v})=\ovl{L(v)}.$$
\edfn

\rm Если $p(x)$ -- многочлен, а $v\in V$, то $p(\ovl{L})\ovl{v}=\ovl{p(L)v}$.
\erm

\rm Если $e_1,\dots,e_n$ базис $V$ и $\lan e_1,\dots,e_k\ran$ инвариантное пространство относительно $L$. Если  матрица $L$ в этом базисе имеет вид $$\pmat A& B \\ 0 & C\epmat,$$
то $C$ -- это матрица $\ovl{L}$ в базисе
$\ovl{e_{k+1}},\dots,\ovl{e_n}$.
И, следовательно, $$\chi_L(t)=\chi_{L|_{V'}}(t)\cdot \chi_{\ovl{L}}(t).$$
\erm

\thrm[Теорема Гамильтона-Кэли] Пусть $L$ -- оператор на $V$. Тогда $\chi_L(L)=0$.
\ethrm
\proof Докажем по индукции. Случай $\dim V=1$ ясен. Шаг. Так как $K$ алгебраически замкнуто, то у характеристического многочлена есть корень $\lambda_1$ и собственный вектор $e_1$. Рассмотрим фактор $V/\lan e_1\ran$. Для него теорема выполнена. Заметим, что $$\chi_L(t)= (t-\lambda_1)\chi_{\ovl{L}}(t).$$
Пусть $v \in V$. Тогда $\chi_{\ovl{L}}(L)v = ce_1$ так как в факторе этот элемент равен 0. Но тогда
$$\chi_L(L)v= (L-\lambda_1 E)\chi_{\ovl{L}}(L)v=(L-\lambda_1 E) ce_1=0$$
\endproof











\dfn
Матрица $k\times k$
$$J_k(\lambda) = \begin{pmatrix}
\lambda& 1&& \\
& \lambda &1& \\
&&\ddots &\ddots& \\
&  && \lambda & 1\\
&  &&& \lambda
\end{pmatrix}
$$
называется жордановой клеткой размера $k$ с собственным числом $\lambda$.
\edfn

\rm Заметим, что для того, чтобы в базисе $e_1,\dots,e_n$ матрица оператора $L$ была жордановой клеткой необходимо и достаточно, чтобы $(L-\lambda E)e_i=e_{i-1}$ для $i\geq 2$ и $(L-\lambda E)e_1=0$. В частности оператор $L-\lambda E$ должен быть нильпотентным.
\erm







\thrm Пусть $L\colon V \to V$ --- оператор на конечномерном пространстве над алгебраически замкнутым полем $K$.
Тогда существует базис $e_1,\dots, e_n$ в котором матрица $L$ имеет вид
$$A=\begin{pmatrix}
J_{k_1}(\lambda_1) &&&\\
& J_{k_2}(\lambda_2) &&\\
&& \ddots& \\
&&& J_{k_s}(\lambda_s)

\end{pmatrix}.
$$

Более того, такая матрица единственна с точностью до перестановки блоков. Эта матрица называется матрицей оператора в форме Жордана. Базис, в котором матрица оператора имеет такой вид называется жордановым базисом.
\ethrm
\proof

Сначала докажем единственность. Прежде всего если нам дана матрица в жордановой форме, то мы легко можем вычислить её характеристический многочлен. Он равен $\prod (t-\lambda_i)$ где $\lambda_i$ -- все числа на диагонали, откуда сразу становится ясно, что $\lambda_i$ -- собственные числа $L$. Более того, алгебраическая кратность
$k$ собственного числа $\lambda$ равна сумме размеров клеток с этим собственным числом.

Пусть сами размеры клеток для собственного числа $\lambda$ заданы как набор чисел $h_i$. Имеем
$\sum h_i =k.$
Итак, набору клеток соответствует разбиение числа $k$ в сумму некоторого числа слагаемых. Удобно упорядочить эти слагаемые по возрастанию $h_i\geq h_{i+1}$.

Можно ли как-то лучше визуализировать себе структуру жордановой формы?
Каждому разбиению числа на слагаемые однозначно соответствует картинка


\begin{figure}[hhh]
\begin{center}
\begin{tikzpicture}[xscale=0.7, yscale=0.7]
\draw (1,0) -- (1,3) -- (0,3) -- (0,0) -- (4,0) -- (4,1) -- (0,1);
\draw (0,2) -- (3,2) -- (3,0);
\draw (2,0) -- (2,2);


\end{tikzpicture}
\end{center}
\caption{$8=3+2+2+1$, соответствует одной клетке размера 3, двум клеткам размера 2, одной клетке размера 1}
\end{figure}

А именно, сопоставим каждому $h_i$ столбик высоты $h_i$. Такие (обычно, правда, перевёрнутые) картинки называются диаграммами Юнга.

Как же восстановить эту картинку зная оператор $L$ и собственное число $\lambda$?

Рассмотрим оператор $N=L-\lambda E$. Понятно, что сколько клеток с $\lambda$ было у $L$ в указанном базисе, столько же клеток с с.ч. 0 будет и у $N$ и они будут такого же размера.
Заметим, что жордановы клетки с собственным числом 0 являются нильпотентными матрицами.



 Вспомним, что каждой клетке соответствует базис $v_i, N v_i, \dots,N^{h_i} v_i$. Заметим, что эти вектора образуют базис пространства $\Ker N^k$. Действительно, $N^k$ обнуляется на векторах $v_i$ и их образах. С другой стороны $N^k$ обратим на дополнительном слагаемом.

 Пририсуем эти вектора к нашей картинке следующим образом -- поместим $v_i$ наверху соответствующего клетке  столбика, $N v_i$ на ступень ниже и т.д. Таким образом заполним все ячейки диаграммы. При действии оператора $N$ на диаграмму происходит следующее -- все вектора съезжают на единицу вниз, кроме самых нижних, которые переходят в $0$.
\begin{figure}[hhh]
\begin{center}
\begin{tikzpicture}[xscale=0.7, yscale=0.7]
\draw (1,0) -- (1,3) -- (0,3) -- (0,0) -- (4,0) -- (4,1) -- (0,1);
\draw (0,2) -- (3,2) -- (3,0);
\draw (2,0) -- (2,2);
\draw[->] ( -0.5, 2.5) -- (-0.5, 0.5);
\node at (0.5, 2.5) {$v_1$};
\node at (0.5, 1.5) {$Nv_1$};
\node at (0.5, 0.5) {};
\node at (1.5, 1.5) {$v_2$};
\node at (1.5, 0.5) {$Nv_2$};
\node at (2.5, 1.5) {$v_3$};
\node at (2.5, 0.5) {$Nv_3$};
\node at (3.5, 0.5) {$v_4$};
\node at (-1.5, 1.5) {$N$};
\end{tikzpicture}
\end{center}
\caption{$8=3+2+2+1$, расставляем базисные вектора}
\end{figure}
Итого, количество ячеек в диаграмме Юнга для собственного числа $\lambda$ оператора $L$ на высоте не более $s$ равно $\dim \Ker(L - \lambda E)^s $.
Это  позволяет однозначно восстановить разбиение числа и, следовательно, конфигурацию клеток, если мы знаем числа $\dim \Ker(L - \lambda E)^s $.

Точнее, число ячеек в строке уровня $s$ равно $\dim \Ker(L - \lambda E)^s -\dim \Ker(L - \lambda E)^{s-1} $.







\proof[Существование]


Для начала надо разбить всё пространство на куски с одним и тем же собственным числом. По теореме Гамильтона-Кэли оператор $L$ аннулируется многочленом $\chi_L(t)$. Разложив последний на простые множители $\chi_L(t)=\prod (x-\lambda_i)^{\alpha_i}$ получим разложение $V=\bigoplus V_i$ на примарные инвариантные подпространства $V_i=\Ker (L-\lambda_i E)^{\alpha_i}$. Ограничимся на пространства $V_i$. Заметим, что оператор $N=L-\lambda_i E$ нильпотентен. Осталось применить следующую  лемму к операторам $N|_{V_i}$.

\lm Для любого нильпотентного оператора $N$ на пространстве $V$ существует базис $e_1,\dots,e_n$ в котором матрица $N$ имеет вид
$$A=\begin{pmatrix}
J_{k_1}(0) &&&\\
& J_{k_2}(0) &&\\
&& \ddots& \\
&&& J_{k_s}(0)
\end{pmatrix}.
$$
\elm
\proof
Докажем по индукции. Если $V=\Ker N$, то матрица просто 0 и всё доказано. Иначе рассмотрим пространство $\Ker N$. Тогда для $V/\Ker N$ теорема верна. Рассмотрим требуемый базис $V/\Ker N$ и поднимем его $\ovl{e_{11}},\dots,\ovl{e_{1h_1}},\dots, \ovl{e_{sh_s}}$, где первый индекс обозначает номер клетки, а второй высоту ячейки диаграммы Юнга. Пусть $h_i\geq h_j$, если $i\geq j$.
Рассмотрим вектора $e_{ih_i}$, которые переходят в $\ovl{e_ih_i}$ и определим $$e_{ij}=N^{h_i-j}(e_{ih_i}),$$ где  $j\in \ovl{1,h_i}$. Имеет место равенство $$\pi{e_{ij}}=\ovl{e_{ij}}.$$
Рассмотрим набор векторов $e_i=N(e_{i,1})$ -- это вектора из $\Ker N$.
Покажем, что они линейно независимы и, в частности, не 0. Пусть $\sum_{i=1}^s c_ie_i=0$. Это значит $N(\sum_{i=1}^s c_i e_{i,1} )=0$. Тогда $\sum_{i=1}^s c_i e_{i,1} $ лежит в ядре $N$. Но в этом случае $\sum c_i \ovl{e_{i,1}}=0$ в $V/\Ker N$, чего не может быть, так как они там линейно независимы.

Теперь дополним набор $e_1,\dots,e_s$ до базиса $\Ker N$  элементами $e_{s+1},\dots,e_k$. Я утверждаю, что $$e_1, e_{11},\dots, e_{1,h_1},e_2\dots, e_s, \dots, e_{s,h_s}, e_{s+1},\dots, e_k$$
нужный базис. Для этого необходимо показать, что он базис. Рассмотрим отображение $V \to V/\Ker N$. Тогда часть этого набора образует базис образа, а оставшаяся часть -- базис ядра. Тогда размерность пространства, порождённого этими векторами равна $\dim \Ker N + \dim V/\Ker N = \dim V$. Откуда получаем, что это базис.
\endproof

\endproof








Допустим мы нашли характеристический многочлен, всё собственные числа. Далее нашли размерности $\dim \Ker(L-\lambda E)^s$. Как теперь найти сам жорданов базис? Для этого нам необходимо заполнить верхушки всех столбцов, остальное заполнится автоматически.



Как расставить векторы в верхней строке диаграммы? Векторы $v_{i_1}, \dots, v_{i_s}$  в верхней строке определяются тем, что их образы при $(L-\lambda E)^{k-1}$ линейно независимы (в частности, не лежат в ядре). Или, (что эквивалентно) система $v_{i_1}, \dots, v_{i_s}$ вместе с базисом (любым) $\Ker (L-\lambda E)^{k-1}$ образуют линейно независимую систему. Напомню, что их число равно
$$s=\dim \Ker (L-\lambda E)^k - \dim \Ker (L-\lambda E)^{k-1}.$$



Что делать с теми клетками, чьи столбики в диаграмме Юнга начинаются не на самом верху? Пусть мы уже заполнили все строки на высоте больше $i$. Заполним остаток строки на высоте $i$.  Очевидно, что оставшиеся векторы лежат в ядре $(L-\lambda E)^{i}$ и при этом их образы при $(L-\lambda E)^{i-1}$ линейно независимы. Однако вектора из уже заполненных клеток на уровне $i$ тоже подходят под это описание. Можно однако заметить, что образы системы <<старые вектора на уровне $i$>>, <<новые вектора на уровне $i$>> при $(L-\lambda E)^{i-1}$ линейно независимы все вместе. Это даёт необходимые условия на оставшиеся вектора в строке $i$.


\lm
Кратность множителя $x-\lambda$ в минимальном многочлене равна наибольшему размеру клетки с собственным числом $\lambda$.
\proof Прежде всего минимальный многочлен клетки $J_k(\lambda)$ имеет вид $(t-\lambda)^k$. Действительно минимальный многочлен делится на характеристический благодаря теореме Гамильтона-Кэли, который и есть $(t-\lambda)^k$. С другой стороны, $(L-\lambda)^{k-1}$ не нулевой оператор -- он не обнуляет самый высокий вектор клетки.
Теперь, если матрица $A$ оператора состоит из клеток $J_1,\dots, J_s$, то минимальный многочлен $\mu_A$ должен аннулировать каждую из этих клеток, но тогда $\mu_A \di \mu_{J_l}$ для всех $l$. Откуда $\mu_A= \Nok_{l=1}^s \mu_{J_l}$, что и приводит к нужному ответу.
\endproof
\elm

Жорданова форма позволяет анализировать как ведёт себя степень от оператора и более общо -- многочлен от оператора.

\thrm
Пусть $L$ -- оператор. Тогда матрица оператора $p(L)$ в жордановом базисе $L$ составлена из блоков вида
$$ \pmat p(\lambda) & p'(\lambda) & \dots & \frac{p^{(k-1)}(\lambda)}{(k-1)!}\\
 &  p(\lambda) & &\\
 &            & \ddots & \\
 &&&  p(\lambda) \epmat,$$
где $\lambda_i$ собственные числа, а число и размер блоков c $\lambda_i$ равны числу и размеру блоков в жордановой форме $L$.
\proof Благодаря линейности по многочлену, достаточно проверить равенство только для возведения $L$ в степень. Матрица $L$ в жордановом базисе составлена из жордановых блоков. Далее, заметим, что достаточно доказать утверждение в случае одного блока. Итак пусть мы имеем матрицу $J_k(\lambda)$. Такая матрица представима в виде суммы $J_k(\lambda)= \lambda E + N$, где $N$ -- нильпотентная матрица, причём $N^k=0$. На самом деле степени $N$ выглядят следующим образом:
$$N^l= \pmat  & &0& 1& \dots &0 \\
   & && \ddots &\ddots& \vdots\\
 &&&&0& 1\\
 &&&&& 0 \\
 &&&&&  \epmat $$
Теперь
$$(\lambda E+N)^n= \sum_{i=0}^{k-1} C_n^i\lambda^{n-i}N^i,$$
что и даёт требуемое.
\endproof
\ethrm

\crl Пусть $A$ матрица, тогда $p(A)=C p(J) C^{-1}$, где $p(J)$ составлена из блоков как выше, а $C$ составлена из жорданового базиса для $A$.
\proof Если $J$ -- жорданова форма для матрицы $A$, то $A=CJC^{-1}$, где матрица $C$ составлена из собственных векторов $A$. Далее необходимо применить предыдущую теорему и то, что отображение $B \to CBC^{-1}$ является гомоморфизмом $M_n(K) \to M_n(K)$.
\endproof
\ecrl

\crl Для всякой матрицы $A$ элемент её степени $A^n$ есть сумма выражений вида $p_i(n)\lambda_i^n$, где $\lambda_i$ -- собственное число, а $p_i(n)$ -- многочлен степени строго меньшей, чем максимальный размер жордановой клетки с собственным числом $\lambda_i$.
\ecrl

\crl Пусть $A$ -- вещественная (или комплексная) матрица с собственным числом $\lambda_1=1$ кратности 1, а все остальные собственные числа $A$ по модулю строго меньше 1. Если вектор $v= \sum c_i e_i$, где $e_i$ жорданов базис, то $$\lim_{n \to \infty}A^nv= c_1 e_1.$$
\ecrl

\rm Если мы находимся над комплексными или вещественными числами, то можно определить не только многочлен от оператора. Пусть функция $$f(x)=a_0 +a_1 x+ \dots + a_n x^n + \dots$$ равна сумме степенного ряда на шаре радису $\rho$ с центром в нуле. Тогда можно посчитать $f(L)$ как предел частичных сумм этого ряда. Это корректно если все собственные числа $|\lambda_i|< \rho$. В этом случае матрицу $f(L)$ можно посчитать следующим аналогичным образом
$$f(A)= C f(J) C^{-1}.$$
Кроме того, есть возможность избежать вычисления самой жордановой формы, а обойтись вычислением характеристического многочлена.
\erm





\chapter{Полилинейная алгебра}

\section{Билинейные и квадратичные формы}

Одним из основных примеров нормы на векторном пространстве над $\mb R$ или над $\mb C$ является корень из суммы квадратов координат $\|x\|=\sqrt{|x_1|^2+\dots +|x_n|^2}$. Однако вместе с такой нормой <<в комплекте>> идёт отображение от двух переменных $(x,y) \to \sum x_iy_i$, которое называется скалярным произведением. Оно обладает свойством билинейности -- то есть линейности по каждой координате. Аналогичным свойством обладает определитель. Наша задача разработать единый подход к таким объектам над произвольным полем и разобраться с самыми важными примерами.  
Всюду далее -- $K$ -- поле характеристики, отличной от 2.

\dfn Пусть $V$ -- векторное пространство над $K$. Отображение $h\colon V\times V \to K$ называется билинейной формой, если\\
1) $\forall \lambda \in K$ $\forall u,v,w \in V$ верно, что $h(u+\lambda v, w) = h(u,w)+\lambda h(v,w)$,\\
2) и по второй координате: $h( w, u+\lambda v) = h(w,u)+\lambda h(w,v)$.
\edfn

\exm\\
1) Рассмотрим пространство $K^n$ и определим на нём билинейную форму $h(u,v)=\sum_{i=1}^n x_iy_i$.\\
2) Рассмотрим пространство многочленов $\mb R[x]$ или пространство непрерывных функций на отрезке $[a,b]$, а так же какую-нибудь непрерывную функцию $w(x)$ и зададим билинейную форму $h(f,g)=\int_a^b f(x)g(x)w(x)dx$.\\
3) Рассмотрим пространство один раз непрерывно дифференцируемых функций на отрезке $C^1[a,b]$ и введём на нём билинейную форму по правилу $h(x,y)=\int_a^b f'(x)g(x)dx$.\\
4) Рассмотрим пространство матриц $M_n(K)$ и введём на нём билинейную форму $h(A,B)= \Tr(AB)$.\\
5) Рассмотрим конечномерную алгебру $A$ над полем $K$. Тогда любой элемент $a\in A$ задаёт линейное отображение $L_a \colon A \to A$, переводящее $x\to ax$. У этого линейного отображения есть след. Для простоты обозначим его как $\tr a$.
Тогда отображение $A\times A \to K$ заданное по правилу $\tr_{A/K}(u,v)=\tr(uv)$ является билинейной формой. Замечу, что конструкция из предыдущего пункта не является частным случаем этой, а отличается на константу.\\

Как обычно, поведение объекта линейной алгебры определяется его взаимодействием с каким-либо базисом.

\dfn Пусть $e_1, \dots, e_n$ базис $V$, а $h$ -- билинейная форма на $V$. Тогда матрица $A$ составленная из элементов $h(e_i,e_j)$ называется матрицей билинейной формы.
\edfn

\lm Пусть $V$ -- пространство с  базисом $e_1,\dots,e_n$. Тогда имеет место взаимооднозначное соответствие между билинейными формами $h$ на $V$ и матрицами  $A\in M_n(K)$. В частности, если вектор $v$ имеет столбец координат $x$, а вектор $u$ -- столбец $y$, то значение $h(u,v)$ можно найти по формуле $y^{\top}Ax$.
\elm

\lm Пусть $V$ -- пространство с билинейной формой $h$ и базисом $e_1,\dots,e_n$. Пусть матрица $h$ в этом базисе -- это $A$. Если выбрать другой базис $f$ с матрицей перехода $C$, то в новом базисе матрица $A$ будет иметь вид 
$$A'=C^{\top}AC.$$
\elm

Есть ещё один способ посмотреть на билинейную форму на пространстве, который в некоторый момент окажется нам полезным.


\begin{defn} Пусть $V$ векторное пространство над полем $K$. Обозначим за $V^{*}=\Hom(V,K)$. Это пространство называется пространством линейных функционалов на $V$ или просто двойственным пространством к $V$. 
\end{defn}




\dfn Пусть $e_1,\dots,e_n$ -- базис $V$. Определим $e^1,\dots, e^n$ -- базис $V^*$, как набор координатных функций базиса $e_i$. Такой базис называется двойственным к базису $e_1,\dots,e_n$. 
\edfn

\rm Двойственный базис однозначно задаётся условием $e^i(e_j)=\delta_{ij}$.
\erm

\exm\\
0) Двойственное пространство к пространству столбцов естественным образом отождествляется с пространством строк $M_{1\times n}(K)$. Двойственный базис к стандартному базису состоит из функционалов, соответствующих строчкам вида $(0,\dots, 1, \dots, 0)$.\\ 
1) Пусть $1,x,\dots, x^n$ -- это базис пространства $K[x]_{\leq n}$. Тогда двойственный базис состоит из линейных функционалов $f(x)\to \frac{1}{i!}f^{(i)}(0)$.\\
2) Рассмотрим набор функционалов $f(x)\to f(\lambda_i)$ двойственный базис к ним это многочлены $f_i=\frac{\prod_{j\neq i} (x-x_j)}{\prod_{j\neq i}{(x_i-x_j)}}$.




\thrm Пусть $V$ конечномерное пространство. Тогда $\dim V = \dim V^{*} $ и имеет место следующий естественный изоморфизм $V \to V^{**} $, который переводит $x \to (f\to f(x))$.
\ethrm
\proof Покажем, что указанное отображение инъективно. Действительно, если $x\neq 0$, то есть такой функционал $f$, что $f(x)=1$. 
Теперь по принципу Дирихле указанное отображение есть изоморфизм.
\endproof


Теперь легко доказать следующую лемму, которая понадобится нам в дальнейшем.

\lm Если набор линейных функционалов $f_1,\dots,f_d$ на пространстве $V$  независим, то отображение $F\colon V\to K^d$, заданное правилом $x \to (f_1(x),\dots, f_d(x))^{\top}$ сюръективно
\proof Дополним линейные функционалы $f_1,\dots,f_d$ до базиса $f_1,\dots,f_n$ пространства $V^*$ и рассмотрим двойственный базис $f^1,\dots,f^n$ в пространстве ${V^*}^*=V$. Тогда $F(f^i)$ для $i\in\ovl{1,d}$ даёт $i$-ый стандартный базисный в $K^d$, что и доказывает сюръективность.
\endproof
\elm


А каким же образом можно отождествить пространство $V$ и пространство $V^*$?
Заметим, что если есть билинейная форма $h$ на $V$, то она задаёт линейное отображение $h\colon V \to V^*$ по правилу $x \to (y \to h(y,x))$. 

\lm Если $e_1,\dots, e_n$ базис $V$, то матрица отображения $h\colon V \to V^*$ в базисе $e$ и двойственном базисе $e^{\vee}$ просто равна матрице $h$, как билинейной формы.
\elm

Указанное соответствие согласуется с принципом, что любая матрица это матрица какого-то линейного отображения.


\dfn Правым (соответственно левым) ядром билинейной формы $h$ на $V$ называется подпространство $$\Ker h= V^{\bot}=\{u \in V\,| \,h(\_,u)=0\}$$
соответственно ${}^{\bot}V=\{u \in V\,| \, h(u,\_)=0\}$. Форма $h$ называется невырожденной, если $\forall x \neq 0$ существует $y \in V$, что $h(y,x)\neq 0$.
\edfn


\rm Вообще говоря, правое и левое ядро у формы $h$ не всегда совпадают.
\erm

\lm Пусть $h$ -- билинейная форма  на конечномерном пространстве $V$. Тогда следующие числа равны 
$$\dim V^{\bot}= \dim {}^{\bot}V=n -\rank A,$$
где $A$ -- матрица $h$ в некотором базисе.
\elm

\dfn Ранг билинейной формы -- это ранг её матрицы.
\edfn

\crl Пусть $h$ -- билинейная форма  на конечномерном пространстве $V$. Тогда следующие условия эквивалентны:\\
1) $h$ невырождена.\\
2) $h$ задаёт изоморфизм $V$ и $V^*$.\\
3) Матрица $h$ в некотором базисе невырождена.\\
4) ${}^{\bot} V =\{0\}$\\
5) $V^{\bot}=\{0\}$.
\ecrl






Наибольший интерес среди билинейных форм вызывают формы со специальными свойствами:

\dfn Билинейная форма $h$ называется симметричной, если $h(u,v)=h(v,u)$. Форма $h$ называется кососимметричной, если $h(u,v)=-h(v,u)$.
\edfn

\lm  Форма $h$ симметрична тогда и только тогда, когда её матрица симметрична, то есть $A^{\top}=A$ и кососимметрична, если $A^{\top}=-A$.
\elm

У нас больше не будет идти речи про билинейные формы общего вида, а только про симметрические или кососимметрические билинейные формы. Они замечательны тем, что понятие левого и правого ортогонала для них совпадают. В частности, совпадают правое и левое ядра билинейной формы. Кроме того изучение билинейной формы можно свести к изучению симметричных и антисимметричных форм благодаря замечанию:

\rm Любая билинейная форма $h$ может быть единственным образом представлена в виде суммы $h^+$ и $h^-$, где $h^+$ -- симметрическая форма, а $h^-$ -- кососимметрическая. ($h^+(u,v)=\frac{h(u,v)+h(v,u)}{2}$, $h^-(u,v)=\frac{h(u,v)-h(v,u)}{2}$)
\erm




Для кососимметрических и билинейных форм понятие левого и правого ядра совпадают. Так же для них удобно работает  следующее определение

\dfn[Ортогонал] Будем говорить, что элемент $x$ ортогонален  элементу $y$, если $h(x,y)=0$ и записывать это как $x\bot y$. 
Если $U$ -- подпространство $V$, то определим множество $$U^{\bot}=\{v\in V\,|\, \forall u \in U \, \, u\bot v\}$$  
\edfn

\rm Если $e_1,\dots,e_k$ базис $U$, то условие $v\in U^{\bot}$ равносильно $e_i\bot v$ по всем $i$.
\erm



\lm Пусть $U$ -- подпространство $V$, то $\dim U^{\bot}\geq \dim V - \dim U$.
Если форма невырождена, то $\dim U^{\bot}= \dim V- \dim U$ и верно равенство ${U^{\bot}}^{\bot}=U$.
\proof Если $e_1,\dots,e_k$ базис $U$, то принадлежность ортогональному дополнению задаётся не более чем $k$ уравнениями, откуда немедленно вытекает неравенство. Теперь покажем , что в случае невырожденности имеет место равенство. Действительно, по условию линейные функционалы $f_i= h(e_i,\_)$ для $e_1,\dots, e_d$ -- базиса $U$, линейно независимы. Тогда пространство $U^{\bot}$ есть ядро сюръективного отображения $F \colon V \to K^d$. По теореме о подсчёте размерности $\dim U^{\bot}= \dim V - d$. 
\endproof 
\elm

Можно поставить вопрос: при каком условии на подпространство $U\leq V$ пространство $V$ раскладывается в виде $U\oplus U^{\bot}$?

\dfn В случае, если пространство $V$ разложилось в виде прямой суммы подпространств $V=U\oplus U'$, таких, что $U \bot U'$, то будем говорить, что имеет место разложение в ортогональную сумму подпространств  $V=U\oplus^{\bot} U'$.
\edfn

\lm Пусть $U\leq V$  и $h$ -- невырожденная билинейная форма на $V$ тогда $V=U\oplus^{\bot} U^{\bot}$ тогда и только тогда, когда $h|_{U}$ невырождена.
\proof Заметим, что $\dim U + \dim U^{\bot}= \dim V$ благодаря невырожденности $h$. Теперь проверим, что $U \cap U^{\bot}= \{0\}$. Пусть это не так и есть такой $0\neq x \in U$, что $x \in U^{\bot}$. Но тогда $\forall y \in U$ $h(x,y)=0$, то есть $x\in \ker h|_{U}$, что противоречит невырожденности ограничения. 

Обратно, если $V=U \oplus U^{\bot}$, то то  $x\in \ker h|_{U}$ лежит одновременно в $U$ и в $U^{\bot}$, что противоречит определению прямой суммы.
\endproof
\elm

\rm Таким образом если форма $h$ невырождена, то для данного подпространства $U$ может найтись не более одного пространства $U'$, что $V=U\oplus^{\bot} U'$ -- ортогональная прямая сумма. А именно, $U'=U^{\bot}$.
\proof Если сумма ортогональная, то $U' \leq U^{\bot}$. Осталось сравнить размерности. 
\endproof
\erm



\section{Метод Лагранжа, критерий Сильвестра}


В этом разделе речь пойдёт про симметрические формы. В начале мы дадим для них альтернативное описание.

\dfn Квадратичная форма -- это отображение $q\colon V \to K$, такое, что в некоторой линейной системе координат это отображение есть однородный многочлен степени 2, то есть имеет вид $\sum_{i\leq j} b_{ij} x_i x_j$. Матрицей квадратичной формы в указанной системе координат называется матрица $$a_{ij}=\begin{cases} b_{ii}, \text{ если $i=j$},\\
\frac{b_{ij}}{2}, \text{ если $i\neq j$}.
\end{cases}.$$
Если вектор $v$ имеет столбец координат $x$, то $q(v)=x^{\top}Ax$.
\edfn

\rm Матрица $A$ единственная симметричная матрица, что $q(v)=x^{\top} A x$. Действительно, единственный способ решить уравнения $b_{ij}=a_{ij}+a_{ji}$ и $a_{ij}=a_{ji}$  есть $a_{ij}=a_{ji}=\frac{b_{ij}}{2}$.
Таким образом, при выборе базиса возникает взаимооднозначное соответствие 
$$\text{ Симметричные билинейные формы } \leftrightarrow \text{ Симметричные матрицы } \leftrightarrow \text{ Квадратичные формы } $$
Покажем, что соответствие симметричных билинейных и квадратичных форм не зависит от базиса.
\erm

\lm Пусть $h$ -- билинейная симметричная форма на $V$. Тогда $q(v)=h(v,v)$ -- это квадратичная форма. При этом, в любой системе координат матрица $q$ есть $A$ -- матрица $h$.
\proof $q(v)=h(v,v)= x^{\top}Ax$. Матрица $A$ -- симметричная и, следовательно, есть матрица $q$.
\elm




\rm
Пусть $q$ -- квадратичная форма. Тогда форма $h(u,v)=\frac{q(u+v)-q(u)-q(v)}{2}$ -- симметричная билинейная. Эта конструкция обратна к конструкции из предыдущего факта. В этом случае форма $h$ называется поляризацией квадратичной формы $q$.
\erm 


\dfn Квадратичная форма невырождена, если соответствующая ей симметричная билинейная форма невырождена.
\edfn

\dfn Пусть $h$ -- симметричная билинейная форма на $V$, тогда система векторов $e_1,\dots,e_k$ называется ортогональной, если $h(e_i,e_j)=0$, при $i\neq j$. Если указанная система векторов является базисом, то такой базис называют ортогональным.
\edfn

\rm Матрица билинейной формы в ортогональном базисе имеет диагональный вид, а выражение для квадратичной формы даёт сумму квадратов с коэффициентами $\sum \lambda_i x_i^2$.
\erm

\dfn Будем говорить, что симметрические билинейные или квадратичные формы эквивалентны, если в некоторых базисах они имеют одинаковые матрицы.
\edfn

Вопрос: всегда ли можно найти ортогональный базис и насколько форма матрицы зависит от выбора ортогонального базиса? 

На первый вопрос ответ положительный.

\thrm Пусть $V$ -- пространство с симметричной билинейной формой $h$. Тогда в $V$ существует ортогональный относительно $h$ базис. 
\ethrm
\proof
Пусть $h$ -- не ноль. Тогда существует вектор $e_1$, что $h(e_1,e_1)=q(e_1)\neq 0$, потому что форма $q$ не нулевая. Действительно, если $h(e_1,e_1)=0$ для какого-то $e_1 \neq 0$, то есть $e_2$, что $h(e_1,e_2)\neq 0$. Тогда либо $h(e_2,e_2)\neq 0$, либо $h(e_1+e_2,e_1+e_2)=2h(e_1,e_2)\neq 0$. 
Теперь $h|_{\lan e_1\ran}$ невырождена и следовательно $V=\lan e_1 \ran \oplus \lan e_1 \ran^{\bot}$. Далее индукция.
\endproof







Обсудим алгоритм который стоит за этим доказательством


Нам удобнее всего будет работать с формой $q$ и представлять её в виде однородного многочлена второй степени. После чего описанный алгоритм можно будет условно назвать выделением полного квадрата.

Пусть форма $q(x)$ в координатах имеет вид
$$q(x)= a_{11}x_1^2+ 2a_{12}x_1x_2 + \dots + 2a_{1n}x_1x_n  + q'(x_2, \dots, x_n).$$

\noindent{\bf Первый случай} Предположим, что $a_{11}\neq 0$. Тогда представим $q(x)$ в виде, выделив полный квадрат 
$$q(x)= a_{11}\left(x_1+\frac{a_{12}}{a_{11}}x_2 + \dots +\frac{a_{1n}}{a_{11}}x_n\right)^2 - \frac{a_{12}^2}{a_{11}}x_2^2 - 2\frac{a_{12}a_{13}}{a_{11}}x_2x_3 - \cdots - \frac{a_{1n}^2}{a_{11}}x_n^2 + q'(x_2,\dots,x_n).$$

Новые переменные выглядят следующим образом:
\begin{align*}
y_1&=x_1+\frac{a_{12}}{a_{11}}x_2 + \dots +\frac{a_{1n}}{a_{11}}x_n,\\
 y_2&=x_2, \\
&\vdots\\
 y_n&=x_n.
\end{align*}
Или в подходящую сторону
$$ \pmat x_1 \\ x_2 \\ \vdots \\ x_n \epmat = \pmat 1 & -\frac{a_{12}}{a_{11}} & \dots & -\frac{a_{1n}}{a_{11}} \\
& 1 && \\
& & \ddots & \\
&&& 1
\epmat \pmat y_1 \\ y_2 \\ \vdots \\ y_n \epmat.
$$

Видно, что кроме формы $q'$ возникает ещё поправка, которая содержит слагаемые $\lambda x_ix_j$ $i,j\geq 2$. Таким образом мы обнулили $a_{1j}$ и сделали первый вектор новой системы координат ортогональным остальным, как и в доказательстве. Заметим так же, что указанное преобразование над матрицей эквивалентно одновременному применению одинаковых элементарных преобразований строк и столбцов.\\




\noindent{\bf Второй случай.} $a_{11}=0$. Если $a_{ii}\neq 0$, то меняем первую и $i$-ую координаты местами  и продолжаем как раньше. \\


\noindent{\bf Третий случай.} Все $a_{ii}=0$.
Пусть $a_{12}\neq 0$. Тогда сделаем замену $x_1=y_1+y_2$, $x_2=y_1-y_2$, $y_i=x_i$, $i\geq 3$. Получим $2a_{12}$ при $y_1^2$ и $-2a_{12}$ при $y_2^2$. Теперь находимся в ситуации первого случая.\\


\noindent{\bf Четвёртый случай.} Все $a_{ii}=0$ и все $a_{1i}=0$. Тогда форма не зависит от первой переменной и можно смело переходить к следующей переменной.\\ 

\dfn Пусть $A$ -- матрица. Числа $d_i=\det A_i$, где $A_i$ -- подматрица $A$ составленная из элементов первых $i$ строк и столбцов  называются главными минорами матрицы $A$.
\edfn


Теперь можно доказать формулу 
\thrm[Теорема Якоби]
Пусть $V$ -- векторное пространство, $q$ -- квадратичная форма, $A$ -- её матрица в некотором базисе $e_1,\dots,e_n$. Пусть главные миноры $d_i$ не равны 0. 
Тогда матрица $A$ -- невырожденная и может быть приведена к диагональному виду с числами $\frac{d_{i}}{d_{i-1}}$ на диагонали.
\proof Заметим, что если мы всё время пользуемся первым случаем из алгоритма, то домножение слева на матрицу перехода $C^{\top}$ будет эквивалентно прибавлению строки ко всем остальным строкам матрицы $A$ c коэффициентами $-\frac{a_{1i}}{a_{11}}$ (аналогично при домножении на $C$ справа происходит преобразование столбцов). Заметим, что при таком преобразовании главные миноры матрицы вообще не меняются (внутри каждого минора происходят преобразования первого типа, которые не меняют определитель). Покажем, что только первый случай реализуется. 

Пусть мы доказали это для шага $i$. На шаге $i+1$ первые $i$ столбцов и строк содержат только ненулевые диагональные элементы $a_{11},\dots, a_{ii}$. Тогда $d_{i+1}=a_{11}\dots a_{i+1 i+1}$. Так как $d_{i+1}$ не  поменялось и, следовательно не равно нулю, то и $a_{i+1 i+1} \neq 0$. Следовательно реализуется первый случай. 

Теперь посмотрим, что происходит, после приведения матрицы к диагональному виду. Заметим, что для всех $i$ $d_i=a_{11}\dots a_{ii}$. Тогда $a_{ii}=\frac{d_i}{d_{i-1}}$, что и требовалось.
\endproof
\ethrm




Насколько диагональный вид для квадратичной формы единственен? Сразу же можно отметить очевидный инвариант -- а именно $\rank A$, который не меняется при замене базиса и в диагональном виде равен числу нулей на диагонали. Но вообще говоря канонический вид не единственен. Прежде всего заметим, что форма 
$$q(x)=\lambda_1 x_1^2+\dots+\lambda_nx_n^2$$
при замене $x_i=c_i y_i$, $c_i\neq 0$ остаётся диагональной, но приобретает вид 
$$q(y)=c_1^2\lambda_1y_1^2+\dots+c_n^2\lambda_ny_n^2.$$
Итого, все коэффициенты можно свободно поменять на произвольный ненулевой квадрат. Но и это не даёт полной характеризации. Действительно  рассмотрим квадратичную форму
$$3x^2+2y^2$$
и соответствующую билинейную $3x_1x_2+2y_1y_2$ и посмотрим на неё в базисе $(1,1)^{\top}, (-2,3)^{\top}$. В новом базисе форма имеет вид
$$5z_1^2+30z_2^2.$$
Возникает вопрос -- так может они вообще все эквивалентны? Заметим, что если рассмотреть невырожденную форму $h$, то определитель её матрицы $A$ вообще говоря меняется при замене базиса. Однако 
$$\det(C^{\top}AC)=\det A (\det C)^2.$$
Таким образом определитель матрицы квадратичной формы меняется при переходе от одного базиса к другому на квадрат обратимого элемента. Это приводит нас к определению:

\dfn Дискриминантом квадратичной формы $q$ называется $\det A \in K^*/(K^*)^2$, где $A$ -- матрица $q$.
\edfn

Дискриминант формы уже довольно сильный инвариант, например над полем $\mb Q$.

\zd Рассмотрим алгебру над $\mb Q$ вида $A=\mb Q[\sqrt{d}]= \mb Q[x]/(x^2-d)$, где $d$ целое число свободное от квадратов. Посчитайте дискриминант формы следа $\tr_{A/\mb Q}$ и покажите, что при разных $d$ алгебры $\mb Q[\sqrt{d}]$ не изоморфны.
\ezd

Однако и в рациональном случае дискриминант не есть полный инвариант формы. Например, вопрос классификации квадратичных форм приводит к содержательным фактам из теории чисел, например, как квадратичный закон взаимности и его обобщения.

Однако есть ситуации, где мы можем полностью описать ответ.
Проще всего это сделать над алгебраически замкнутым полем $\mb C$. Над $\mb C$ любая квадратичная форма определяется своим рангом, так как приводится к виду $$q(x)=x_1^2+\dots+x_r^2.$$


Рассмотрим поле вещественных чисел $\mb R$. К какому виду можно привести форму над $\mb R$? 

\crl Пусть $q$ -- квадратичная форма на вещественном векторном пространстве $V$. Тогда существует линейная система координат в которой форма имеет вид $$q(x)= x_1^2+\dots + x_k^2 - x_{k+1}^2-\dots-x_{l+k}^2.$$
Такой вид квадратичной формы будем называть каноническим. 
\ecrl

\dfn Сигнатурой формы над $\mb R$ называется разность пара чисел $(k, l)$ -- числа плюсов и числа минусов в каноническом виде. Заметим, что сумма $l+k= \rank q$.
\edfn

\dfn Квадратичная форма называется положительно определённой, если $\forall v\neq 0$ $q(v)>0$. Симметричная билинейная форма называется положительно определённой, если соответствующая форма $q(v)=h(v,v)$ положительно определена. Симметричная матрица называется положительно определённой, если соответствующая форма положительно определена.
\edfn



\thrm Сигнатура формы $q$ не зависит от способа приведения формы к каноническому виду. Точнее -- число $k$ равно размерности наибольшего подпространства, ограничение формы  на которое положительно определено.
\proof Рассмотрим базис $e_1,\dots,e_k,e_{k+1},\dots,e_{l+k}, \dots, e_n$, что матрица формы $q$ диагональна, и первые $k$ её диагональных компонент положительны, следующие $l$ отрицательны, а остальные 0. 
Пусть $U$ подпространство $\dim U \geq k+1$, что $q|_{U}>0$. Тогда исходя из подсчёта размерности $U\cap \lan e_{k+1},\dots,e_n\ran \neq \{0\}$. Но это приводит к противоречию, так как $q(v)$ для $0\neq v \in U\cap \lan e_{k+1},\dots,e_n\ran $ выполнено, что $q(v)>0$ и $q(v)\leq 0$ одновременно.
\endproof
\ethrm

\crl Пусть $q$ -- форма на вещественном пространстве  $V$ размерности $n$. Тогда канонический вид $q$ однозначно определяется $n$ и её  сигнатурой. 
\ecrl

Можно ли как-то ещё найти сигнатуру не приводя форму к диагональному виду, а воспользовавшись другими знаниями? Ответ: да, можно. А именно

\crl[Критерий Сильвестра]
Пусть $V$ -- векторное пространство над $\mb R$, $q$ -- квадратичная форма, $A$ -- её матрица в некотором базисе $e_1,\dots,e_n$. Пусть главные миноры  $d_i$ матрицы $A$ все не равны $0$. Тогда число перемен знака в последовательности $d_i$ равно числу отрицательных квадратов в нормальном виде.
\proof По теореме Якоби существует система координат в которой форма имеет диагональную матрицу с числами $\lambda_i=\frac{d_i}{d_{i-1}}$ на диагонали. Тогда последовательность $\delta_i$ меняет знак тогда и только тогда, когда $\lambda_i<0$.
\endproof
\ecrl

Поговорим теперь про частный случай положительно определённых форм


\lm Положительно определённая форма всегда невырождена.
\proof $h(x,x)>0$ и поэтому не равно 0.
\endproof
\elm

\thrm
Пусть дана форма $q$ на вещественном пространстве $V$ и её матрица $A$ в некотором базисе. Следующие условия эквивалентны:\\
1) Форма $q$ положительно определена.\\
2) Матрица $A$ представима в виде $A=C^{\top}C$ для некоторой невырожденной матрицы $C$.\\
3) Главные миноры матрицы $A$ положительны.
\proof
Если форма $q>0$, то она невырождена и имеет сигнатуру $(n,0)$. Тогда есть обратимая матрица $C$,  что $E_n=C^{\top}AC$. Тогда $A={C^{-1}}^{\top}C^{-1}$. Обратно, пусть $A=C^{\top}C$, тогда если $x^{\top}Ax=(Cx)^{\top}Cx\geq 0$. Более того, это выражение равно нулю только если $Cx=0$. Но такое возможно, только если $x=0$.


Пусть теперь $d_i>0$, тогда $q>0$ по критерию Сильвестра. Обратно, если $q>0$, то $q|_{\lan e_1,\dots,e_k\ran} >0$. Но определитель матрицы этой формы и есть главный минор порядка $k$. Отсюда $d_k>0$ как определитель произведения $\det C^{\top}C=\det^2 C$.
\endproof
\ethrm

Критерий Сильвестра пригоден, в случае если имеется задача с параметром. Иначе, вычисление определителя практически эквивалентно приведению к диагональному виду. 

Покажем, как можно применять понятие положительной определённости. Точнее, того, что положительная определённость гарантирует невырожденность.


{\bf Задача:}\\ Рассмотрим множество $\left\{ 1,\dots, n\right\}$. Сколько может быть различных подмножеств $C_1,\dots,C_m$, таких, что $|C_i \cap C_j|=t$ одинаково для всех $i \neq j$?

Мы покажем, что есть ограничение $m\leq n$. Прежде всего, это надо сделать в случае, если $|C_i|=t$ для некоторого $i$. В этом случае $C_i \subseteq C_j$ для всех $j$. Тогда $C_j'=C_j\setminus C_i$ лежат в множестве $\{1,\dots,n\}\setminus C_i$, которое состоит из $n-t$ элементов. Множества $C_j'$ не пусты, но имеют пустое пересечение друг с другом. Тогда их меньше чем $n-t$ штук, откуда -- каждое множество должно содержать уникальный элемент. Итого
$$m-1\leq n-t \leq n-1 \text{ или, по-другому, } m \leq n.$$

Теперь покажем, что в ситуации $d_i=|C_i|>t$ выполнено то же неравенство. Для этого составим матрицу инцидентности
$$B_{ij}= \begin{cases} 1, \text{ если } i\in C_j. \\
0, \text{ иначе}
\end{cases}. $$
Теперь матрица $B^{\top}B$ есть квадратная симметричная матрица размера $m$. Я утверждаю, что $B^{\top}B$ положительно определена. Для этого найдём её явно.
$$B^{\top}B= \pmat
d_1 & t &t\\ 
t & \ddots & t \\
t & t & d_n
\epmat = \pmat
d_1-t &  &0\\ 
 & \ddots &  \\
0 &  & d_n-t
\epmat + \pmat
t & \dots & t\\ 
\vdots & \ddots & \vdots \\
t & \dots & t
\epmat.$$
Тогда при $x\neq 0$ имеем 
$$x^{\top} B^{\top}B x = \sum (d_i-t) x_i^2 + t(\sum x_i)^2>0.$$
Теперь 
$$m=\rank B^{\top}B \leq \rank B \leq n.$$ 



\begin{comment} 
\section{Полуторалинейные формы}
Однако кроме билинейных форм интерес представляют так называемые полуторалинейные формы.

\dfn Пусть $K$ -- поле. Тогда инволюцией на $K$ называется изоморфизм $\sigma \colon K \to K$, такой, что $\sigma^2=\id_K$.
\edfn

\dfn Пусть $K$ -- поле c инволюцией $\sigma \colon K \to K$, а $V$ -- векторное пространство над $K$. Отображение $h\colon V \times V \to K$ называется полуторалинейным, если \\
1) $h(x,y+\lambda z)=h(x,y)+\lambda h(x,z)$. \\
2) $h(x+\lambda y,z)=\sigma(\lambda)h(x,z)+h(y,z)$.
\edfn

\exm\\
1) Основным примером поля $K$ c инволюцией будет служить $\mb C$ и автоморфизм сопряжения. А основным примером полуторалинейной формы на $\mb C^n$ будет форма $(x,y)\to \sum \ovl{x_i}y_i$\\
2) Рассмотрим пространство комплекснозначных непрерывных функций на отрезке $C([a,b])$. Определим полуторалинейную форму по следующему правилу:
$$h(f,g)=\int_a^b \ovl{f(x)}g(x)w(x)dx,$$
где $w(x)$ -- непрерывная на $[a,b]$ комплекснозначная функция, которая обычно называется весом.\\
3) Самый общий вид полуторалинейной формы на пространстве $\mb C^n$ имеет вид $\ovl{x^{\top}}Ay$ для некоторой матрицы $A\in M_n(\mb C)$.\\
4) Пусть $K$ произвольное поле, а $\sigma=\id$. Тогда понятие билинейности  и полуторалинейности совпадают.\\


Приведём основные определения и утверждения про полуторалинейные формы, копирующие ситуация с билинейными формами. Доказательства опустим.

\dfn Матрицей полуторалинейной формы $h$
в базисе $e$ называется матрица $a_{ij}=h(e_i,e_j)$. 
\edfn






\lm Если $x$ и $y$ координаты векторов $u$ и $v$, то $$h(u,v)=\ovl{x^{\top}}Ay.$$
\elm

\dfn[Левый и правый ортогонал, левое и правое ядро, невырожденность]
\edfn

\lm Размерности левых и правых ядер совпадают и равны $n -\rank A$. Форма $h$ невырождена тогда и только тогда, когда её матрица $A$ невырождена.
\elm



\lm Если форма $h$ невырождена, то размерность $\dim U^{\bot}= n - \dim U
$
\elm

\dfn Полуторалинейная форма $h$ называется эрмитовой, если $h(u,v)=\ovl{h(v,u)}$ и косоэрмитовой, если $h(u,v)=-\ovl{h(v,u)}$.
\edfn

\rm Полуторалинейная форма над $\mb C$ эрмитова тогда и только тогда, когда её матрица $A$ в некотором базисе удовлетворяет соотношению $\ovl{A^{\top}}=A$. В случае тождественного автоморфизма эрмитовость эквивалентна симметричности.
\erm
\end{comment}




\section{Евклидовы и унитарные пространства}

Напомню, что основной нашей мотивацией для изучения билинейных форм было, то, что вместе с понятием расстояния часто идёт вместе некоторая билинейная форма.





\dfn Векторное пространство $V$ над $\mb R$ вместе с заданным на нём билинейной симметричной формой $\lan\cdot \, , \cdot \ran$ называется евклидовым пространством. Форма $\lan\cdot \, , \cdot \ran$ называется скалярным произведением. 
\edfn

\dfn Определим  норму на евклидовом пространстве как $\|v\|=\sqrt{\lan v , v\ran }$. Норма задаёт расстояние по правилу $\rho(u,v)=\|u-v\|$. Будем говорить, что два вектора ортогональны, если $\lan u,v \ran =0$.
\edfn

\lm[Неравенство Коши-Буняковского] В евклидовом пространстве выполнено неравенство
$$ \lan u,v\ran \leq \|u\|\|v\|.$$
\proof  Квадратный трёхчлен $\lan u,u\ran +2\lambda\lan u,v\ran +\lambda^2\lan v,v\ran=\lan u+\lambda v, u+\lambda v\ran \geq 0$ всегда положителен. Значит у него нет корней, то есть дискриминант отрицaтелен. То есть $$4\lan u,v\ran^2 \leq 4 ||u||^2||v||^2.$$
\endproof
\elm

\lm Введённая норма действительно является нормой.
\proof Необходимо показать неравенство $||u+v||\leq ||u||+||v||$. Оно эквивалентно $$||u+v||^2 \leq ||u||^2+||v||^2+2||u||||v||$$
Расписывая левую часть получаем эквивалентное
$$ ||u||^2+||v||^2+2\lan u,v\ran \leq ||u||^2+||v||^2+2||u||||v||.$$
Сокращая справа и слева приходим к уже известному неравенству.
\endproof
\elm

 
\lm Пусть $V$ -- евклидово пространство. Тогда для всякого подпространства $U$ имеет место ортогональное разложение $V=U\oplus U^{\bot}$. Если есть такое разложение, то оператор проекции на $U$ называется ортогональной проекцией.
\proof Положительно определённая форма невырождена. Ограничение положительно определённой формы на любое подпространство положительно определено.
\endproof
\elm

Мы с вами помним, что в задачах алгебры  удобнее бывает работать над алгебраически замкнутым полем. Однако, если мы дословно перенесём все определения с $\mb R$ на $\mb C$, то нас постигнет неудача. Прежде всего в плане положительной определённости. А именно, любая комплексная квадратичная форма не будет принимать вещественные значения. Это делает невозможным аналогичное вещественному случаю определение расстояния. 

Это приводит нас к тому, что язык билинейных форм не совсем адекватен в комплексной ситуации. С другой стороны у нас есть пример удачного понятия расстояния на $\mb C$, которое задаётся формулой $\sqrt{\ovl{z}z}$. Такой выражение получается не из билинейной формы $xy$ подстановкой $x=y$, а из менее ожидаемого $\ovl{x}y$. Заметим, что и вообще на пространстве $\mb C^n$ можно ввести операцию $
(x,y) \to \sum_{i=1}^n \ovl{x_i}y_i$, которая даст по стандартной схеме норму $\sqrt{\sum_{i=1}^n |x_i|^2}$. Попробуем разобраться в общей ситуации.

\dfn Пусть $V$ -- комплексное пространство.  Отображение $h\colon V \times V \to \mb C$ называется полуторалинейным, если \\
1) $h(x,y+\lambda z)=h(x,y)+\ovl{\lambda} h(x,z)$. \\
2) $h(x+\lambda y,z)=\lambda h(x,z)+h(y,z)$.
\edfn

\exm\\
1) Основным примером полуторалинейной формы на $\mb C^n$ будет форма $(x,y)\to \sum \ovl{x_i}y_i$\\
2) В более общем виде, если взять матрицу $A\in M_n(\mb C)$  то выражение $x^{\top}A\ovl{y}$ задаёт полуторалинейную форму на $\mb C^n$.\\
3) Рассмотрим пространство комплекснозначных непрерывных функций на отрезке $C([a,b])$. Определим полуторалинейную форму по следующему правилу:
$$h(f,g)=\int_a^b \ovl{f(x)}g(x)w(x)dx,$$
где $w(x)$ -- непрерывная на $[a,b]$ комплекснозначная функция, которая обычно называется весом.\\



\dfn Матрицей полуторалинейной формы $h$
в базисе $e$ называется матрица $a_{ij}=h(e_i,e_j)$. 
\edfn

\lm Если $x$ и $y$ координаты векторов $u$ и $v$, то $$h(u,v)=\ovl{x^{\top}}Ay$$
и обратно, если $$h(u,v)=\ovl{x^{\top}}Ay,$$
то $A$ -- это матрица $h$.
\elm



Заметим, что понятия симметричности  от таких форм не приходится ожидать.

Действительно, если $\lambda h(u,v)=h(u,\lambda v) = h(\lambda v,u)=\ovl{\lambda}h(v,u)=\ovl{\lambda}h(u,v)$ для всех $\lambda$. Тогда, конечно, $h(u,v)=0$. Однако приведённые примеры подсказывают нам необходимое свойство.

\dfn Полуторалинейная форма $h$ называется эрмитовой, если $h(u,v)=\ovl{h(v,u)}$ и косоэрмитовой, если $h(u,v)=-\ovl{h(v,u)}$
\edfn

\lm Полуторалинейная форма эрмитова тогда и только тогда, когда её матрица $A$ удовлетворяет соотношению $\ovl{A^{\top}}=A$ и косоэрмитовой, если $\ovl{A^{\top}}=-A$.
\elm

\rm  Заметим, что если матрица была вещественными коэффициентами, то это просто условия  симметричности и кососимметричности.
\erm

\dfn Эрмитова форма называется положительно определённой, если для всех $0\neq v\in V$ $h(v,v)>0$.
\edfn

\lm Матрица положительно определённой эрмитовой формы невырождена.
\proof Если есть вектор $0\neq x\in \Ker A$, то $0<x^{\top}Ax = 0$.
\endproof
\elm

\dfn Пространство $V$ над $\mb C$ вместе с положительно определённой эрмитовой формой $\lan \cdot, \cdot \ran$ называется унитарным пространством. Форма $\lan \cdot, \cdot \ran$ называется скалярным произведением. 
\edfn

Попробуем доказать аналог неравенства Коши-Буняковского для унитарного пространства $V$ и вывести из него, что отображение $\|\cdot\| \colon V \to \mb R$, заданное по правилу $v\to \sqrt{\lan v,v\ran}$ так же как и в вещественном случае задаёт  норму на $V$.

\lm Пусть $u,v \in V$ -- два вектора в унитарном пространстве. Тогда имеет место неравенство:
$$|\lan u,v\ran| \leq \|u\| \|v\|.$$
\proof Как и раньше  $\lan u+\lambda v, u+\lambda v\ran \geq 0$ всегда положителен для всех $\lambda \in \mb R$. Раскрывая скобки получаем однако $\lan u,u\ran +2\lambda \Re\lan u,v\ran +\lambda^2\lan v,v\ran\geq 0$.  Итого имеем  $$4\Re\lan u,v\ran^2 \leq 4 ||u||^2||v||^2.$$
Пусть $\lan u ,v \ran = r e^{i\ffi}$. Тогда $\Re \lan  e^{i\ffi}u , v \ran= e^{-i\ffi} r e^{i\ffi}=r=|\lan u,v\ran|$, а $||e^{i\ffi}u||=||u||$. Применение предыдущего неравенства  к паре $e^{i\ffi} u, v$ доказывает общее неравенство.
\endproof
\elm

\crl Отображение $\|\cdot\| \colon V \to \mb R$, заданное по правилу $v\to \sqrt{\lan v,v\ran}$ задаёт  норму на $V$.
\ecrl


Так же полезно будет проинтерпретировать геометрическое понятие угла. Прежде всего разберёмся с углом между двумя векторами.

\dfn Пусть $x,y\neq 0$ два вектора в $V$. Если $V$ -- евклидово, то углом между ними называется такое число $0\leq\ffi\leq \pi$, что 
$$\cos\ffi = \lan x,y\ran.$$
В случае унитарного пространства $V$ угол $\ffi$ может принимать значения $0\leq \ffi \leq \frac{\pi}{2}$ и определяется соотношением
$$\cos\ffi = |\lan x,y\ran|.$$
\edfn

Оба определения корректны благодаря неравенству Коши-Буняковского. По каждому из этих определений угол между векторами равен $\frac{\pi}{2}$ тогда и только тогда, когда $\lan x,y 
\ran=0$, то есть когда векторы ортогональны. 

В дальнейшем мы будем обсуждать  евклидовы и унитарные пространства. Свойства евклидовых и унитарных пространств похожи, поэтому  мы будем стараться доказывать общие утверждения в обоих случаях.


\section{Ортогонализация Грама-Шмидта}

В случае евклидовых пространств ограничение положительно определённой формы на любое подпространство невырождено и, таким образом, нахождение базиса, в котором форма имеет канонический вид несколько облегчается.  Это приводит к тому приводит к тому, что мы можем уточнить сам результат и немного изменить алгоритм нахождения подходящего базиса. Так же мы покажем, что ровно тот же алгоритм работает в унитарных пространствах и позволяет найти базис, что матрица скалярного произведения в этом базисе диагональна или даже единична.

Итак пусть дан набор векторов $e_1,\dots, e_n $ евклидового или унитарного пространства $V$. Ортогонализацией набора $e_1,\dots,e_n$ называется  новый набор векторов $f_1,\dots,f_n$ такой, что\\
1) $f_i \bot f_j$, если $i\neq j$\\
2) $\|f_i\|=1$.\\
3) $\lan e_1,\dots,e_k\ran=\lan f_1,\dots,f_k\ran$\\


\dfn Набор векторов со свойством 2)
называется нормированным. со свойствами 1),2) -- ортонормированным. Если этот набор базис, то он называется ортогональным или ортонормированным базисом соответственно.
\edfn 

\rm Заметим, что если мы нашли набор ненулевых векторов со свойствами 1) и 3), то несложно сделать из него нормированный набор, взяв вектора $\frac{f_i}{\|f_i\|}$. 
\erm

\thrm Пусть $V$ -- евклидово или унитарное пространство. Задача ортогонализации разрешима для линейно независимого набора векторов из $V$.
\proof

Перейдём к решению задачи добиваясь только условий 1) и 2). Будем последовательно искать вектора $f_i$ в виде $f_i=e_i+\lambda_1 f_1 +\dots + \lambda_{i-1} f_{i-1}$. Этот подход приводит к ответу
$$f_i=e_i-\sum_{j<i} \frac{\lan f_j,e_i\ran}{\lan f_j,f_j\ran}f_j.$$
Так как вектора линейно независимы, то $f_i\neq 0$. Это означает, что можно поделить на его норму и добиться нормированности.
\endproof
\ethrm


Процесс ортогонализации позволяет строить ортогональный базис для различных подпространств.

\crl В евклидовом и унитарном пространстве любой ортонормированный набор векторов можно дополнить до ортонормированного базиса.
\ecrl

\crl Если $U$ подпространство в унитарном пространстве $V$, то $U^{\bot}$ есть подпространство $V$ и  $V= U \oplus U^{\bot}$. 
\proof Пусть $e_1,\dots,e_k$ -- ортонормированный базис $U$, а $e_{k+1}, \dots, e_n$ -- его дополнение до ортонормированного базиса всего пространства. По определению $e_{k+1},\dots,e_n \in U^{\bot}$, откуда $\dim U^{\bot} \geq n-k$. Покажем, что $U\cap U^{\bot}=\{0\}$. Это одновременно покажет нам, что $\dim U^{\bot}=n-k$, так как иначе из формулы Грассмана будет следовать, что пересечение ненулевое.
Если вектор лежит в $U\cap U^{\bot}$, то $\lan x,x\ran=0$, что противоречит положительной определённости.
\endproof
\ecrl

\lm[Теорема Пифагора] Пусть $x= \sum c_i e_i \in V$, где $e_i$ это ортонормированный базис $V$. Тогда $\|x\|^2=\sum |c_i|^2$.
\proof Раскрыть скобки в выражении $\lan x,x\ran$.
\endproof
\elm

\lm Пусть набор $e_i$ --- ортогональный базис $V$. Тогда для любого $x\in V$ 
$$x= \sum \frac{\lan e_i,  x\ran}{\lan e_i, e_i\ran}e_i \text{ и следовательно  } \|x\|^2= \sum \frac{|\lan x, e_i\ran|^2}{\lan e_i, e_i\ran}.$$
 В случае нормированного базиса формула упрощается -- исчезает знаменатель.
\proof Пусть $x=\sum c_i e_i$. Тогда $\lan e_i, x\ran = c_i \lan e_i,e_i \ran$, что и требовалось.
\endproof
\elm



\crl Пусть $ e_1,\dots, e_k$ --- ортогональный базис $U$ -- подпространства $V$. Тогда для проекции $x$ на $U$ имеет место формула:
$$ pr_U(x)= \sum \frac{\lan x,e_i\ran}{\lan e_i,e_i\ran} e_i.$$
Так же, справедливо, что
$$||pr_{U^{\bot}(x)}||^2 + ||pr_U x||^2=||x||^2.$$
\proof Рассмотрим базис $e_1,\dots, e_k$ и дополним его векторами $e_{k+1},\dots,e_n$ до ортогонального базиса всего $V$. Тогда $$x= \sum_{i=1}^k \frac{\lan e_i,  x\ran}{\lan e_i, e_i\ran}e_i + \sum_{i=k+1}^n \frac{\lan e_i,  x\ran}{\lan e_i, e_i\ran}e_i $$
Первая часть суммы лежит в $U$, а вторая в ортогональном дополнении. По единственности такого разложения получаем требуемое. Осталось применить теорему Пифагора.
\endproof
\ecrl







Для чего может пригодиться понятие ортогональной проекции? Прежде всего для определения расстояния. Расстояния между подмножествами.

\dfn Пусть $A$ и $B$ подмножества метрического пространства. Тогда расстоянием $\rho(A,B)$ положим равным
$$\rho(A,B)=\inf_{\substack{x\in A\\ y \in B}} \rho(x,y).$$
\edfn

Наша задача --- научиться считать расстояние между аффинными подпространствами $A_1$ и $A_2$ в евклидовом пространстве. Попробуем это сделать. Представим $A_1=L_1+x$ и $A_2=L_2+y$. 

Разберём сначала случай расстояния от точки до линейного подпространства. 

\thrm Пусть $U \leq V$ подпространство, $x \in V$. Тогда расстояние $\rho(x,U)$ достигается на проекции  $pr_U(x)$ и равно $\|x-pr_U(x)\|=\|pr_{U^{\bot}}(x)\|$.
\proof Рассмотрим $u \in U$, и представим $x=y+z$, где $pr_Ux=y \in U$, $pr_{U^{\bot}}x =z \in U^{\bot}$. Тогда $$\rho(x,u)^2= ||x-u||^2= ||y - u ||^2+ ||z||^2 \geq ||z||^2=||pr_{U^{\bot}}x||^2 .$$
С другой стороны равенство достигается при $u=y=pr_{U} x$.
\endproof
\ethrm



\rm Если размерность $U^{\bot}$ мала, то может быть легче найти проекцию на $U^{\bot}$, найдя ортогональный базис $U^{\bot}$.
\erm

Всё это немедленно приводит к решению общей задачи.

\crl Пусть  $A_1=L_1+x$ и $A_2=L_2+y$ -- аффинные подпространства. Тогда $\rho(A_1,A_2)=\rho(y-x, L_1+L_2)$. То есть задача сводится к ранее разобранной.
\proof $$\inf_{\substack{u+x\in L_1+x\\ v+y \in L_2+y}} ||u+x-v-y||=\inf_{\substack{u-v\in L_1+ L_2}}||u-v - (y-x)||=\inf_{\substack{u\in L_1+ L_2}}||u- (y-x)||.$$
\endproof
\ecrl

Мы с вами ввели понятие объёма параллелепипеда с помощью определителя и показали, что оно однозначно определено с точностью до множителя, который фиксируется некоторой нормировкой. Вопрос -- даёт ли структура евклидового пространства возможность зафиксировать эту нормировку? Ответ да, с точностью до знака, то есть, до ориентации пространства.

\dfn Пусть $V$ -- евклидово или унитарное пространство. Тогда базис будем говорить, что форма объёма согласована с метрикой, если объём параллелепипеда, натянутого на некоторый ортогональный базис равен 1.
\edfn

\dfn Пусть $e_1,\dots, e_k$ набор векторов $V$. Тогда матрицей Грамма называется матрица 
$$G(e_1,\dots,e_k)_{ij}= \lan e_i, e_j\ran.$$
Матрица Грамма отличается от матрицы скалярного произведения как би(полутора)линейной формы, только тем, что даёт некоторый ответ в случае зависимого набора векторов. 
\edfn

\lm
Форма объёма принимающая значение $1$ на ортогональном базисе принимает значения $\pm 1$ на всех остальных ортонормированных базисах и более того, имеет место формула
$$(\Vol(v_1,\dots,v_n))^2= \det G(v_1,\dots,v_n)$$
\proof Пусть $e_1,\dots,e_n$ ортогональный базис, в котором форма объёма принимает значение 1. Тогда $\Vol= \Vol_e$. Обозначим за $A$ матрицу $e(v_1),\dots,e(v_n)$. Тогда $$\Vol(v_1,\dots,v_n)^2= \det\nolimits^2 (A)= \det A^{\top} \det A= \det A^{\top}A= \det G(v_1,\dots,v_n)$$
В частности, если $v_i$ ортонормирован, то $G(v_1,\dots,v_n)=E_n$ и $\Vol(v_1,\dots,v_n)^2=1$.
\endproof
\elm

\rm Определитель матрицы Грамма обнуляется тогда и только тогда, когда вектора $v_i$ линейно зависимы.
\erm

\crl Форма объёма, согласованная с метрикой на евклидовом пространстве единственна с точностью до знака. А понятие объёма параллелепипеда -- единственно.
\ecrl


\begin{comment}

Так же можно дать явную формулу для расстояния через определители.
\lm Пусть подпространство $U$ имеет базис $e_1,\dots,e_k$. Тогда $\rho(x_0, U)^2 = \frac{\det G(e_1,\dots,e_k,x_0)}{\det G(e_1,\dots,e_k)}$ 
\elm 



\dfn $V$ --- евклидово пространство $U$ -- его подпространство, а $x_0$ вектор из $V$. Определим косинус угла $\angle x_0,U$ как 
$$\cos \angle x_0,U=\sup_{0\neq y\in U} \cos \angle x_0,y =\sup_{0\neq y\in U} \frac{|\lan x_0,y\ran| }{\|x_0\|\|y\|}.$$  
То есть мы ищем минимальный угол из отрезка $[0,\frac{\pi}{2}]$. Заметим, что в евклидовом случае отрицательные косинусы не имеют смысла, так как всегда можно домножить $y$ на $-1$.
\edfn

\lm Наименьший угол достигается между $x_0$ и его проекцией на $U$ и его косинус равен $\frac{\|pr_{U}x_0\|}{\|x_0\|}$.  
\elm


\zd Докажите аналогичное свойство в унитарном случае.
\ezd


\end{comment}



\section{Ортогональные и унитарные операторы}

Основным отличием структуры евклидового и унитарного пространства от просто векторного пространства является понятие расстояние и поэтому некоторое время мы посвятим преобразованиям это расстояние сохраняющим.

\dfn Пусть $X$ и $Y$ -- метрические пространства. Тогда отображение $f\colon X \to Y$ называется изометрическим вложением, если для всех $x_1,x_2 \in X$ $\rho(x_1,x_2)=\rho(f(x_1),f(x_2))$. Если к тому же $f$ -- биекция, то $f$ называют изометрией.
\edfn

\rm Изометрическое вложение всегда инъективно, так как $f(x)=f(y)$ тогда и только тогда, когда $\rho(f(x),f(y))=0$ что происходит только если $\rho(x,y)=0$, то есть $x=y$.
\erm


\dfn Пусть $V$ --- евклидово пространство. Ортогональным оператором на $V$ называется такой линейный оператор $L \colon V \to V$, что $||Lx||=
||x||$.
\edfn

\thrm Пусть $L\colon V \to V$ -- оператор на евклидовом пространстве. Следующие условия эквивалентны:\\
1) $L$ -- ортогональный оператор.\\
2) $\lan Lx,Ly\ran=\lan x,y \ran$ для всех $x,y \in V$.\\
3) $L$ переводит любой ортонормированный базис в ортонормированный базис.\\
4) В любом ортонормированном базисе $U$ -- матрица $L$ -- удовлетворяет условию $A^{\top} A = E_n$.\\
5) $L$ переводит некоторый ортонормированный базис в ортонормированный базис.\\
6) В некотором ортонормированном базисе $A$ -- матрица $L$ --  удовлетворяет условию $A^{\top} A = E_n$.
\ethrm
\proof
Переход $1\to 2$ верен, так как из равенства квадратичных форм следует равенство билинейных форм. Переход $2\to 3$ ясен.


Покажем, что 3) эквивалентно 4) и 5) эквивалентно 6). Для этого покажем, что если фиксировать ортонормированный базис $e_1,\dots, e_n$, то $L(e_i)$ ортонормирован, тогда и только тогда, когда $A$ -- матрица $L$ удовлетворяет условию $A^{\top}A=E_n$. Для этого необходимо и достаточно заметить, что $$\lan Le_i, Le_j \ran = (A^{\top}A)_{ij}.$$
Действительно $v_j$ -- $j$-ый столбец $A$ составлен из координат $Le_j$. $i$-ая строка $A^{\top}$ равна тогда $v_i^{\top}$. Но тогда $$\lan Le_i,Le_j\ran = v_i^{\top}E_n v_j= v_i^{\top}v_j= (A^{\top}A)_{ij}.$$
Во втором равенстве $E_n$ играет роль матрицы Грама ортонормированного базиса $e_1,\dots, e_n$.
Итак $3 \leftrightarrow 4$ и $5 \leftrightarrow 6$. Из 3) очевидно следует 5).

Покажем $6 \to 1$. Пусть $e_1,\dots,e_n$ ортонормированный  базис, в котором имеет место свойство  $A^{\top} A=E_n$. Тогда 
$$||Lx||^2=e(x)^{\top} A^{\top} A e(x)= e(x)^{\top} e(x)= ||x||^2.$$




Если отображение $L$ сохраняет норму, то оно сохраняет и расстояние, следовательно, получившееся отображение изометрия из $V$ в себя. Оно является автоморфизмом $V$.
С другой стороны, в определении мы видим равенство двух квадратичных форм $\|Lx\|^2=\|x\|^2$, значит равны соответствующие симметрические билинейные формы $\lan Lx,Ly\ran = \lan x,y\ran$.  Расписав это равенство в ортонормированном базисе получаем для $B$ -- матрицы ортогонального  оператора $L$ соотношение $B^{\top} B=E$.
\endproof

\crl В частности, ортогональный оператор  $L$ сохраняет углы между векторами.
\ecrl

\lm Пусть $e_1,\dots,e_n$ --- ортонормированный базис $V$. Линейный оператор $L$, который в базисе $e_i$ имеет матрицу, составленную из столбцов $v_1,\dots,v_n$, является ортогональным тогда и только тогда, когда $v_1,\dots,v_n$ --- ортонормированный базис $K^n$. 
\proof
Пусть $B$ -- матрица $L$, составлена из столбцов $v_i$ в ортогональном базисе $e$. Тогда соотношение $B^{\top}B=E$ эквивалентно ортогональности и нормированности $v_i$.
\endproof
\elm




\dfn Пусть $V$ -- унитарное пространство. Оператор $L\colon V \to V$ называется унитарным, если $\|Lx\|=\|x\|$. 
\edfn

\rm Унитарное отображение $L$ является изометрией. \erm

\thrm Пусть $L \colon V \to V$ -- линейный оператор. Тогда следующие условия эквивалентны:\\
1) $L$ -- унитарный оператор.\\
2) $\lan Lx,Ly\ran=\lan x,y \ran$ для всех $x,y \in V$.\\
3) $L$ переводит любой ортонормированный базис в ортонормированный базис.\\
4) В любом ортонормированном базисе $A$ -- матрица $L$ -- удовлетворяет условию $\ovl{A}^{\top} A = E_n$.\\
5) $L$ переводит некоторый ортонормированный базис в ортонормированный базис.\\
6) В некотором ортонормированном базисе $A$ --  матрица $L$ -- удовлетворяет условию $\ovl{A}^{\top} A = E_n$.
\proof Покажем $1\to 2$. Пусть $x$ и $y$ из $V$, тогда $$||x||^2+ ||y||^2+ 2\Re\lan x,y\ran= ||x+y||^2= ||L(x+y)||^2= ||x||^2+||y||^2+2\Re \lan Lx,Ly\ran.$$
Итого вещественные части скалярного произведения сохраняются. Теперь взяв вместо $x$ вектор $ix$ получаем $\Re \lan ix,y\ran = - \Re i\lan x,y\ran=\Im \lan x,y\ran$ и аналогично $\Re \lan L(ix),Ly\ran =\Im \lan Lx,Ly \ran$ откуда и мнимые части совпадают.
$2\to 3$ ясно.


Покажем, что 3) эквивалентно 4) и 5) эквивалентно 6). Для этого покажем, что если фиксировать ортонормированный базис $e_1,\dots, e_n$, то $L(e_i)$ ортонормирован, тогда и только тогда, когда $A$ -- матрица $L$ удовлетворяет условию $\ovl{A}^{\top}A=E_n$. Для этого необходимо и достаточно заметить, что $$\lan Le_i, Le_j \ran = (\ovl{A}^{\top}A)_{ij}.$$
Действительно $v_j$ -- $j$-ый столбец $A$ составлен из координат $Le_j$. $i$-ая строка $\ovl{A}^{\top}$ равна тогда $\ovl{v}_i^{\top}$. Но тогда $$\lan Le_i,Le_j\ran = \ovl{v}_i^{\top}E_n v_j= \ovl{v}_i^{\top}v_j= (A^{\top}A)_{ij}.$$
Во втором равенстве $E_n$ играет роль матрицы Грама ортонормированного базиса $e_1,\dots, e_n$.
Итак $3 \leftrightarrow 4$ и $5 \leftrightarrow 6$. Из 3) очевидно следует 5). $6\to 1$. $$||Lx||^2=\ovl{e(x)}^{\top} \ovl{A}^{\top} A e(x)= \ovl{e(x)}^{\top} e(x)= ||x||^2.$$
\endproof
\ethrm


\rm Как и в случае ортогональных операторов, условие $\ovl{U}^{\top}U=E_n$ равносильно тому, что столбцы $U$ образуют ортонормированный базис $\mb C^{n}$.
\erm


\dfn Матрица $A\in M_n(\mb R)$ называется ортогональной, если $A^{\top}A=E_n$. Множество всех ортогональных матриц размера обозначается $\O_n(\mb R)$. Такие матрицы описывают все линейные изометрии $\mb R^n$  и поэтому образуют подгруппу в группе $\GL_n(\mb R)$.
\edfn

\dfn Аналогично определяется группа унитарных матриц $\U_n(\mb C)$, как подгруппа в $\GL_n(\mb C)$, состоящая из матриц, удовлетворяющих равенству $\ovl{A}^{\top}A=E_n$.
\edfn

\rm Заметим, что определитель ортогональной матрицы либо плюс, либо минус единица. Определим подгруппу $\SO_n(\mb R) \leq \O_n(\mb R)$ -- специальную ортогональную группу состоящую из матриц $$\SO_n(\mb R)= \{ A \in \O_n(\mb R)\, | \, \det A=1 \}.$$
сохраняющих ориентацию. Это подгруппа индекса 2. Её ещё называют группой вращений.
\erm 


 














\section{Сопряжённые операторы}

Геометрия евклидовых и унитарных пространств, позволяет сводить определённые вопросы про билинейные формы к вопросам про операторы и наоборот. А именно всякому оператору $L$ соответствует билинейная (полуторалинейная) форма $\lan x,Ly\ran$. Или ещё -- форма $\lan Lx,y\ran$ тоже билинейная (полуторалинейная). Посмотрим на эти соответствия поближе.

\lm Пусть $h$ -- билинейная (или полуторалинейная) форма на $V$, тогда существуют единственные операторы $L$ и $L'$, что $h(u,v) = \lan u,Lv \ran= \lan L'u,v\ran $. Точнее, если $e_1,\dots,e_n$ ортонормированный базис, то матрица  матрице $L$ равна матрице $h$ равна, а матрица $L'$ -- $B$ удовлетворяет $B=\ovl{A}^{\top}$, где $A$ -- матрица $h$. 
\proof Покажем вторую часть теоремы, чтобы прояснить, откуда возникает транспонирование и сопряжение в нашей конструкции. Фиксируем ортонормированный базис $e$ пространства $V$, а $A$ -- матрица $h$. Если $L'$ имеет матрицу $B$, то для выполнения равенства полутора(би)линейных форм необходимо и достаточно  $$\ovl{Bx}^{\top}y=\ovl{x}^{\top}\ovl{B}^{\top}y=\ovl{x}^{\top}Ay,$$ 
что равносильно равенству матриц $\ovl{B}^{\top}=A$ или $B=\ovl{A}^{\top}$.
\endproof
\elm

\dfn Пусть $L$ --- оператор на евклидовом пространстве $V$. Сопряжённым оператором к $L$ называется единственный такой оператор $L^*\colon V \to V$, что $\lan L^*x,y\ran=\lan x,Ly \ran$ для всех $x,y \in V$.
\edfn

\crl Сопряжённый оператор к оператору $L$ существует и единственен. Более того, если задан ортонормированный базис $e_1,\dots,e_n$ и $A$ -- матрица $L$, то матрица $L^*$ есть $\ovl{A}^{\top}$.
\ecrl









\lm[Общие свойства]
$(L+T)^*=L^*+T^*$\\
$(LT)^*=T^*L^*$\\
$(\lambda L)^*=\ovl{\lambda}L^*$.\\
$(L^{-1})^*=(L^*)^{-1}$.\\
${L^*}^*=L$.
\proof Фиксируем ортонормированный базис. Тогда все свойства следуют из свойств транспонирования и сопряжения ($\ovl{AB} = \ovl{ A} \,\ovl{B}$).
Однако их можно показать и из определения сопряжённого оператора. Например,
$$\lan x, ABy\ran = \lan A^*x, By\ran = \lan B^*A^*x,y\ran,$$
откуда видно, что $(AB)^*=B^* A^*$. 
\endproof
\elm




\dfn Пусть $L$ оператор на евклидовом или унитарном пространстве $V$ называется самосопряжённым, если $L^*=L$.
\edfn

\dfn Пусть $L$ оператор на евклидовом или унитарном пространстве $V$ называется кососимметрическим (косоэрмитовым), если $L^*=-L$.
\edfn

\rm Пусть $e_1,\dots e_n$ ортонормированный базис, тогда оператор $L$ cамосопряжён тогда и только тогда, когда его матрица в этом базисе удовлетворяет условию $\ovl{L}^{\top}=L$ и косоэрмитов, если $\ovl{L}^{\top}=-L$.
\proof Равенство операторов равносильно равенству их матриц в ортонормированном базисе.
\endproof
\erm



\exm\\
0) Любая симметричная матрица $A=A^{\top}$ на  задаёт самосопряжённый оператор на $\mb R^n$ относительно стандартного скалярного произведения.\\
1) Условие ортогональности оператора можно переписать в виде $L^*L=1$ или, что эквивалентно, $L^*=L^{-1}$. Таким образом, сопряжённый оператор к ортогональному -- это обратный оператор.\\
2) Рассмотрим оператор дифференцирования на пространстве $C^{\infty}_{per=1}(\mb R)$ -- бесконечно дифференцируемых функций, c периодом 1. Тогда сопряжённый оператор к оператору дифференцирования  $f \to \frac{df}{dx}$ есть оператор $-\frac{df}{dx}$.\\
3) Пусть $\mb R[x]$ пространство многочленов и $g(x,y) \in \mb R[x,y]$ -- многочлен. Тогда сопряжённый к оператору $f \to \int_{a}^b f(y)g(x,y)dy$ это оператор $f\to\int_{a}^b f(x)g(x,y)dx$.





\section{Спектральные теоремы}

Как всегда при обсуждении линейных операторов разумно задать вопрос про их собственные числа. Для того, чтобы облегчить нашу задачу заметим, что унитарные, самосопряжённые и косоэрмитовы операторы обладают следующим свойством.

\dfn Пусть $L$ оператор на евклидовом или унитарном пространстве $V$. Оператор $L$ называется нормальным, если $LL^*=L^*L$.
\edfn



Мы остановились на том, что хотим показать, что инвариантные пространства относительно $L$ инварианты относительно $L^*$. Какой бы ни был оператор верен факт.

\lm Если подпространство $U$ инвариантно относительно $L$, то $U^{\bot}$ инвариантно относительно $L^{*}$.
\proof Пусть $v\in U^{\bot}$. Тогда для всех $u\in U$ верно $\lan L^* v, u\ran = \lan v, Lu\ran =0 $ так как $Lu\in U$, что и требовалось.
\endproof
\elm

\lm Пусть $L$ и $T$ два оператора на комплексном векторном пространстве $V$, которые коммутируют между собой, то есть $LT=TL$. Тогда у $L$ и $T$ есть общий собственный вектор. 
\proof Пусть $\lambda$ -- собственное число $L$. Покажем, что $\Ker L -\lambda E$ -- инвариантное пространство, относительно $U$. Действительно, пусть $v\in \Ker L-\lambda E$. Тогда $(L-\lambda E)Tv=T(L-\lambda E)v=0$.

Теперь, так как $\Ker L-\lambda E$ инвариантно, то у $T$ есть собственный вектор на $\Ker L-\lambda E$. Он же собственный для $L$.
\endproof
\elm

\thrm 
Оператор $L$ на  унитарном пространстве $V$ нормален тогда и только тогда, когда существует ортонормированный базис $e_1,\dots,e_n$ в котором матрица $L$ диагональна.
\proof
Доказательство идёт индукцией по размерности. Если оператор нормален, то у $L$ и $L^*$ есть общий собственный вектор $v_1$, так как они коммутируют.

Возьмём к нему ортогональное дополнение $U=\lan v_1\ran^{\bot}$. Это будет инвариантное подпространство для $A$ и $A^*$. При этом ограничение $A^*$ на $U$ -- это сопряжённый к $A|_U$. По индукции это даёт ортонормированный базис для $A$ на $U$, а вместе с $v_1$ базис из собственных векторов на всём $V$.
\endproof
\ethrm


Прежде чем перейти к вещественному случаю покажем несколько лемм.

\lm Если $u$ -- собственный вектор с собственным числом $\lambda$, а $v$ -- собственный вектор с собственным числом $\mu$ нормального оператора $L$ на унитарном пространстве $V$, и $\mu\neq \lambda$, то $u$ ортогонален $v$. 
\elm
\proof Пусть $e_1\dots,e_k, e_{k+1},\dots,e_l, \dots,e_n$ ортонормированный базис собственных векторов $L$, причём вначале стоят  $e_1,\dots,e_k$ -- все собственные вектора с собственным числом $\lambda$ в этом базисе, а затем $e_{k+1},\dots,e_{k+l}$ -- собственные вектора с собственным числом $\mu$. Заметим, что тогда $e_1,\dots,e_k$ -- базис $\Ker L - \lambda E$ (если посмотреть на матрицу) и, аналогично $e_{k+1},\dots,e_{k+l}$ базис $\Ker L- \mu E$. Тогда  $\Ker L - \lambda E$ ортогонален $\Ker L- \mu E$, а вместе с ними вектора $u$ и $v$.
\endproof

\lm Если $x$ -- собственный вектор нормального оператора в унитарном пространстве c собственным числом $\lambda$, то $x$ -- собственный вектор $L^*$ c собственным числом $\ovl{\lambda}$. В частности, ортогональное дополнение к любому собственному вектору инвариантно относительно $L$ и $L^*$.
\elm
\proof Пусть $e_i$ -- ортонормированный базис из собственных векторов $V$. Тогда A - матрица $L$ -- диагональна, как и матрица $L^*$, которая равна $\ovl{A}^{\top}= \ovl{A}$ благодаря диагональности. Заметим, что тогда любой собственный вектор $e_i$ с собственным числом $\lambda$ есть собственный вектор $L^*$ с собственным числом $\ovl{\lambda}$.
Но такие $e_i$ образуют базис $\Ker L- \lambda E$ и
$x=c_1 e_1+\dots +c_k e_k$, где $e_1,\dots, e_k$ базис $\Ker L- \lambda E$. Но тогда $x$ собственный вектор и для $L^*$с собственным числом $\ovl{\lambda}$, как сумма собственных векторов с одинаковым собственным числом $\ovl{\lambda}$.
\endproof


\thrm
Оператор $L$ на евклидовом пространстве $V$ нормален  тогда и только тогда, когда существует ортонормированный базис в котором его матрица $A$ блочно-диагональная, при этом блоки имеют  или размер $1\times 1$ или $2\times 2$ вида
$$\begin{pmatrix}
a  & b\\
-b & a
\end{pmatrix}.$$
Такое представление единственно с точностью до порядка блоков.
\ethrm
\proof
Прежде всего покажем единственность блочной структуры. Пусть $e_i$ -- ортонормированный базис в котором $L$ имеет матрицу в указанном виде. Тогда блочное представление матрицы даёт возможность посчитать характеристический многочлен. Клетка $1\times 1$ даёт множитель $t-\lambda$, то есть соответствует вещественному собственному числу. Клетке $2\times 2$ соответствует множитель в характеристическом многочлене $t^2-2at+a^2+b^2$. Этот многочлен раскладывается на множители $$t^2-2at+a^2+b^2= (t-a-bi)(t-a+bi).$$
Если $b\neq 0$, то корни этого многочлена не вещественны. Тогда количество клеток вида 
$$\begin{pmatrix}
a  & b\\
-b & a
\end{pmatrix}$$
равно кратности корня $a+bi$ у характеристического многочлена.



Перейдём к доказательству основного факта. Индукция по размерности $V$. В случае $\dim V=1$ любая матрица в любом базисе диагональна и говорить не о чем. 



Рассмотрим какой-то ортогональный базис $e$ пространства $V$. Тогда $A$ -- матрица $L$ -- вещественна и удовлетворяет соотношению $\ovl{A}^{\top}A=A\ovl{A}^{\top}$, и, следовательно, задаёт нормальный оператор $\mb C^n \to \mb C^n$.

Теперь рассмотрим два случая. Первый, когда у $A$ есть вещественное собственное число $\lambda$. Значит у системы $A-\lambda E$ есть вещественное ненулевое решение $x\in \mb R^n$. Покажем, что ортогональное дополнение к $x$ в $\mb R^n$ инвариантно относительно $A$. Пусть $e_1,\dots,e_{n-1} $ -- базис ортогонального дополнения (в $\mb R^n$). Но тогда они лежат в $x^{\bot}_{\mb C}$ в $\mb C^n$ и образуют базис этого пространства. Пространство $x^{\bot}_{\mb C}$ инвариантно в $\mb C^n$ относительно $A$, так как  


Рассмотрим собственные вектора $v=e_1 + i e_2 $ и $\ovl{v}=e_1-ie_2$ для собственных чисел $\lambda= a+bi$ и $\ovl{\lambda}=a-bi$. Тогда
$$\lan e_1, e_2\ran = \frac{1}{4} \lan v+\ovl{v}, -i( v-\ovl{v})\ran= \frac{1}{4}(||v||^2-||\ovl{v}||^2)=0.$$
Покажем, что норма $e_i$ равна $\frac{1}{\sqrt{2}}$.
$$||e_1||^2=\lan e_1, e_1\ran = \frac{1}{4} \lan v+\ovl{v},  v+\ovl{v}\ran= \frac{1}{4}(||v||^2+||\ovl{v}||^2)=\frac{1}{2}.$$

Итого, вектора $\sqrt{2}e_1,\sqrt{2}e_2$ независимы ортогональны и нормированы над $\mb C$ и следовательно над $\mb R$. В базисе $e_1,e_2$ матрица $A$ действует как 
$$e_1 \to \frac{1}{2}(\lambda v + \ovl{\lambda}\ovl{v})=ae_1 - be_2 $$
$$e_2 \to \frac{1}{2i}(\lambda v - \ovl{\lambda}\ovl{v})=\frac{1}{2i}( 2i b e_1 + 2i  a e_2) $$
Разумеется, она действует так же и в базисе $\sqrt{2}e_1,\sqrt{2}e_2$. Откуда получаем, что пространство $\lan e_1,e_2\ran $ инвариантно относительно $A$ над $\mb R$, и в базисе $\sqrt{2}e_1,\sqrt{2}e_2$ матрица $A$  действует как  
$$\begin{pmatrix}
a & b\\
-b & a
\end{pmatrix}$$
Покажем, что ортогональное дополнение к $\lan e_1, e_2 \ran$ инвариантно относительно $A$. Для этого достаточно знать, что $\lan e_1, e_2\ran = \lan x, \ovl{x} \ran $ инвариантно относительно $A^{\top}$. Но это так, потому что $x$ и $\ovl{x}$ есть собственные вектора  относительно $A^{\top}$ и следовательно их ортогональное дополнение инвариантно относительно $A$ в $\mb C^n$ и следовательно в $\mb R^n$.

Теперь векторам $\sqrt{2}e_1$ и $\sqrt{2}e_2$ соответствуют $f_1$, $f_2$ в $V$. Пространство $\lan v_1,v_2 \ran$ инвариантно относительно $L$ и матрица ограничения $L$ в этом базисе имеет нужный вид. Так же ортогональное дополнение к $\lan f_1,f_2\ran$ в $V$ инвариантно относительно $L$ и, следовательно, можно применить предположение индукции к $\lan f_1,f_2\ran^{\bot}$. 
\endproof






Теперь можно легко получить характеризацию самосопряжённых, унитарных, вещественных ортогональных, и кососимметрических операторов.

\thrm Пусть $L$ -- оператор в евклидовом (унитарном) пространстве $V$. Тогда $L$ -- самосопряжённый тогда и только тогда, когда существует ортонормированный базис $V$ состоящий из собственных векторов оператора $L$ и все собственные числа $L$ -- вещественны.
\proof Пусть $L$ оператор на унитарном пространстве. Возьмём ортонормированный базис из его собственных векторов и распишем условие самосопряжённости. Оно означает, что $\ovl{A}^{\top}=A$. Но $A$ диагональна и на диагонали стоят собственные числа. Итого на них получается уравнение $\ovl{\lambda}=\lambda$, что гарантирует их вещественность.

Пусть теперь  $L$ оператор на евклидовом пространстве. Тогда есть ортонормированный базис, в котором матрица $L$ составлена из блоков $1\times 1$ или $2\times 2$ вида 
$$\pmat a& b \\ -b & a \epmat.$$
Но матрица $L$ в ортогональном базисе должна быть симметрична. Отсюда $b=-b$, то есть $b=0$, что и требовалось.
\endproof
\ethrm



\zd Докажите спектральную теорему в случае самосопряжённого оператора напрямую.
\ezd

\thrm Оператор $L$ -- унитарный тогда и только тогда, когда его собственные числа по модулю равны 1 и существует ортонормированный базис из собственных векторов. 
\proof Рассмотрим ортонормированный базис из собственных векторов $e$. Матрица $A$ оператора $L$ в этом базисе диагональна и удовлетворяет соотношению $\ovl{A}^{\top}A=E_n$. Для собственных чисел $A$ это означает, что $$|\lambda|^2=\ovl{\lambda}\lambda=1.$$ 
\endproof
\ethrm






\thrm Оператор $L$ на евклидовом пространстве $V$ ортогональный  тогда и только тогда, когда существует ортонормированный базис в котором матрица $A$ блочно-диагональная, при этом блоки имеют размер $1$ или $2$ и блоки размера $1$ состоят из $\pm 1$, а блоки размера 2 имеют вид
$$\begin{pmatrix}
\cos \varphi & \sin \varphi\\

-\sin \varphi &\cos \varphi
\end{pmatrix}.$$
\ethrm
\proof
По вещественной версии спектральной теоремы для самосопряжённых операторов есть ортонормированный базис в котором матрица $L$ состоит из блоков $1\times 1$ или $2\times 2$ вида 
$$\pmat a& b \\ -b & a \epmat.$$
Но матрица $L$ в ортогональном базисе должна быть ортогональной и, следовательно, вещественной унитарной. Откуда получаем, что её собственные числа по модулю равны 1. Если эти числа вещественные, то они равны $\pm 1$. Если же они не вещественные, то имеют вид $a+bi=\cos \ffi + i \sin \ffi$, что даёт необходимый вид для $a$ и $b$.
\endproof




\zd Покажите аналогичные теоремы для кососимметричных и косоэрмитовых операторов.
\ezd

\crl Пусть $L$ -- ортогональный оператор на $\mb R^3$. Если $\det L=-1$, то в некотором ортогональном базисе матрица $L$ имеет вид 
$$\begin{pmatrix}
- 1 &&\\
&\cos \varphi & \sin \varphi\\
&-\sin \varphi &\cos \varphi
\end{pmatrix}$$
Если же, $\det L=1$, то в некотором ортогональном базисе матрица $L$ имеет вид 
$$\begin{pmatrix}
1 &&\\
&\cos \varphi & \sin \varphi\\
&-\sin \varphi &\cos \varphi
\end{pmatrix}.$$
То есть любой элемент $\SO_3(\mb R)$ есть вращение относительно некоторой прямой. 
\ecrl

О матрицах из $\SO_n(\mb R)$ мы поговорим позже, а пока приведём пример преобразований не сохраняющих ориентацию. Самый геометрически наглядный пример -- это отражение относительно гиперплоскости $H$. Чтобы получить линейное, а не аффинное преобразование потребуем, чтобы $H$ проходила через 0, то есть была линейным подпространством. Тогда все точки из $H$ задаются тем, что они ортогональны некоторому вектору $v$ (любая гиперплоскость есть ядро любого линейного функционала, а любой линейный функционал имеет вид $\lan \_ ,v \ran$ для некоторого $v$ из-за невырожденности скалярного произведения). Не умаляя общности потребуем, чтобы вектор $v$ был нормирован. Тогда отражение относительно $H$ действует следующим образом:
$$u \to u - 2 \lan u,v\ran v.$$
Это очевидно линейное преобразование. Покажем, что оно ортогонально. Для этого проще всего найти ортонормированный базис из собственных векторов и проверить, что все с.ч. равны $\pm 1$. Первый кандидат -- это $v$, который есть собственный вектор с собственным числом 1. Остальные вектора -- это $e_1,\dots, e_{n-1}$ -- ортонормированный базис $H$. Это вектора с собственным числом 1. 


\end{document}
